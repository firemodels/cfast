
\chapter{Verification}

The terms {\em verification} and {\em validation} are often used interchangeably to mean the process of checking the accuracy of a numerical model. For many, this entails comparing model predictions with experimental measurements. However, there is now a fairly broad-based consensus that comparing model and experiment is largely what is considered {\em validation}. So what is {\em verification}? ASTM~E~1355~\cite{ASTM:E1355}, ``Standard Guide for Evaluating the Predictive Capability of Deterministic Fire Models,'' defines verification as
\begin{quote}
The process of determining that the implementation of a calculation method accurately represents the developer's conceptual description of the calculation method and the solution to the calculation method.
\end{quote}
and it defines validation as
\begin{quote}
The process of determining the degree to which a calculation method is an accurate representation of the real world from the perspective of the intended uses of the calculation method.
\end{quote}
Simply put, verification is a check of the math; validation is a check of the physics. If the model predictions closely match the results of experiments, using whatever metric is appropriate, it is assumed by most that the model suitably describes, via its mathematical equations, what is happening. It is also assumed that the solution of these equations must be correct. So why do we need to perform model verification? Why not just skip to validation and be done with it? The reason is that rarely do model and measurement agree so well in all applications that anyone would just accept its results unquestionably. Because there is inevitably differences between model and experiment, we need to know if these differences are due to limitations or errors in the numerical solution, or the physical sub-models, or both.

Whereas model validation consists mainly of comparing predictions with experimental measurements, as documented later in this guide, model verification consists of a much broader range of activities, from checking the computer program itself to comparing calculations to analytical (exact) solutions to understanding the impact on model outputs from a range of different model inputs.


\section{Thermal Equlibrium}

For most of the examples presented in this section, the same basic geometry is used, a single 5~m by 5~m by 5~m compartment.

\subsection{Temperature Equilibrium via Heat Conduction}

As a simple test of the energy balance, raising the external temperature of the base case compartment from an initial condition of 20~\degc to 25~\degc allows the internal temperature to equilibrate to the exterior. From the ideal gas law, the pressure inside the compartment is expected to rise to
\begin{equation}
   P_{\rm final} = P_{\rm initial} \; \frac{T_{\rm final}}{T_{\rm initial}} = 101325 \; {\rm Pa} \times \frac{298.15 \; {\rm K}}{293.15 \; {\rm K}} = 103027.78 \; {\rm Pa} \label{eq:Temperature_Equilibrium}
\end{equation}
or a pressure rise of 1728.21, matching the output from CFAST.  Figure \ref{fig:Temperature_Equilibrium} shows the simulated conditions for this test.

\begin{figure}[!ht]
\begin{tabular*}{\textwidth}{l@{\extracolsep{\fill}}r}
\includegraphics[width=3.0in]{FIGURES/Verification/basic_tempequilib_temp} &
\includegraphics[width=3.0in]{FIGURES/Verification/basic_tempequilib_pres}
\end{tabular*}
\caption[Results of the test case {\ct basic\_tempequilib.in}]{Interior temperature and pressure in equilibrium with exterior in the case {\ct basic\_tempequilib.in}.}
\label{fig:Temperature_Equilibrium}
\end{figure}

\subsection{Temperature Equilibrium via a Winodw}

Now an open window is added to the compartment, with an with an exterior temperature of 25~\degc. Figure~\ref{fig:Temperature_Equilibrium_With_Window} shows the interior conditions coming into equilibrium with the exterior.

\begin{figure}[!ht]
\begin{tabular*}{\textwidth}{l@{\extracolsep{\fill}}r}
\includegraphics[width=3.0in]{FIGURES/Verification/basic_tempequilib_window_temp} &
\includegraphics[width=3.0in]{FIGURES/Verification/basic_tempequilib_window_pres}
\end{tabular*}
\caption[Results of the test case {\ct basic\_tempequilib\_window.in}]{Interior temperature and pressure in equilibrium with exterior in the case {\ct basic\_tempequilib\_window.in}.}
\label{fig:Temperature_Equilibrium_With_Window}
\end{figure}

\subsection{Temperature Equilibrium via a Winodw at a High Elevation}

With the exterior temperature still set to 25~\degc, the elevation is raised to 1500~m, approximately the average elevation of Idaho.  Since CFAST calculations are relative to the exterior ambient, conditions are expected to be identical to the previous examples and equilibrate to those of the exterior. Figure \ref{fig:Temperature_Equilibrium_Elevation} shows the simulated conditions for the test case.

\begin{figure}[!ht]
\begin{tabular*}{\textwidth}{l@{\extracolsep{\fill}}r}
\includegraphics[width=3.0in]{FIGURES/Verification/basic_tempequilib_window_elevation_temp} &
\includegraphics[width=3.0in]{FIGURES/Verification/basic_tempequilib_window_elevation_pres}
\end{tabular*}
\caption[Results of the test case {\ct basic\_tempequilib\_window\_elevation.in}]{Interior temperature and pressure in equilibrium with exterior in the case {\ct basic\_tempequilib\_window\_elevation.in}.}
\label{fig:Temperature_Equilibrium_Elevation}
\end{figure}


\section{Conservation of Mass}

\label{mass_conservation}
\subsection{Case 1: Fire Within a Single Compartment}
\label{sec:spec1}
A natural gas fire burns in a sealed compartment of dimension 5~m by 6~m by 3~m. The heat release rate ramps up linearly to 1~kW in 30~s, then remains steady for 5~min, and then ramps down linearly to 0 in 30~s. The total energy released is 330~kJ, and the total mass of fuel consumed is
\begin{equation}
  \frac{ 330 \; {\rm kJ} }{ 50000 \; {\rm kJ/kg} } = 0.0066 \; {\rm kg}
\end{equation}
The combustion chemistry is given by
\begin{equation}
   \mathrm{CH_4 + 2 \, O_2 \to CO_2 + 2 \, H_2O}
\end{equation}
The molecular weight of CH$_4$ is 16~g/mol and CO$_2$ is 44~g/mol; thus, the mass of CO$_2$ produced by the fire is
\begin{equation}
   m_{\rm CO_2} = 0.0066 \; {\rm kg} \times  \frac{ 44 \; {\rm g/mol} }{ 16 \; {\rm g/mol} } = 0.01815 \; {\rm kg}
\end{equation}
The molecular weight of H$_2$O is 18~g/mol; thus, the mass of H$_2$O produced by the fire is
\begin{equation}
   m_{\rm H_2O} = 0.0066 \; {\rm kg} \times  \frac{ 2(18) \; {\rm g/mol} }{ 16 \; {\rm g/mol} } = 0.01485 \; {\rm kg}
\end{equation}
The experimentally determined mole fractions of O$_2$, CO$_2$ and H$_2$O in the upper layer are 0.2069, 0.00012 and 0.00024, respectively. This leaves the mole fraction of N$_2$ to be 0.7927. These mole fractions can be converted to mass fractions by
\begin{equation}
Y_k = \frac{X_{k}M_{k}}{\sum_{i=1}^N X_{i}M_{i}}
\end{equation}
Given that the temperature of the upper layer is 297~K, the volume of the upper layer is 81~m$^3$ and the pressure in the compartment is 102665~Pa, the mass of the upper layer can be calculated by
\begin{equation}
m_{u} = \frac{PV}{T\frac{R}{M_{gas}}}
\end{equation}
To find the mass of CO$_2$ and H$_2$O produced in the experiment, the following equation must be applied
\begin{equation}
m_{k} = m_{u}Y_{k}
\end{equation}
Figure~\ref{specmass1} shows the result in calculating the mass of both CO$_2$ and H$_2$O to be about 0.2~\% greater than expected.

\begin{figure}[!ht]
\centering
\includegraphics[width=3.0in]{FIGURES/Verification/species_mass_1}
\caption[Results of the test case {\ct species\_mass\_1.in}]{Expected and predicted masses of CO$_2$ and H$_2$O for the case {\ct species\_mass\_1.in}.}
\label{specmass1}
\end{figure}


\subsection{Case 2: Fire Within a Two Compartment System Connected by a Wall Vent}
\label{sec:spec2}
The same natural gas fire described in section~\ref{sec:spec1} burns in a compartment of dimension 2~m by 5~m by 8~m which is connected to another compartment of dimension 5~m by 3~m by 8~m. A doorway connects the compartments, which has a width of 1~m and a height of 6~m. Because the fire and the fuel source have not changed, the theoretical calculations for the mass of CO$_2$ and H$_2$O produced will remain the same. The remaining portion of the problem is approached in the same manner, but since there are two compartments, the mass of CO$_2$ and H$_2$O produced in each layer of each compartment must be individually calculated and then summed together to produce the net yields of CO$_2$ and H$_2$O.
Figure~\ref{specmass2} shows the result in calculating the mass of both CO$_2$ and H$_2$O to be about 0.2~\% greater than expected.
\begin{figure}[!ht]
\centering
\includegraphics[width=3.0in]{FIGURES/Verification/species_mass_2}
\caption[Results of the test case {\ct species\_mass\_1.in}]{Expected and predicted masses of CO$_2$ and H$_2$O for the case {\ct species\_mass\_2.in}.}
\label{specmass2}
\end{figure}

\subsection{Case 3: Fire Within a Two Compartment System Connected by a Ceiling Vent}
The same natural gas fire described in section~\ref{sec:spec1} burns in a compartment of dimension 9~m by 5~m by 4~m which is connected to another compartment of dimension 9~m by 5~m by 2~m. The compartments are placed such that the second one is located directly above the first one. There is a sqaure ceiling vent between the compartments that has an area of 4~m$^2$. This problem is approached in the same exact manner as in section~\ref{sec:spec2} because the only difference between the two scenarios is the specific alignment of the compartments.
Figure~\ref{specmass2} shows the result in calculating the mass of both CO$_2$ and H$_2$O to be about 0.2~\% greater than expected.
\begin{figure}[!ht]
\centering
\includegraphics[width=3.0in]{FIGURES/Verification/species_mass_3}
\caption[Results of the test case {\ct species\_mass\_3.in}]{Expected and predicted masses of CO$_2$ and H$_2$O for the case {\ct species\_mass\_3.in}.}
\label{specmass3}
\end{figure}

\section{Energy Balance}

\subsection{A Fire in a Single, Sealed Compartment with a Single Zone}

A 100~kW methane fire burns in a sealed compartment with no ventilation, adiabatic walls, and no radiative emission. A single zone simulation is run in which it is assumed that the entire volume is taken up by the upper layer.  The governing equation for the temperature is
\be
   \frac{{\rm d} (c_v \, m \, T)}{{\rm d}t} = \dot{Q}  \quad \Longrightarrow \quad \frac{{\rm d}T}{{\rm d}t} = \frac{ \dot{Q} - c_v \, \dot{m}_{\rm f} \, T }{c_v \, m}
\ee
where $\dm_{\rm f}$ is the fuel flow rate and $m$ is the total mass. Figure \ref{fig:Analytical_Closed_Compartment} includes comparisons of the temperature and pressure as predicted by CFAST and a simple second-order accurate ODE solver. The difference is due to the way CFAST handles a single layer calculation while maintaining its default equation set that includes both a lower and upper layer.

\begin{figure}[!ht]
\begin{tabular*}{\textwidth}{l@{\extracolsep{\fill}}r}
\includegraphics[width=3.0in]{FIGURES/Verification/sealed_test_temp} &
\includegraphics[width=3.0in]{FIGURES/Verification/sealed_test_pres}
\end{tabular*}
\caption[Results of the test case {\ct sealed\_test.in}]{Calculated and expected temperature and pressure rise for a fire in a closed compartment. The case is called {\ct sealed\_test.in}.}
\label{fig:Analytical_Closed_Compartment}
\end{figure}



\section{Ventilation}

Two identical 5~m x 5~m x 5~m compartments are stacked on each other.  A 1~m$^2$ mechanical vent is added on the front face of compartment one, the shared ceiling/floor between compartment one and two, and the rear wall of compartment two.  The flow rate is set to 0.1~m$^3$/s or 0.12~kg/s of air.  The mass flow through each of these vents is expected to be the same because the flow rate in is constant and there is no change in temperature.   Figure~\ref{fig:Mechanical_Flow_Two_Compartments} shows vent flows for all vents in the simulation.

\begin{figure}[!ht]
\begin{center}
\includegraphics[width=3.0in]{FIGURES/Verification/ceiling_mechvent}
\caption[Results of the test case {\ct ceiling\_mechvent.in}]{Mass flow rates of a mechanical ventilation system connecting two compartments {\ct ceiling\_mechvent.in}.}
\label{fig:Mechanical_Flow_Two_Compartments}
\end{center}
\end{figure}

\subsection{Case 2: Measuring Temperature and Pressure Changes as Air Flows Through a Multi-Compartment System}
A chamber is made by combining four 4~m by 4~m by 4~m compartments together. The chamber is arranged such that two compartments are placed adjacent to one another and the following two compartments are placed directly on top of the first two. Air flows into the first compartment, on the ground floor, at a rate of 1~m$^3$/s and then continues to the second compartment after passing through a doorway. The air then travels to the third compartment by passing through a ceiling vent with an area of 3~m$^2$ and continues to compartment four through another doorway. Air is finally extracted from compartment four at a rate of 1~m$^3$/s. Figure~\ref{ventilation_2TP} shows how temperature and pressure change as the air flows through the fourth compartment.



\begin{figure}[!ht]
\begin{tabular*}{\textwidth}{l@{\extracolsep{\fill}}r}
\includegraphics[width=3.0in]{FIGURES/Verification/ventilation_2_temperature} &
\includegraphics[width=3.0in]{FIGURES/Verification/ventilation_2_pressure}
\end{tabular*}
\caption[Results of the test case {\ct ventilation\_2.in}]{Expected and CFAST calculated values for pressure and temperature of the fourth compartment in the chamber {\ct ventilation\_2.in}.}
\label{ventilation_2TP}
\end{figure}



