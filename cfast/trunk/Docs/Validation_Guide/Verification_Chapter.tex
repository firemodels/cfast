
\chapter{Verification}

The terms {\em verification} and {\em validation} are often used interchangeably to mean the process of checking the accuracy of a numerical model. For many, this entails comparing model predictions with experimental measurements. However, there is now a fairly broad-based consensus that comparing model and experiment is largely what is considered {\em validation}. So what is {\em verification}? ASTM~E~1355~\cite{ASTM:E1355}, ``Standard Guide for Evaluating the Predictive Capability of Deterministic Fire Models,'' defines verification as
\begin{quote}
The process of determining that the implementation of a calculation method accurately represents the developer's conceptual description of the calculation method and the solution to the calculation method.
\end{quote}
and it defines validation as
\begin{quote}
The process of determining the degree to which a calculation method is an accurate representation of the real world from the perspective of the intended uses of the calculation method.
\end{quote}
Simply put, verification is a check of the math; validation is a check of the physics. If the model predictions closely match the results of experiments, using whatever metric is appropriate, it is assumed by most that the model suitably describes, via its mathematical equations, what is happening. It is also assumed that the solution of these equations must be correct. So why do we need to perform model verification? Why not just skip to validation and be done with it? The reason is that rarely do model and measurement agree so well in all applications that anyone would just accept its results unquestionably. Because there is inevitably differences between model and experiment, we need to know if these differences are due to limitations or errors in the numerical solution, or the physical sub-models, or both.

Whereas model validation consists mainly of comparing predictions with experimental measurements, as documented later in this guide, model verification consists of a much broader range of activities, from checking the computer program itself to comparing calculations to analytical (exact) solutions to understanding the impact on model outputs from a range of different model inputs.


\section{Simple Analytical Comparisons}

Certain CFAST sub-models address phenomena that have analytical solutions, for example, one dimensional heat conduction through a solid or pressure increase in a sealed or slightly leaky compartment as a result of a fire or fan.  The developers of CFAST use analytical solutions to test sub-models to verify the correctness of the coding of the model as part of the development. This section provides an overview of the verification testing conducted with each change of the model to verify the basic underlying principles of the model, the mass and energy balances.

\subsection{Thermal Equlibrium}


For most of the examples presented in this section, the same basic geometry is used, a single 5~m by 5~m by 5~m compartment. As a simple test of the energy balance, raising the external temperature of the base case compartment from an initial condition of 20~\degc to 25~\degc allows the internal temperature to equilibrate to the exterior. From the ideal gas law, the pressure inside the compartment is expected to rise to
\begin{equation}
   P_{\rm final} = P_{\rm initial} \; \frac{T_{\rm final}}{T_{\rm initial}} = 101300 \; {\rm Pa} \times \frac{298.15 \; {\rm K}}{293.15 \; {\rm K}} = 103027.78 \; {\rm Pa} \label{eq:Temperature_Equilibrium}  
\end{equation}
or a pressure rise of 1727.78, matching the output from CFAST.  Figure \ref{fig:Temperature_Equilibrium} shows the simulated conditions for this test.

\begin{figure}[!ht]
\begin{tabular*}{\textwidth}{l@{\extracolsep{\fill}}r}
\includegraphics[width=3.0in]{FIGURES/Verification/Temperature_Equilibrium_Test} &
\includegraphics[width=3.0in]{FIGURES/Verification/Pressure_Change_Temperature_Equilibrium_Test}
\end{tabular*}
\caption[Results of the test case {\ct basic\_tempequilib.in}]{Interior temperature and pressure in equilibrium with exterior in the case {\ct basic\_tempequilib.in}.} 
\label{fig:Temperature_Equilibrium}
\end{figure}

Now an open window is added to the compartment, with an with an exterior temperature of 25~\degc. Figure~\ref{fig:Temperature_Equilibrium_With_Window} shows the interior conditions coming into equilibrium with the exterior.

\begin{figure}[!ht]
\begin{tabular*}{\textwidth}{l@{\extracolsep{\fill}}r}
\includegraphics[width=3.0in]{FIGURES/Verification/Temperature_Equilibrium_Test_With_Window} &
\includegraphics[width=3.0in]{FIGURES/Verification/Pressure_Change_Temperature_Equilibrium_Test_With_Window}
\end{tabular*}
\caption[Results of the test case {\ct basic\_tempequilib\_window.in}]{Interior temperature and pressure in equilibrium with exterior in the case {\ct basic\_tempequilib\_window.in}.}
\label{fig:Temperature_Equilibrium_With_Window}
\end{figure}

With the exterior temperature still set to 25~\degc, the elevation is raised to 1500~m, approximately the average elevation of Idaho.  Since CFAST calculations are relative to the exterior ambient, conditions are expected to be identical to the previous examples and equilibrate to those of the exterior. Figure \ref{fig:Temperature_Equilibrium_Elevation} shows the simulated conditions for the test case.

\begin{figure}
\begin{tabular*}{\textwidth}{l@{\extracolsep{\fill}}r}
\includegraphics[width=3.0in]{FIGURES/Verification/Temperature_Equilibrium_Elevation_Change} &
\includegraphics[width=3.0in]{FIGURES/Verification/Pressure_Change_Temperature_Equilibrium_Test_Elevation}
\end{tabular*}
\caption[Results of the test case {\ct basic\_tempequilib\_window\_elevation.in}]{Interior temperature and pressure in equilibrium with exterior in the case {\ct basic\_tempequilib\_window\_elevation.in}.}
\label{fig:Temperature_Equilibrium_Elevation}
\end{figure}

\subsection{Air Injection}

To test both the mass and energy balance, consider the base case with the addition of a fixed amount of ambient air (injected through a mechanical vent.  The vent is set to deliver 0.1 m$^3$/s into the compartment for 10 s. After this, the vent is closed.  Over the 10 s period, an additional 1 m$^3$ of air at ambient temperature is added.  Since CFAST, by default, closes vent over a 1 s time period, an additional 0.05 m$^3$ is also added. With a compartment volume of 125~m$^3$ at 20 \degc, the initial mass of air is 150.5 kg or 5.195 moles. With the added mass, the final mass of air in the compartment is 151.7 kg or 5.239 moles. From the CFAST output, the final temperature of the compartment rises slightly due to the pressure increase to 20.98 \degc. From the ideal gas law, the expected pressure rise can be calculated as
\begin{equation}
   P_{\rm final} = P_{\rm initial} \; \frac{m_{\rm final} \, T_{\rm final}}{m_{\rm initial} \, T_{\rm initial}} = 101300 \; {\rm Pa} \times \frac{151.7 \; {\rm kg} \times 294.13 \; {\rm K}}{150.5 \; {\rm kg} \times 293.15 \; {\rm K}} = 102449.1 \; {\rm Pa} \label{eq:Temperature_Equilibrium}
\end{equation}
or a pressure rise of 1149.1~Pa. Figure \ref{fig:Added_Mass_Test} shows the simulated conditions for this test.

\begin{figure}[!ht]
\begin{tabular*}{\textwidth}{l@{\extracolsep{\fill}}r}
\includegraphics[width=3.0in]{FIGURES/Verification/Ambient_Temperature_Added_Mass} &
\includegraphics[width=3.0in]{FIGURES/Verification/Ambient_Pressure_Added_Mass}
\end{tabular*}
\caption[Results of the test case {\ct Added\_Mass.in}]{Compartment temperature and pressure due to a specified injected of air in the case {\ct Added\_Mass.in}.} 
\label{fig:Added_Mass_Test}
\end{figure}

\subsection{A Simple Fire}

A model examining heat added to a system can be demonstrated with a test case containing a constant 100 kW fire.  With non-conducting surface and no ventilation, the heat and mass released by the fire (and added to the compartment) can be determined.  Here we use a single zone simulation to simplify the calculations (CFAST simply assumes the entire volume is taken up by the upper layer).  Densities are obtained using the calculated temperature. The energy and mass added to the system can be calculated as

\begin{eqnarray}
M_0 &=& V \cdot \rho_{ambient} \nonumber \\
 &=& 150 \cdot 1.195 \nonumber \\
 &=& 149.39 \text{\ kg} \nonumber \\
M &=& M_0 + \dm_f \cdot t \\
E_0 &=& M_0 \cdot c_v \cdot T_{ambient} \nonumber \\
 &=& 149.39 \cdot 1012/1.4 \cdot 293.15 \nonumber \\
 &=& 31.65  \text{\ MJ} \nonumber \\
E &=& E_0 + Q_f \cdot t + \dm_f \cdot c_v \cdot T
\end{eqnarray}
where $M_0$ is the initial mass of air in the compartment, $V$ is the compartment volume, $\rho_{ambient}$ is the air density at ambient conditions, $M$ is the mass of gases in the compartment at time $t$\, $\dm_f$ is the pyrolysis rate of the burning fuel, $E_0$ is the initial internal energy of the system, $c_v$ is the heat capacity of air at constant volume, $T_{ambient}$ is the temperature of the compartment at ambient conditions, $E$ is the internal energy of the system at time $t$, $Q_f$ is the convective heat release rate of the fire, and $T$ is the temperature of the compartment at time $t$.

Finally, the temperature of the compartment can be calculated from the definition of internal energy of the system

\begin{equation}
E = M \cdot c_v \cdot T \text{, or, rearranging, } T = \frac{E}{c_v \cdot M}
\end{equation}

Figure \ref{fig:Analytical_Closed_Compartment} shows the comparison of the calculated and CFAST results for this test. The average difference between the calculations is approximately 0.01 \%, with the difference due to the way CFAST handles a single layer calculation while maintaining its default equation set that includes both a lower and upper layer.

\begin{figure}
\begin{center}
\includegraphics[width=3.0in]{FIGURES/Verification/Sealed_Compartment_Temperature}
\caption{Comparison of CFAST calculations and analytical solution for a 100 kW fire in a closed 5 m x 5 m x 5 m compartment.  CFAST verification file sealed\_test.in.}
\label{fig:Analytical_Closed_Compartment}
\end{center}
\end{figure}



\subsection{Ventilation}

All ambient conditions are returned to the standard base case.  A mechanical vent is added to the test case with a flow rate of 0.1~m$^3$/s for 10~s. Figure \ref{fig:Mechanical_Vent_Cutoff} shows the pressure building in the compartment for the default fan cutoff pressure of 200 Pa to 300 Pa compared to a raised fan cutoff pressure of 2000 Pa.

\begin{figure}
\begin{center}
\includegraphics[width=3.0in]{FIGURES/Verification/Pressure_Dropoff_Test}
\caption{Model with mechanical vent and non-inhibited drop off pressure of 2000 Pa versus model with mechanical vent and interfering drop off pressure of 200~Pa. CFAST verification files basic\_mechvent\_n.csv and basic\_mechvent\_dropoff\_n.csv.}
\label{fig:Mechanical_Vent_Cutoff}
\end{center}
\end{figure}

Two identical 5~m x 5~m x 5~m compartments are stacked on each other.  A 1~m$^2$ mechanical vent is added on the front face of compartment one, the shared ceiling/floor between compartment one and two, and the rear wall of compartment two.  The flow rate is set to 0.1~m$^3$/s or 0.12~kg/s of air.  The mass flow through each of these vents is expected to the same because the flow rate in is constant and there is no change in temperature.   Figure \ref{fig:Mechanical_Flow_Two_Compartments} shows vent flows for all vents in the simulation.

\begin{figure}
\begin{center}
\includegraphics[width=3.0in]{FIGURES/Verification/Mass_Flow_Test_Mechanical_Vent}
\caption{Side-by-side base case compartments with mechanical vents added from outside to compartment one, compartment one to compartment two, and compartment two to outside.  CFAST verification file basic\_connection\_floorceiling\_mechvent.in.}
\label{fig:Mechanical_Flow_Two_Compartments}
\end{center}
\end{figure}





