\chapter{Sensitivity of the Model}

A sensitivity analysis considers the extent to which uncertainty in model inputs influences model 
output.  For a sensitivity analysis, this uncertainty includes not only that inherent in the input of 
data for specific scenarios by the model user, but also uncertainty in empirical data or numerical 
parameters in the model such as the time step size used by the model to obtain a solution. 

Among the purposes for conducting a sensitivity analysis are to determine 
\begin{itemize}
\item the important variables in the models, 
\item the computationally valid range of values for each input variable, and 
\item the sensitivity of output variables to variations in input data. 
\end{itemize}

Conducting a sensitivity analysis of a complex model is not a simple task and it will differ 
depending on the application. CFAST typically requires the user to provide numerous input 
parameters that describe the building geometry, compartment connections, construction 
materials, and description of one or more fires. 