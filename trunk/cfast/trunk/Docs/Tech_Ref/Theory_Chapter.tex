\chapter{The Basic Transport Equations}
\label{sec:Theory_Chapter}

The equations used in CFAST take the form of an initial value problem for a system of ordinary differential equations. These equations are derived from the conservation laws of mass and energy (equivalently the first law of thermodynamics) and the ideal gas law. These equations predict as functions of time quantities such as pressure, layer height and temperatures given the gains and losses of mass and energy in the two layers. The assumption of a zone model is that properties such as temperature can be approximated throughout a control volume by an average value. Many formulations based upon these assumptions can be derived. Though equivalent mathematically, these formulations differ in their numerical solution.

The exchange of mass and enthalpy between zones is due to physical phenomena such as fire plumes, natural and forced ventilation, convective and radiative heat transfer, and so on. For example, a vent exchanges mass and enthalpy between zones in connected rooms, a fire plume typically adds heat to the upper layer and transfers entrained mass and enthalpy from the lower to the upper layer, and convection transfers enthalpy from the gas layers to the surrounding walls.

\section{Thermodynamic Variables}


It is assumed that each compartment is divided into two control volumes, a relatively hot upper layer and a relatively cool lower layer, as illustrated in Fig.~\ref{fig:Control_Volumes}. The gas temperature and density are assumed constant in each layer. The compartment as a whole is assumed to have a single value of pressure, $P$. It is also assumed that all thermodynamic parameters are constant. The specific heat at constant volume and at constant pressure, $c_v$ and $c_p$, the universal gas constant, $R$, and the ratio of specific heats, $\gamma$, are related by $\gamma = c_p / c_v$ and $R = c_p- c_v$.  For ambient air, $c_p \approx 1$~kJ/(kg $\cdot$ K) and $\gamma = 1.4$.
\begin{figure}[h]
\begin{center}
\includegraphics[width=\textwidth]{FIGURES/Theory/Control_Volumes}\\
\end{center}
\caption{Schematic of control volumes in a two-layer zone model.}
 \label{fig:Control_Volumes}
\end{figure}

The fundamental equations start with the conservation of mass. The change of mass in each layer is expressed:
\be
   \dbydt{m_i} = \dot m_i  \label{mass_con}
\ee
where $\dm_i$ represents the sum of all mass flow terms, such as plume mass entrainment and ventilation, entering and leaving the control volume. Conservation of energy takes the form of the first law of thermodynamics, which states that the rate of increase of internal energy plus the rate at which the layer does work by expansion is equal to the rate at which enthalpy is added to the gas:
\be
   \dbydt{(c_v m_i T_i)} +  P \dbydt{V_i} =  \dot h_i \label{eq:first_law}
\ee
The enthalpy source term, $\dot h_i$, consists of the fire's heat release rate, conduction losses to walls, and radiation exchange. The layer temperature and mass are related to the layer volume and compartment pressure via the ideal gas law:
\be
  P \, V_i = m_i \, R \, T_i \label{EoS}
\ee
A system of ordinary differential equations for the compartment pressure, upper layer volume, and layer temperatures can be derived from these three basic principles:
\begin{eqnarray}
\dbydt{P} &=& \frac{{\gamma-1}}{{V}} \left( \dhl + \dhu \right)  \label{eq1} \\[.1in]
\dbydt{\Vu} &=& \frac{1}{P \gamma} \left( (\gamma-1) \, \dhu - \Vu \dbydt{P} \right) \label{eq2} \\[.1in]
\dbydt{\Tu} &=& \frac{1}{c_p \, m_{\rm u}} \left( \dhu - c_p \, \dmau \, \Tu + \Vu \dbydt{P} \right) \label{eq3} \\[.1in]
\dbydt{\Tl} &=& \frac{1}{c_p \, m_{\rm l}} \left( \dhl - c_p \, \dmau \, \Tl + \Vl \dbydt{P} \right) \label{eq4}
\end{eqnarray}
Equation~(\ref{eq1}) is derived by summing Eq.~(\ref{eq:first_law}) for the upper and lower layer, applying the equation of state~(\ref{EoS}), and noting that the time derivative of the total room volume is zero. Equation~(\ref{eq2}) is derived by rearranging Eq.~(\ref{eq:first_law}) applied to the upper layer and applying the equation of state to replace the internal energy term. Equations~(\ref{eq3}) and (\ref{eq4}) are derived from Eq.~(\ref{eq:first_law}) applied to each layer, along with substitutions of the equation of state and mass conservation equation.

As discussed in Refs.~\cite{Forney:1994} and \cite{Rehm:1992}, Eqs.~(\ref{eq1}) through (\ref{eq4}) are stiff, meaning that the pressure adjusts to changing conditions more quickly than the other variables. Runge-Kutta methods or predictor-corrector methods such as Adams-Bashforth require prohibitively small time steps in order to track the short time scale phenomena (pressure in our case). Methods that calculate the Jacobian (or at least approximate it) have a much larger stability region for stiff problems and are thus more successful at their solution.

\section{Species Transport}

The equation of state, (\ref{EoS}), assumes that the molecular weight of the gaseous mixture throughout the domain is approximately that of air, 29~g/mol. However, CFAST does track the products of combustion and the depletion of oxygen in each zone. At the start of the simulation, the composition of each layer is set to ambient conditions. The initial temperature is specified by the user. The oxygen mass fraction is 23~\% (21~\% volume fraction) and the nitrogen mass fraction is 77~\% (79~\% volume fraction). The mass fraction of water vapor is specified by the user in terms of a relative humidity and the oxygen and nitrogen mass fractions are adjusted accordingly. All other gas species are zero.  As fuel is burned, product species are produced in direct proportion to the rate of fuel consumption (the major products of combustion are determined from the specified fuel molecule and the minor species yields are specified by the user). The mass fraction of oxygen can limit the combustion rate as discussed in Section~\ref{section:HRR}). Two special separate species are included in the model -- a generic toxic species and an arbitrary trace species. Both are excluded from the overall mass balance, but they are generated by the fire and transported in a manner identical to the other species.

Each unit mass of a species produced by a fire is carried in the flow to the various rooms and accumulates in the layers. The species mass divided by the layer volume is the mass concentration. Filters can be used in mechanical ventilation systems to remove species. The phenomenon has been implemented in CFAST to remove trace species and soot. It is implemented by modifying the source terms which describe gas flow. See Ref.~\cite{Jones:2008} for an example on the use of filtering.

The calculation of radiation exchange in CFAST also depends in part on the species concentrations calculated by the model (and thus the user inputs for species yields). There are two separate radiation calculations performed by CFAST. The first is for thermal radiation as part of the overall heat transfer calculation, discussed in Section~\ref{sec:Radiation}. The second is for visible light extinction to determine visibility, discussed in Section~\ref{Visibility}.






\chapter{The Fire Plume}
\label{sec:TheFire}

Fires in CFAST are specified by the user in terms of a time-dependent heat release rate (HRR), an effective fuel molecule, and the yields of the products of incomplete combustion like soot and CO. Fires can be specified in multiple compartments and are treated as totally separate entities, with no interaction of the plumes. These fires are generally referred to as ``objects'' and can be ignited at a prescribed time, temperature or heat flux.

CFAST does not include a pyrolysis model to {\em predict}, as opposed to specify, the growth and spread of the fire. Rather, pyrolysis rates for each fire are prescribed by the user. While this approach does not directly account for increased pyrolysis due to radiative feedback from the flame or compartment, in theory these effects could be prescribed by the user. In an actual fire, this is an important consideration, and the specification used should consider the experimental conditions as closely as possible.

\section{Combustion Chemistry}

 The HRR of the fire is specified by the user, but it may be constrained by the availability of oxygen in the compartment. The combustion of a hydrocarbon fuel is described by the following single-step reaction:
\begin{eqnarray}
   \mathrm{C_{n_\C}H_{n_H}O_{n_O}N_{n_N}Cl_{n_{Cl}}} &+&  \nu_\OTWO \, \mathrm{O_2}  \rightarrow  \nonumber \\[.1in]
   \nu_\COTWO \, \mathrm{CO_2} &+& \nu_\HTWOO \, \mathrm{H_2O} \; + \; \nu_\CO \, \mathrm{CO} \; + \; \nu_\So \, \mathrm{Soot} \; + \; \nu_\HCl \mathrm{HCl} \; + \; \nu_\HCN \mathrm{HCN} \label{stoich}
\end{eqnarray}
The user specifies the composition of the fuel molecule and the yields of soot and CO, $y_\So$ and $y_\CO$, which are related to their stoichiometric coefficients as follows:
\begin{eqnarray}
   \nu_\So &=& \frac{M_\F}{M_\So} \; y_\So \label{soot_yield} \\[.1in]
   \nu_\CO &=& \frac{M_\F}{M_\CO} \; y_\CO \label{CO_yield}
\end{eqnarray}
Under the assumption that all of the nitrogen and chlorine in the fuel are converted to HCN and HCl, the other stoichiometric coefficients are:
\begin{eqnarray}
  \nu_\COTWO &=& \mathrm{n_\C} - \brackets{\nu_\CO + \nu_\HCN + \nu_\So} \\[.1in]
  \nu_\HTWOO &=& \frac{\mathrm{n_\Hy} - \brackets{\nu_\HCl + \nu_\HCN}}{2} \\[.1in]
  \nu_\OTWO  &=& \nu_\COTWO + \frac{\nu_\HTWOO + \nu_\CO - \mathrm{n_\Oh}}{2} \label{Oxygen_yield} \\[.1in]
  \nu_\HCl   &=& \mathrm{n_{Cl}} \\[.1in]
  \nu_\HCN   &=& \mathrm{n_{N}}
\end{eqnarray}
Note that the nitrogen in the air acts only as a diluent. The yields of hydrogen cyanide and hydrogen chloride are based solely on the composition of the fuel molecule. Finally, a user-specified trace species can be specified to follow the transport that results from fire-induced flow for an arbitrary species. This may be of particular interest for radiological releases \cite{Jones:2008}, but may be useful for any trace amounts released by a fire.



\section{Heat Release Rate}
\label{section:HRR}

As fuel and oxygen are consumed, heat is released and various products of combustion are formed. The heat is released as radiation and convected enthalpy:
\begin{eqnarray}
   \dQr &=& \chi_{\rm r} \, \dQ \\[.1in]
   \dQc &=& (1-\chi_{\rm r}) \, \dQ
\end{eqnarray}
where, $\chi_{\rm r}$ is the fraction  of the fire's heat release rate given off as radiation. The user specifies the total heat release rate as a function of time, along with a characteristic base diameter, $D$, which is used in the plume temperature and mass entrainment correlations. The user also specifies $\chi_{\rm r}$ whose default value is taken to be 0.35~\cite{Drysdale:1985}.

While it is convenient for the user to directly specify the heat release rate of the fire, it is actually the pyrolysis rate of fuel, $\dmf$, that is specified:
\be
   \dmf = \frac{\dQ}{\Dh}
\ee
where $\Dh$ is the heat of combustion. In the event that the HRR is constrained by the availability of oxygen, the pyrolysis rate does not change, but the HRR becomes:
\be
   \dQ = \min \Big( \dmf \, \Dh \, , \, \dme \, Y_\OTWO \, C_{\rm LOL} \, \DhO \Big)
\ee
where $\dme$ is the entrainment rate, $Y_\OTWO$ is the mass fraction of oxygen in the layer containing the fire, $\DhO$ is the heat of combustion based on oxygen consumption\footnote{The heat of combustion based on oxygen consumption is taken to be 13.1~MJ/kg, representative of typical hydrocarbon fuels~\cite{Huggett:1980}.}, and $C_{\rm LOL}$ is the smoothing function ranging from 0 to 1:
\be
   C_{\rm LOL} = \frac{\tanh \Big( 800 (Y_\OTWO - Y_{\OTWO,{\rm l}}) - 4 \Big) + 1}{2}
\ee
The limiting oxygen mass fraction, $Y_{\OTWO,{\rm l}}$, is 0.1, by default.



\section{Plume Entrainment}

The plume mass entrainment, $\dme(z)$, at a height $z$ above the base of the fire is estimated using Heskestad's correlation~\cite{Heskestad:2002}:
\be
   \dme(z) = 0.196 \, \left( \frac{g \, \rho_\infty^2}{c_p \, T_\infty} \right)^{1/3} \, \dQ_{\rm c}^{1/3} \; \brackets{z - z_0}^{5/3} \;
   \left( 1 + \frac{2.9 \, \dQ_{\rm c}^{2/3}}{ \left( \sqrt{g} \, c_p \, \rho_\infty \, T_\infty \right)^{2/3} \, (z-z_0)^{5/3}} \right) \label{mdot_e}
\ee
where $z_0$ is a virtual origin defined as
\be
   \frac{z_0}{D} = -1.02 + 1.4 \, \dQ^{* \, 2/5} \quad ; \quad \dQ^* = \frac{\dQ}{\rho_\infty \, c_p \, T_\infty \, \sqrt{g \, D} \, D^2} \label{virtual_origin}
\ee
Note that the virtual origin is defined in terms of the total heat release rate of the fire, $\dQ$. Equation~(\ref{mdot_e}) is recommended above the mean flame height, $L$. Below the flame height, Heskestad recommends the following:
\be
  \dme(z) = \dme(L) \, \frac{z}{L} \quad ; \quad \frac{L}{D} = -1.02 + 3.7 \, \dQ^{* \, 2/5}
\ee
The mean flame height is defined as the distance from the fuel source to the top of the visible flame where the intermittency is 0.5.  A flame intermittency of 0.5 means that the visible flame is above the mean 50~\% of the time and below the mean 50~\% of the time.

In CFAST, there is a constraint on the mass entrainment rate because the plume can rise only so high for a given HRR.  Early in a fire, the plume may not have sufficient energy to reach the compartment ceiling. Therefore, a limit is placed on the entrainment rate. For the plume to be able to penetrate the hot upper layer, the density of the gas in the plume must be less than the density of the gas in the upper layer. This implies that the upper layer temperature must be less than the plume temperature:
\be
   \Tu < \Tp \approx \frac{ \dQc + \dme \, c_p \, \Tl }{ \dme \, c_p}
\ee
Rearranging terms yields a limit on the mass entrainment:
\be
   \dm_e < \frac{\dQc}{c_p (\Tu - \Tl)}
\ee


\section{Plume Temperature and Velocity}
\label{sec:Plume_Temp_Velocity}

The centerline plume temperature rise, $\Delta T_0(z)$, and velocity, $u_0(z)$, at a height $z$ above the base of the fire is estimated using Heskestad's correlations~\cite{Heskestad:2002}:
\be
   \Delta T_0(z) = 9.1 \, \left( \frac{T_\infty}{g \, c_p^2 \, \rho_\infty^2} \right)^{1/3} \, \dQ_{\rm c}^{2/3} \, (z-z_0)^{-5/3}  \label{plume_temperature}
\ee
\be
   u_0(z) = 3.4 \, \left( \frac{g}{c_p \, \rho_\infty \, T_\infty} \right)^{1/3} \, \dQ_{\rm c}^{1/3} \, (z-z_0)^{-1/3}  \label{plume_velocity}
\ee
where the virtual origin, $z_0$, is defined in Eq.~(\ref{virtual_origin}). It is assumed that the temperature and velocity decrease following a Gaussian profile off the centerline:
\be
   \Delta T(r,z) = \Delta T_0(z) \, \exp \left[ -\left( \frac{r}{\sigma_{\Delta T}} \right)^2 \right] \quad ; \quad \sigma_{\Delta T} = 0.14 \, \left( \frac{T_0(z)}{T_\infty} \right)^{1/2} \, (z-z_0) \label{plume_temperature2}
\ee
\be
   u(r,z) = u_0(z) \, \exp \left[ -\left( \frac{r}{\sigma_u} \right)^2 \right] \quad ; \quad \sigma_u \approx 1.1 \, \sigma_{\Delta T} \label{plume_velocity2}
\ee
The temperature equations~(\ref{plume_temperature}) and (\ref{plume_temperature2}) are valid above the flame height, $L$, and below the hot gas layer interface, $z_{\rm I}$. Below the flame height, the temperature rise is approximated at 900~K. Above the layer interface, the temperature and density of the entrained air is significantly different than that of the lower layer. To account for this, the temperature correlation is modified for values of $z$ greater than the interface height, $z_{\rm I}$:
\be
   \Delta T_0(z) = 9.1 \, \left( \frac{\Tu}{g \, c_p^2 \, \rho_{\rm u}^2} \right)^{1/3} \, \dQ_{\rm c}^{2/3} \, (z-z_0')^{-5/3}  \quad ; \quad z>z_{\rm I}  \label{plume_temperature_upper}
\ee
where the modified virtual origin is given by:
\be
   z_0' = z_{\rm I} - \left( \frac{\Tu}{\Tl} \right)^{3/5} \, (z_{\rm I}-z_0)  \label{virtual_origin_modified}
\ee
Equations~(\ref{plume_temperature_upper}) and (\ref{virtual_origin_modified}) are obtained by asserting that $\Delta T_0$ is continuous across the layer interface and that $\Tu \rho_{\rm u} \approx \Tl \rho_{\rm l}$.

The mass entrainment correlation, Eq.~(\ref{mdot_e}), is modified the same way above the layer interface.



\chapter{Ventilation}

CFAST models three types of vent flow: natural flow through vertical vents (such as doors or windows),  natural flow through horizontal vents (such as ceiling holes or hatches), and forced flow via mechanical ventilation. Forced flow can occur through either vertical or horizontal vents.

Atmospheric pressure is about 100~kPa. Fires produce pressure changes from 1~Pa to 1~kPa and mechanical ventilation systems typically involve pressure differentials of about 1~Pa to 100~Pa.  The pressure variables are solved to a higher accuracy than other solution variables because of the subtraction (with resulting loss of precision) needed to calculate vent flows from pressure differences.


\section{Vertically-Oriented Vents (Doors and Windows)}

Natural flow through windows and doors is governed by the vertical stratification of the pressure difference across the opening~\cite{Emmons:SFPE}. The mass flow is calculated by dividing the opening into discrete horizontal segments, each of which is bounded by either the top or bottom of the opening, the zone interface of either compartment, or the neutral plane, which is where the velocity changes direction. This is shown schematically in Fig.~\ref{fig:Flow_Patterns}.

\begin{figure}[t]
\setlength{\unitlength}{1in}
\begin{picture}(6.5,4)
\thicklines
\put(0,0){\line(1,0){6.5}}
\put(0,4){\line(1,0){6.5}}
\put(3.25,4){\line(0,-1){1}}
\thinlines
\put(0,1.5){\line(1,0){3.25}}
\put(3.25,2.5){\line(1,0){3.25}}
\put(0,2.0){\line(1,0){6.5}}
\put(3.5,0.75){\vector(-1,0){0.5}}
\qbezier(3.5,1.75)(3.25,1.75)(3.0,1.5)
\put(3.0,1.5){\vector(-1,-1){0.2}}
\qbezier(3,2.25)(3.25,2.25)(3.5,2.5)
\put(3.5,2.5){\vector(1,1){0.2}}
\put(3,2.75){\vector(1,0){0.5}}

\put(1.625,3.8){\makebox(0,0){Compartment 1}}
\put(4.875,3.8){\makebox(0,0){Compartment 2}}
\put(6,1.85){\makebox(0,0){Neutral Plane}}
\put(5.95,2.35){\makebox(0,0){Layer Interface}}
\put(0,1.35){Layer Interface}
\put(3.75,0.75){\makebox(0,0){$\dm_{\rm l \to l}$}}
\put(3.75,1.75){\makebox(0,0){$\dm_{\rm l \to u}$}}
\put(2.75,2.25){\makebox(0,0){$\dm_{\rm u \to l}$}}
\put(2.75,2.75){\makebox(0,0){$\dm_{\rm u \to u}$}}

\end{picture}
\caption{Flow patterns for horizontal flow through a vertical vent.}
\label{fig:Flow_Patterns}
\end{figure}

Let $z=b$ and $z=t$ denote the height of the bottom and top of the segment, and $\Delta P_b$ and $\Delta P_t$ denote the pressure differences at these heights.  Because a given segment is either completely above or completely below the neutral plane, the two pressure differences will have the same sign. The mass flow through the segment can then be computed by integrating Bernoulli's equation from $b$ to $t$:
\begin{eqnarray}
\dm &=& \int_b^t C \sqrt{2 \rho \, \Delta P(z)} \, w \, dz  \\[.1in]
    &=& C\sqrt{2\rho} \, w \int_b^t\sqrt{\frac{|(t-z) \, \Delta P_b + (z-b) \, \Delta P_t|}{t-b}} \; dz \\[.1in]
    &=& \frac{2}{3} \, C \sqrt{2\rho} \, w \, (t-b)\frac{|\Delta P_t|^{3/2}-|\Delta P_b|^{3/2}}{|\Delta P_t|-|\Delta P_b|}
\label{eq:massflowone}
\end{eqnarray}
Here, $C$ is the orifice coefficient taken to be 0.7~\cite{Steckler_Coefficients}, $\rho$ is the gas density of the upwind compartment, $w$ is the width of the opening, and $\Delta P(z)$ is the pressure across the interface at elevation $z$. Note the use of the integral formula:
\begin{eqnarray}
\int \sqrt{A+Bz} \, dz = \frac{2}{3B}(A+Bz)^{3/2}+\mbox{constant}
\end{eqnarray}
where $A=(|t\,\Delta P_t|-b\,|\Delta P_b|)/(t-b)$ and $B=(|\Delta P_t|-|\Delta P_b|)/(t-b)$. Equation \ref{eq:massflowone} can be written:
\be
   \dm = \frac{2}{3} C \sqrt{2 \rho} \, w \, (t-b)  \frac{|\Delta P_t|+\sqrt{|\Delta P_t \,\Delta P_b|}+|\Delta P_b|}{\sqrt{|\Delta P_t|}+\sqrt{|\Delta P_b|} }
\ee
This is the way it is written in Ref.~\cite{Emmons:SFPE}.

Figure~\ref{fig:Flow_Patterns} indicates schematically how the various mass flows through the opening are distributed. For the flow originating in the upper layer of the upstream compartment flowing into the upper layer of the downstream compartment, $\dm_{\rm u \to u}$, or the flow from the lower layer to the lower layer, $\dm_{\rm l \to l}$, the mass is applied directly to the downstream layer.

The mass flow from the upper layer of the upstream compartment to the lower layer of the downstream, $\dm_{\rm u \to l}$, is assumed to rise into the upper layer via a spill plume. The enthalpy flow rate of the plume is:
\be
   \doh_{\rm u \to l} = c_p \brackets{T_{\rm u,1}-T_{\rm l,2}} \, \dm_{\rm u \to l}
\ee
Assuming that $T_{\rm u,1} > T_{\rm l,2}$, the mass entrainment of the spill plume is given by Poreh {\em et al.}~\cite{Poreh:1998}:
\be
   \dm_{\rm e,p} = \dm_{\rm u \to l} \, + \, C_{\rm m} \, \left( \frac{T_{\rm l,2}}{T_{\rm u,1}} \right)^{2/3} \, \left( \frac{ g \, \rho_{\rm l,2}^2}{c_p \, T_{\rm l,2}} \right)^{1/3} \, \doh_{\rm u \to l}^{1/3} \; w^{2/3} \;
   (z_{\rm I}-z_{\rm N})
\ee
where $C_{\rm m}$ is an empirical constant equal to 0.44, $w$ is the width of the opening, $z_{\rm I}$ is the height of the layer interface, and $z_{\rm N}$ is the height of the neutral plane. Suggested values for $C_{\rm m}$ vary between 0.44 and 0.66, all of which were determined empirically in a number of different experimental configurations.

For the mass flow leaving the lower layer of the upstream compartment and entering the upper layer of the downstream compartment, $\dm_{\rm l \to u}$, the shear flow causes vortex shedding that entrains upper layer gas and deposits it in the lower layer. It is assumed that the incoming cold plume behaves like the inverse of the usual door jet between adjacent hot layers; forming a descending plume.  The same equations are used to calculate this inverse plume as are used for the upright door mixing, above.

\section{Horizontally-Oriented Vents (Floor and Ceiling Vents)}

Flow through a ceiling or floor vent is governed by both pressure and density differences. The simplest form is uni-directional flow driven primarily by a relatively large pressure difference. When the pressure difference is relatively small, the density difference, where hot gas underlies colder gas, can lead to bi-directional flow where the gas in the lower compartment rises into the upper compartment and {\em vice versa}.  This situation might arise in a real fire if the room of origin suddenly has a hole open up in the ceiling.

Cooper's algorithm~\cite{Cooper:1989, Cooper:1990, Cooper:1995} is used for computing mass flow through ceiling and floor vents:
\be
   \dm = 0.1 \brackets{\frac{g \, \Delta \rho \, A_{\rm v}^{5/2}}{\rho_{\rm avg}}} \brackets{1 - \frac{2 \, A_{\rm v}^2 \, \Delta P}{S^2 \, g \, \Delta \rho \, D^5}}
\ee
where $D = 2 \sqrt{A_{\rm v} / \pi}$ and $S$ is 0.754 for round or 0.942 for square openings, respectively. For each layer on either side of the vent, there are two values of the mass flow, $\dm_{\rm in}$ and $\dm_{\rm out}$. These terms are symmetric: the outgoing flow from one compartment is the incoming flow to the other. The corresponding enthalpy flows are determined from the relative size and temperature of the lower and upper layers:
\begin{eqnarray}
  \doh_{\rm in} &=& c_p \, \dmu \, \Tu + c_p \, \dml \, \Tl \\[0.1in]
  \dmu &=& \dm_{\rm in} \, \frac{\Vu}{V} \\[0.1in]
  \dml &=& \dm_{\rm in} \, \frac{\Vl}{V}
\end{eqnarray}
The mass and energy are then deposited into the upper or lower layer of the receiving compartment based on the effective temperature of the incoming flow relative to the upper and lower layers of the receiving compartment. If the temperature of the incoming flow is higher than the temperature of the lower layer, then the flow is deposited into the upper layer. This is similar to the idea of using a virtual plume for a doorway flow.


\section{Forced Flow}

CFAST models mechanical ventilation in terms of user-specified volume flows at various points in the compartment. The model does not include duct work or fan curves. These equations are high-order, non-linear and in some cases ill-posed, which caused a great deal of difficulty in reaching a numerical solution.

The flow through mechanical vents can be filtered. Filtering affects particulates such as smoke and the trace species. Filtering can be turned on at any time. Effectiveness is from 0~\% (no effect) to 100~\% which completely blocks the flow of these two species.





\chapter{Heat Transfer}

This section discusses thermal radiation, convection and conduction, the three mechanisms by which heat is transferred between the gas layers and the enclosing compartment walls. Hot gases exchange heat with solid surfaces via convection and radiation. Heat is transferred through solids via conduction. Different material properties can be used for the ceiling, floor, and walls of each compartment (although all the walls of a compartment must be the same).  Additionally, each surface can be composed of up to three distinct layers.  This allows the user to deal naturally with the actual building construction.  Material thermophysical properties are assumed to be constant. Radiative transfer occurs among the fire(s), gas layers and compartment surfaces (ceiling, walls and floor).  This transfer is a function of the temperature differences and the emissivity of the gas layers as well as the compartment surfaces.  Typical surface emissivity values only vary over a small range.  For the gas layers, however, the emissivity is a function of the concentration of species which are strong radiators, predominately smoke particulates, carbon dioxide, and water.

\section{Radiation}
\label{sec:Radiation}

Radiation heat transfer is calculated between the ceiling, floor, wall layers, and fire, with the inclusion of emission and absorption by the hot gas layer~\cite{Forney_radiation}. The following assumptions are made:
\begin{itemize}
\item Each gas layer and each wall segment is assumed to be at a uniform temperature.
\item The wall and gas layer temperatures are assumed to change slowly over the duration of the time step of the governing equations.
\item The fire is assumed to radiate uniformly in all directions emitting a fraction, $\chi_{\rm r}$, of the total heat release rate.  This radiation is assumed to originate from a single point.  Radiation feedback to the fire and radiation from the plume is not modeled in the radiation exchange algorithm.
\item The radiation emitted is assumed to be diffuse and gray.  In other words, the radiant fluxes emitted are independent of direction and wavelength. At a solid surface, the emittance, $\epsilon$, absorptance, $\alpha$ and reflectance, $\rho$, are related via $\epsilon = \alpha = 1 - \rho$. In the gas phase, the emittance, $\epsilon$, absorptance, $\alpha$ and transmittance, $\tau$, are related via $\epsilon = \alpha = 1 - \tau$.
\item Rooms or compartments are assumed to be rectangular boxes.  Each wall is either perpendicular or parallel to every other wall.  Radiation transfer through vent openings is lost from the room.
\end{itemize}
The compartment lining is divided into four parts: the ceiling, the floor, and the wall sections above and below the layer interface. The net radiative heat flux at surface~$k$, $\dq_k''$, is found by solving the simplified radiation transport equation given by Eq.~(17-20) in Siegel and Howell~\cite{SiegelandHowell:1981}:
\be
   \frac{\dq_k''}{\epsilon_k} - \displaystyle\sum_{j=1}^N \frac{1 - \epsilon_j}{\epsilon_j} \, \dq_j'' \, F_{k-j} \, \tau_{k-j} = \sigma T_k^4 - \displaystyle\sum_{j=1}^N \brackets{\sigma T_j^4 \, F_{k-j} \, \tau_{k-j}} - c_k \label{RTE}
\ee
where $F_{k-j}$ is the configuration factor (fraction of radiant energy emitted by surface $j$ that is intercepted by surface $k$), $\tau_{k-j}$ is the transmittance, $\sigma$ is the Stefan-Boltzman constant, $\epsilon_k$ is the emissivity, $A_k$ is the area, and $T_k$ is the temperature of surface $k$. The radiation from the hot gas layer and the fire is included in the last term:
\be
   c_k = \epsilon_{\rm u} \, F_{{\rm u}-k} \, \sigma \, \Tu^4 + \frac{\omega_{{\rm f}-k}}{4 \pi} \frac{\chi_{\rm r} \, \dQ}{A_k}  \label{ckeq}
\ee
where $\epsilon_{\rm u}$ is the emittance (absorptance) of the upper layer, $F_{{\rm u}-k}$ is the view factor between the upper layer and solid surface, $\omega_{{\rm f}-k}$ is the solid angle between the fire and wall\footnote{Note that as the area of surface $k$ shrinks to zero, $\omega_{{\rm f}-k}/A_k \to 1/R^2$, yielding the classic point source radiation model}, and $\dQ$ is the heat release rate of the fire. If the solid surface, $k$, is the floor or the lower wall, the view factor refers to the layer interface. If the solid surface, is the upper wall or ceiling, the view factor is 1.

Reference~\cite{Forney_radiation} describes the solution of Eq.~\ref{RTE}.


\subsubsection{Configuration Factors}

The configuration factor, $F_{1-2}$, is the fraction of radiant energy emitted by surface 1 that is intercepted by  surface 2, and is calculated:
\be
   F_{1-2} = \frac{1}{A_1} \int_{A_1} \int_{A_2} \frac{\cos \theta_1 \, \cos \theta_2}{\pi L^2} \, dA_2 \, dA_1 \label{eq:config_factor}
\ee
where $L$ is the distance along the line of integration,  $\theta_1$ and $\theta_2$ are the angles for surface 1 and 2 between the respective normal vectors and the line of integration, and $A_1$ and $A_2$ are the areas of the two surfaces.  These terms are illustrated in Fig.~\ref{fig:Rad_Config_Factor}.
\begin{figure}
\begin{center}
\includegraphics[width=3.0in]{FIGURES/Theory/Radiation_Config_Factor}\\
\end{center}
\caption{Setup for a configuration factor calculation between two arbitrarily oriented finite areas.}
 \label{fig:Rad_Config_Factor}
\end{figure}
There are two types of configuration factors in CFAST. First, consider two rectangles perpendicular to each other having a common edge of the same length, $l$. The dimension of the source rectangle is $l \times w$ and the target rectangle is $l \times h$. The configuration factor from source to target is:
\begin{eqnarray}
\lefteqn{F_{1-2} = \frac{1}{\pi \, W} \left[ W \, \tan^{-1} \frac{1}{W} + H \, \tan^{-1} \frac{1}{H} - \sqrt{H^2+W^2} \, \tan^{-1} \frac{1}{\sqrt{H^2+W^2}} + \right.}  \nonumber \\[.1in]
& &   \left. \frac{1}{4} \, \ln \left\{ \frac{(1+W^2)(1+H^2)}{1+W^2+H^2} \, \left( \frac{W^2(1+W^2+H^2)}{(1+W^2)(W^2+H^2)} \right)^{W^2} \left( \frac{H^2(1+W^2+H^2)}{(1+H^2)(W^2+H^2)} \right)^{H^2} \right\} \right]
\end{eqnarray}
where $H=h/l$ and $W=w/l$.

Next, consider two identical, parallel, directly opposed rectangles. The dimension of the rectangles is $a \times b$ and the separation distance is $c$. The configuration factor is:
\begin{eqnarray}
\lefteqn{F_{1-2} = \frac{2}{\pi \, X \, Y} \left\{ \ln \left[ \frac{(1+X^2)(1+Y^2)}{1+X^2+Y^2} \right]^{1/2} + X \sqrt{1+Y^2} \, \tan^{-1} \frac{X}{\sqrt{1+Y^2}} + \right.} \nonumber \\[.1in]
& & \left. Y \sqrt{1+X^2} \, \tan^{-1} \frac{Y}{\sqrt{1+X^2}} - X \, \tan^{-1} X - Y\, \tan^{-1} Y \right\}
\end{eqnarray}
where $X=a/c$ and $Y=b/c$. These formulae are found in Appendix~C of Ref.~\cite{SiegelandHowell:1981}.

Next, the normalized solid angle, $\omega_{\rm f-k}$, in Eq.~(\ref{ckeq}) is computed. First, place a sphere of radius, $r$ tangent to a rectangle of dimension $x \times y$ such that the point of tangency is the corner of the rectangle. The normalized solid angle formed by the center of the sphere and the edges of the rectangle is given by:
\be
   \omega(x,y;r) = \frac{1}{4\pi} \left[ \sin^{-1} \left( \frac{A \, y}{\sqrt{y^2+r^2}} \right) + \sin^{-1} \left( \frac{A \, x}{\sqrt{x^2+r^2}} \right) - \frac{\pi}{2} \right] \quad ; \quad A=\sqrt{1+\frac{r^2}{x^2+y^2}}
\ee
Suppose now that the radiation of a fire is assumed to emanate from a point a distance $r$ above the floor whose dimension is $L \times W$. Suppose also that the floor is divided into four quadrants based on the location of the fire. The dimensions are partitioned $L=L_1+L_2$ and $W=W_1+W_2$. The normalized solid angle between the point source fire, f, and the floor, fl, is:
\be
   \omega_{\rm f-fl} = \sum_{j=1}^2 \sum_{i=1}^2 \omega(L_i,W_j;r)
\ee

\subsubsection{Transmittance and Absorptance}

The transmittance is the fraction of radiant energy that will pass through a volume filled with an absorbing media. It is usually expressed in the form:
\be
   \tau = {\rm e}^{-a L}
\ee
where $a$ is the absorption coefficient and $L$ is the path length. The absorptance is the fraction of radiant energy absorbed by that volume. For a gray gas, $\alpha + \tau = 1$.

In general, the transmittance and absorptance are functions of wavelength. This is an important factor to consider for the major gaseous products, $\textnormal{CO}_2$  and $\textnormal{H}_2 \textnormal{O}$. However soot has a continuous absorption spectrum that allows the transmittance and absorptance to be approximated as ``gray'' \cite{SiegelandHowell:1981} across the entire spectrum. The total transmittance over a path length $L$ through a volume of combustion products is taken as the product of the transmittance of the soot and major product gases:
\be
   \tau = e^{-a_{\rm s}L} \brackets{1 - \alpha_{\rm H_2O} - 0.5 \, \alpha_{\rm CO_2}}
\ee
The factor of 0.5 applied to the absorptance of CO$_2$ accounts for the overlap of the wavelength bands of the two gases. Tien~et~al.~\cite{Tien:2002} suggest that the absorption coefficient for soot may be approximated $a_{\rm s} = k f_v T$ where $k$ is a constant that depends on the optical properties of the soot particles, $f_v$ is the soot volume fraction, and $T$ is the (absolute) temperature. Values of $k$, have been found to be about constant for a wide range of fuels~\cite{Tien:1978}.

Absorptance data for $\textnormal{H}_2 \textnormal{O}$ and $\textnormal{CO}_2$ are reported in Ref.~\cite{Edwards:1985}. For each gas, these data are tabulated in a look-up table, implemented as a two-dimensional array based on temperature and gas concentration.

The effective path length, $L$, for the upper gas layer is approximated to be 1.8 times its depth~\cite{Tien:2002}.




\section{Convection}

The transfer of heat between the gas and solid surfaces is handled slightly differently at the ceiling, floor and walls, due to the difference in orientation and the presence of a relatively thin hot flow near the ceiling known as the ceiling jet. The following two sections describe how the convective heat transfer is done for these different surfaces.

\subsection{Walls and Floor}

In general, the convective heat flux to a solid surface is given by:
\be
   \dqc'' = h \, \brackets{\Tg - \Ts}  \label{convective_heat_flux}
\ee
The convective heat transfer coefficient, $h$, is a function of the gas properties, temperature, and velocity. In CFAST, simple correlations for natural convection are used since the gas velocity is unknown:
\be
   h = C {|\Tg - \Ts|}^{1/3}
\ee
where $C$ is an empirical coefficient (1.52 for the floor and ceiling (in the absence of a ceiling jet) and 1.31 for the walls~\cite{Holman:1990}), $\Tg$ is the average gas layer temperature adjacent to the surface, and $\Ts$ is the surface temperature.

\subsection{Ceiling}

During the early stage of a fire before a hot gas layer has formed, the convective heat transfer to the ceiling is governed by the temperature and velocity of the ceiling jet. Alpert's chapter in the {\em SFPE Handbook}~\cite{Alpert:SFPE} presents an empirical correlation for the convective heat flux from the ceiling jet to a relatively cool surface:
\be
   \dqc'' = 1.323 \, f \, \frac{\dQ_{\rm c}}{H^2} \, \left( \frac{r}{H} \right)^{-1.36}  \label{eq:cjflux}
\ee
where $f$ is a friction factor estimated to be 0.03, $r$ is the radial distance to the plume centerline, $H$ is the ceiling height, and $\dQ_{\rm c}$ is the convective fraction of the heat release rate. The average convective heat flux to the ceiling can be obtained by integrating this expression over the entire ceiling:
\be
   \dq_{\rm c,avg}'' = \frac{1}{LW} \int_0^{2\pi} \int_0^R \dqc'' \, r \, dr \, d\theta = \frac{0.27 \, \dQ_{\rm c}}{(LW)^{0.68} \, H^{0.64}} \label{eq:cjfluxavg}
\ee
Note that the integration is carried out over a circle whose area, $\pi R^2$, is equal to the area of the ceiling, $LW$.

Equation~(\ref{eq:cjfluxavg}) applies to the early stage of the fire; thus, a modified heat transfer coefficient is used so that there is a transition from the early to later stages when a layer has formed:
\be
   h = \max \left( \frac{\dq_{\rm c,avg}''}{\Tu-\Ts} \, , \, C {|\Tu - \Ts|}^{1/3} \right)
\ee
Here, $\Tu$ is the average temperature of the upper layer and $\Ts$ is the ceiling surface temperature. Notice that the rightmost term is simply the correlation used for the walls and floor.

\section{Heat Conduction within Solid Walls or Targets}

The heat conduction equation is solved in the direction normal to solid target or wall surfaces using non-uniformly spaced nodes and a second order accurate central difference scheme for the spatial derivatives and a semi-implicit time marching scheme. At each time step, the internal solid temperatures are updated in time until the net convective and radiative heat flux striking the wall equals with the heat flux into the solid~\cite{Moss:1992}:
\be
   \dq'' \equiv \dqr'' + \dqc'' = -k \, \frac{dT}{dx} \Big|_{x=0}
\ee
where $k$ is the thermal conductivity of the solid.  This solution strategy requires a differential algebraic equation (DAE) solver that can simultaneously solve both differential and algebraic equations.  With this method, only one or two extra equations are required per wall segment (two if both the interior and exterior wall segment surface temperatures are computed).  This solution strategy is more efficient than the method of lines since fewer equations need to be solved. Conduction is then coupled to the gas phase energy exchange.

A non-uniform array of internal nodes is used to capture steep gradients in temperature near the surface. Define a penetration depth of
\be
   x_p = 2 \sqrt{\alpha \, t_{\rm end}} \; \hbox{erfc}^{-1} \brackets{0.05}
\ee
where $\hbox{erfc}^{-1}$ denotes the inverse of the complementary error function. The value $x_p$ is the location in a semi-infinite wall where the temperature rise is 5~\% after $t_{\rm end}$ seconds. Eighty percent of the nodes are placed on the interior side of $x_p$ and the remaining 20~\% are placed on the exterior side.

The heat conduction equation normal to the solid surface is:
\be \frac{\partial T}{\partial t} = \frac{k}{\rho c}\frac{\partial^2 T}{\partial x^2}
\label{eq:Target_PDE} \ee
where $k$, $\rho$ and $c$ are the thermal conductivity, density and heat capacity of the target. At the surface, $x=0$, the boundary condition is:
\be
   \dq''=-k\frac{dT}{dx} \label{eq:Target_Fourier}
\ee
where $\dq''$ is the net convective and radiative heat flux.

\newcommand{\Dt}{\Delta t}
\newcommand{\Dr}{\Delta r}
\newcommand{\Tipo}{T_{i+1}^{n+1}}
\newcommand{\Ti}{T_{i}^{n+1}}
\newcommand{\Timo}{T_{i-1}^{n+1}}

The 1-D heat conduction equation can be solved in either Cartesian or cylindrical coordinates. The solution methodology shall be presented for cylindrical coordinates:
\be
  \frac{\partial T}{\partial t} = \frac{k}{\rho c} \frac{1}{r} \frac{\partial}{\partial r} \left( r \frac{\partial T}{\partial r} \right)
\ee
Dividing the cylinder into $N$ uniformly spaced concentric control volumes, this equation can be written in discretized form:
\begin{eqnarray}
\Ti-T_i^n&=& \frac{\Dt}{\Dr} \frac{k}{\rho c}
\left[
\left(\frac{\Tipo-\Ti}{\Dr}\right)
\frac{r_i}{r_{i-1/2}}-
\left(\frac{\Ti-\Timo}{\Dr}\right)
\frac{r_{i-1}}{r_{i-1/2}}
\right]
\nonumber\\[0.2in]
&=&\frac{\Dt \, \alpha}{\Dr^2}
\left[
\left(\Tipo-\Ti\right)
\left(\frac{i}{i-0.5}\right)-
\left(\Ti-\Timo\right)
\left(\frac{i-1}{i-0.5}\right)
\right]
\label{eq:cylheat6}
\end{eqnarray}
where $\alpha=k/(\rho c)$. Defining $C_i$ and $D_i$ as
\be
C_i = \frac{\alpha\Dt}{\Dr^2}\left(\frac{i-1}{i-0.5}\right) \quad ; \quad D_i = \frac{\alpha\Dt}{\Dr^2}\left(\frac{i}{i-0.5}\right)
\ee
Eq.~(\ref{eq:cylheat6}) can be written:
\be
-C_i \, \Timo + \left( 1+2\frac{\alpha\Dt}{\Dr^2} \right) \, \Ti - D_i \, \Tipo = T_i^n  \quad \quad i=1...N-1
\label{eq:cylheat8}
\ee
The boundary condition is applied at control volume $N$:
\begin{eqnarray*}
T_N^{n+1}-T_N^n=\frac{\alpha\Dt}{\Dr^2}
\left[\frac{\Dr \, \dq''}{k} \frac{N}{N-0.5} -(T_N^{n+1}-T_{N-1}^{n+1}) \frac{N-1}{N-0.5} \right]
\end{eqnarray*}
or
\be
-C_N \, T_{N-1}^{n+1}+ \left( 1+C_N \right) T_N^{n+1} = T_N^n+D_N\frac{\Dr}{k} \dq''
\label{eq:cylheat10}
\ee
The internal temperature profile, $T_i$, is then obtained with a tri-diagonal linear solver.




\section{Coupling the Gas and Solid Phase Calculations}

To illustrate the method, consider a one room case with one active wall.  There are four gas phase equations (pressure, upper layer volume, upper and lower layer temperatures) and one wall temperature equation.  Implementation of the gradient matching method requires that storage be allocated for the temperature profiles at the current time step, $t$, and at the next time step, $t + \Delta t$.  Given the profile at time $t$ and values for the five unknowns at time $t + \Delta t$ (initial guess by the solver), the temperature profile is advanced from time $t$ to $t + \Delta t$.  The temperature gradient at the wall surface is computed followed by the residuals for the five equations.  The DAE solver adjusts the solution variables and the time step until the residuals for all the equations are below an error tolerance.  Once the solver has completed the step, the array storing the temperature profile for the previous time is updated, and the DAE solver is ready to take its next step.

Heat transfer between connected compartments is modeled by merging the back surfaces of the connected ceiling and floor of the compartments or the back wall surfaces of the connected horizontal compartments.  A heat conduction problem is solved for the merged walls using a temperature boundary condition for both the near and far wall.  As before, temperatures are determined by the DAE solver so that the heat flux striking the wall surface (both interior and exterior) is consistent with the temperature gradient at that surface.

For horizontal heat transfer between compartments, the connections may be between partial wall surfaces, expressed as a fraction of the wall surface. CFAST first estimates conduction fractions analogous to radiation configuration factors. For example, if only one half of the rear wall in one compartment is adjacent to the front wall in a second compartment, the conduction fraction between the two compartments is 0.5. Once these fractions are determined, an average flux, $\dq_{\rm avg}''$, is calculated using
\be
   \dq_{\rm avg}'' = \sum_{\rm walls} \, F_{i-j} \, \dq_j''
\ee
where $F_{ij}$ is the fraction of flux from wall $i$ that contributes to wall $j$, $\dq_j''$ is the flux striking wall $j$.



\chapter{Fire Protection Devices}



\section{Sprinkler and Heat Detector Activation}

The link temperature of a sprinkler or heat detector is modeled using the differential equation~\cite{Schifiliti:2002}:
\be
   \frac{d \TL}{dt} = \frac{\sqrt{v}}{\rm RTI} \brackets{\Tg - \TL}  \label{eq:RTI}
\ee
where $\TL$ and $\Tg$ are the link and gas temperatures, $v$ is the gas velocity, and RTI (Response Time Index) is a measure of the sensor's thermal inertia. The gas temperature and velocity obtained from the plume algorithm (section \ref{sec:Plume_Temp_Velocity}) and the ceiling jet algorithm, described below. Rooms without fires do not have ceiling jets, in which case the upper layer temperature is used, along with a fixed velocity of 0.1~m/s {\bf REFERENCE?}. The link and gas temperatures and the velocity are functions of time; the RTI is a constant for a given detector type. The detector equation is solved numerically using the semi-implicit updating scheme:
\be
   \frac{\TL^{n+1}-\TL^n}{\delta t} = \frac{1}{2} \left( \frac{\sqrt{v^n}}{\rm RTI} \brackets{\Tg^n - \TL^n}  + \frac{\sqrt{v^{n+1}}}{\rm RTI} \brackets{\Tg^{n+1} - \TL^{n+1}}  \right) \label{eq:RTI_rewritten}
\ee
where the superscript $n$ denotes the value at the current time, and $\delta t$ is the time step.

The temperature and velocity just below the ceiling is typically greater than that of the upper layer due to the presence of a ceiling jet. From the work of Alpert and Heskestad~\cite{Alpert:SFPE}, the temperature of an unconfined ceiling jet is given by:

\be
   T_{\rm cj}(r) - T_\infty = T_\infty \, \dQ^{* \, 2/3}_H \, \left\{ \begin{array}{l@{\quad}l}
   6.3 & r/H\le 0.2 \\[0.1in]
   \left( 0.225 + 0.27 \, r/H \right)^{-4/3} & 0.2 < r/H < 4.0
    \end{array} \right.
\ee
To be consistent with the calculated plume temperature, this is adjusted so that the calculated plume and ceiling jet temperatures are the same for $r/H \le 0.2$:
\be
   T_{\rm cj}(r) - T_\infty = \left\{ \begin{array}{l@{\quad}l}
   \Delta T_0(H)  & r/H\le 0.2 \\[0.1in]
   0.182 \, \Delta T_0(H) \, \left( 0.225 + 0.27 \, r/H \right)^{-4/3} & 0.2 < r/H < 4.0
    \end{array} \right. \label{Tcj}
\ee
where $\Delta T_0(H)$ is the calculated centerline plume temperature rise at the compartment ceiling from  eq. \ref{plume_temperature} and the constant 0.182 is $\left({0.225 + 0.27 \cdot 0.2} \right)^{4/3}$.

The radial velocity is given by:
\be
   v_{\rm cj}(r) = \sqrt{g \, H} \, \dQ^{* \, 1/3}_H \left\{ \begin{array}{l@{\quad}l}
   3.61 & r/H \le 0.17 \\[0.1in]
   1.06 \, \left( r/H \right)^{-0.69} & 0.17 < r/H < 4.0 \end{array} \right. \label{Ucj}
\ee
where the heat release rate of the fire is contained within the non-dimensional expression:
\be
\dQ^*_H = \frac{\dQ}{\rho_\infty \, c_p \, T_\infty \, \sqrt{g} \, H^{5/2}}  \label{QsH}
\ee
To account for the presence of an upper layer, the background temperature, $T_\infty$, in Eq.~(\ref{Tcj}) is replaced with the upper layer temperature, $\Tu$. No adjustment need be made to Eq.~(\ref{QsH}) because $\rho_\infty \, T_\infty$ is assumed constant.

An estimate of the ceiling jet thickness, $\delta$, where the excess temperature drops to 1/e of its maximum value, is given by Motevalli and Marks~\cite{Alpert:SFPE}:
\be
   \frac{\delta}{H} = 0.112 \, \left[ 1-\exp\left( -2.24 \, \frac{r}{H} \right) \right] \quad \quad 0.26 \le \frac{r}{H} \le 2.0
\ee
If the detector or sprinkler is below the ceiling jet layer or in a compartment without a fire, the upper layer temperature and a default velocity of 0.1~m/s is used {\bf REFERENCE?}.

If the sprinkler or detector is in a corridor of width, $W$, Delichatsios~\cite{Alpert:SFPE} suggests the following alternative to the ceiling jet correlation, Eqs.~(\ref{Tcj}) and (\ref{Ucj}). For a given distance down the corridor, $x$, from the plume centerline, the excess temperature and velocity are estimated as:
\begin{eqnarray}
   \frac{\Delta T_{\rm cj}(x)}{\Delta \Tp} &=& 0.37 \, \left( \frac{H}{W} \right)^{1/3} \, \exp \left[ -0.16 \, \left( \frac{x}{H} \right) \left( \frac{W}{H} \right)^{1/3} \right]  \\[0.1in]
   v_{\rm cj}(x) &=& 0.114 \, \sqrt{H \, \Delta T_{\rm cj}(x) } \, \left( \frac{H}{W} \right)^{1/6}
\end{eqnarray}
where $\Delta \Tp$ is the excess plume temperature at the ceiling. This correlation is applicable once the plume has reached the full width of the corridor and the ceiling jet flow is parallel to the corridor walls so that $x>W/2$. For $x<W/2$, the normal correlations apply.


\section{Visibility}
\label{Visibility}

The visibility calculation depends solely on the soot concentration. Soot production is specified by the user in terms of a yield. Soot is transported like all other gas species and its concentration is used in the optical density calculation. The optical density is given by the expression:
\be
   D = \frac{K_m \, m_{\rm s}'''}{\ln \, 10}
\ee
where $m_{\rm s}'''$ is the mass of soot per unit volume and $K_m$ is the specific extinction coefficient. The default value is 8700~m$^2$/kg based on the recommendation of Mulholland~\cite{Mulholland:SFPE}.

\section{Smoke Detection}

CFAST does not contain an algorithm that accounts for the time delay owing to smoke transport from the fire to the detector location, nor to the penetration of smoke into the detector chamber itself. Instead, CFAST treats a smoke detector as a very sensitive heat detector with an activation temperature rise of 5~$^\circ$C and RTI of 5~(m$\cdot$s)$^{1/2}$. The temperature rise estimate is based on a study performed by Bukowski and Averill~\cite{Bukowski:1998}. The RTI is not based on experiment; it simply provides a slight time delay in activation.

Users are cautioned that this model is very crude and the uncertainty in its predictions are substantial. Consult the CFAST Validation Guide for more details on its validation.


\section{Fire Suppression} \label{sec:suppression}

Fire suppression by water is predicted using a simple empirical model developed by Madrzykowski \cite{Madrzykowski:1992} and Evans~\cite{Evans:1993}. After activation of the sprinkler, $t > t_{\rm act}$, the heat release rate is assumed to decrease exponentially:
\be
   \dQ(t) = \dQ(t_{\rm act}) \; {\rm e}^{-(t-t_{\rm act}) /\tau}   \quad ; \quad \tau = 3 \, u_{\rm w}^{-1.8}
\ee
where $u_{\rm w}$ is the water spray density, expressed in units of m/s. The product species mass production rates are reduced by the same amount as the heat release rate.

There are assumptions and limitations in this approach. Its main deficiency is that it assumes that sufficient water is applied to the fire to cause a decrease in the rate of heat release. This suppression model cannot handle the case when the fire overwhelms the sprinkler.  The suppression model as implemented does not include the effect of a second sprinkler. Detection of all sprinklers are noted but their activation does not make the fire go out any faster. Further, multiple fires in a room imply multiple ceiling jets. It is not clear how the two ceiling jets should interact. When there is more than one fire, the detection algorithm uses the fire that results in the highest ceiling jet temperature in order to calculate the sprinkler link temperature.


