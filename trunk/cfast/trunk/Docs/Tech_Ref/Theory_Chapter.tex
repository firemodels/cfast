\chapter{The Basic Transport Equations}
\label{sec:Theory_Chapter}

The equations used in CFAST take the form of an initial value problem for a system of ordinary differential equations. These equations are derived from the conservation laws of mass and energy (equivalently the first law of thermodynamics) and the ideal gas law. These equations predict as functions of time quantities such as pressure, layer height and temperatures given the gains and losses of mass and energy in the two layers. The assumption of a zone model is that properties such as temperature can be approximated throughout a control volume by an average value. Many formulations based upon these assumptions can be derived. Though equivalent mathematically, these formulations differ in their numerical solution.

The exchange of mass and enthalpy between zones is due to physical phenomena such as fire plumes, natural and forced ventilation, convective and radiative heat transfer, and so on. For example, a vent exchanges mass and enthalpy between zones in connected rooms, a fire plume typically adds heat to the upper layer and transfers entrained mass and enthalpy from the lower to the upper layer, and convection transfers enthalpy from the gas layers to the surrounding walls.


 It is assumed that each compartment is divided into two control volumes, a relatively hot upper layer and a relatively cool lower layer, as illustrated in Fig.~\ref{fig:Control_Volumes}. The gas temperature and density are assumed constant in each layer. The compartment as a whole is assumed to have a single value of pressure, $P$. It is also assumed that all thermodynamic parameters are constant. The specific heat at constant volume and at constant pressure, $c_v$ and $c_p$, the universal gas constant, $R$, and the ratio of specific heats, $\gamma$, are related by $\gamma = c_p / c_v$ and $R = c_p- c_v$.  For ambient air, $c_p \approx 1$~kJ/(kg $\cdot$ K) and $\gamma = 1.4$.
\begin{figure}[h]
\begin{center}
\includegraphics[width=\textwidth]{FIGURES/Theory/Control_Volumes}\\
\end{center}
\caption{Schematic of control volumes in a two-layer zone model.}
 \label{fig:Control_Volumes}
\end{figure}
Conservation of mass in each layer, $\dot m_i$, is expressed
\be
   \dbydt{m_i} = \dot m_i  \label{mass_con}
\ee
Conservation of energy takes the form of the first law of thermodynamics, which states that the rate of increase of internal energy plus the rate at which the layer does work by expansion is equal to the rate at which enthalpy is added to the gas:
\be
   \dbydt{(c_v m_i T_i)} +  P \dbydt{V_i} =  \dot h_i \label{eq:first_law}
\ee
The enthalpy source term, $\dot h_i$, consists of the fire's heat release rate, conduction losses to walls, and radiation exchange. The layer temperature and mass are related to the layer volume and compartment pressure via the ideal gas law:
\be
  P \, V_i = m_i \, R \, T_i \label{EoS}
\ee
A system of ordinary differential equations for the compartment pressure, upper layer volume, and layer temperatures can be derived from these three basic principles:
\begin{eqnarray}
\dbydt{P} &=& \frac{{\gamma-1}}{{V}} \left( \dhl + \dhu \right)  \\[.1in]
\dbydt{\Vu} &=& \frac{1}{P \gamma} \left( (\gamma-1) \, \dhu - \Vu \dbydt{P} \right) \\[.1in]
\dbydt{\Tu} &=& \frac{1}{c_p \, m_{\rm u}} \left( \dhu - c_p \, \dmau \, \Tu + \Vu \dbydt{P} \right) \\[.1in]
\dbydt{\Tl} &=& \frac{1}{c_p \, m_{\rm l}} \left( \dhl - c_p \, \dmau \, \Tl + \Vl \dbydt{P} \right)
\end{eqnarray}
As discussed in Refs.~\cite{Forney:1994} and \cite{Rehm:1992}, these equations are stiff, meaning that the pressure adjusts to changing conditions more quickly than the other variables. Runge-Kutta methods or predictor-corrector methods such as Adams-Bashforth require prohibitively small time steps in order to track the short time scale phenomena (pressure in our case). Methods that calculate the Jacobian (or at least approximate it) have a much larger stability region for stiff problems and are thus more successful at their solution.





\chapter{The Fire Plume}
\label{sec:TheFire}

Fires in CFAST are specified by the user in terms of a time-dependent heat release rate (HRR), an effective fuel molecule, and the yields of the products of incomplete combustion like soot and CO. Fires can be specified in multiple compartments and are treated as totally separate entities, with no interaction of the plumes. These fires are generally referred to as ``objects'' and can be ignited at a prescribed time, temperature or heat flux.

CFAST does not include a pyrolysis model to {\em predict}, as opposed to specify, the growth and spread of the fire. Rather, pyrolysis rates for each fire are prescribed by the user. While this approach does not directly account for increased pyrolysis due to radiative feedback from the flame or compartment, in theory these effects could be prescribed by the user. In an actual fire, this is an important consideration, and the specification used should consider the experimental conditions as closely as possible.

\section{Combustion Chemistry}

 The HRR of the fire is specified by the user, but it may be constrained by the availability of oxygen in the compartment. The combustion of a hydrocarbon fuel is described by the following single-step reaction:
\begin{eqnarray}
   \mathrm{C_{n_\C}H_{n_H}O_{n_O}N_{n_N}Cl_{n_{Cl}}} &+&  \nu_\OTWO \, \mathrm{O_2}  \rightarrow  \nonumber \\[.1in]
   \nu_\COTWO \, \mathrm{CO_2} &+& \nu_\HTWOO \, \mathrm{H_2O} \; + \; \nu_\CO \, \mathrm{CO} \; + \; \nu_\So \, \mathrm{Soot} \; + \; \nu_\HCl \mathrm{HCl} \; + \; \nu_\HCN \mathrm{HCN} \label{stoich}
\end{eqnarray}
The user specifies the composition of the fuel molecule and the yields of soot and CO, $y_\So$ and $y_\CO$, which are related to their stoichiometric coefficients as follows:
\begin{eqnarray}
   \nu_\So &=& \frac{M_\F}{M_\So} \; y_\So \label{soot_yield} \\[.1in]
   \nu_\CO &=& \frac{M_\F}{M_\CO} \; y_\CO \label{CO_yield}
\end{eqnarray}
Under the assumption that all of the nitrogen and chlorine in the fuel are converted to HCN and HCl, the other stoichiometric coefficients are:
\begin{eqnarray}
  \nu_\COTWO &=& \mathrm{n_\C} - \brackets{\nu_\CO + \nu_\HCN + \nu_\So} \\[.1in]
  \nu_\HTWOO &=& \frac{\mathrm{n_\Hy} - \brackets{\nu_\HCl + \nu_\HCN}}{2} \\[.1in]
  \nu_\OTWO  &=& \nu_\COTWO + \frac{\nu_\HTWOO + \nu_\CO - \mathrm{n_\Oh}}{2} \label{Oxygen_yield} \\[.1in]
  \nu_\HCl   &=& \mathrm{n_{Cl}} \\[.1in]
  \nu_\HCN   &=& \mathrm{n_{N}}
\end{eqnarray}
Note that the nitrogen in the air acts only as a diluent. The yields of hydrogen cyanide and hydrogen chloride are based solely on the composition of the fuel molecule. Finally, a user-specified trace species can be specified to follow the transport that results from fire-induced flow for an arbitrary species. This may be of particular interest for radiological releases \cite{Jones:2008}, but may be useful for any trace amounts released by a fire.

\section{Heat Release Rate}

As fuel and oxygen are consumed, heat is released and various products of combustion are formed. The heat is released as radiation and convected enthalpy:
\begin{eqnarray}
   \dQr &=& \chi_{\rm r} \, \dQ \\[.1in]
   \dQc &=& (1-\chi_{\rm r}) \, \dQ
\end{eqnarray}
where, $\chi_{\rm r}$ is the fraction  of the fire's heat release rate given off as radiation. The default value to 0.30~\cite{Drysdale:1985}.

While it is convenient for the user to directly specify the heat release rate of the fire, it is actually the pyrolysis rate of fuel, $\dmf$, that is specified:
\be
   \dmf = \frac{\dQ}{\Dh}
\ee
where $\Dh$ is the heat of combustion. In the event that the HRR is constrained by the availability of oxygen, the pyrolysis rate does not change, but the HRR becomes:
\be
   \dQ = \min \Big( \dmf \, \Dh \, , \, \dme \, Y_\OTWO \, C_{\rm LOL} \, \DhO \Big)
\ee
where $\dme$ is the entrainment rate, $Y_\OTWO$ is the mass fraction of oxygen in the layer containing the fire, $\DhO$ is the heat of combustion based on oxygen consumption\footnote{The heat of combustion based on oxygen consumption is taken to be 13.1~MJ/kg, representative of typical hydrocarbon fuels~\cite{Huggett:1980}.}, and $C_{\rm LOL}$ is the smoothing function ranging from 0 to 1:
\be 
   C_{\rm LOL} = \frac{\tanh \Big( 800 (Y_\OTWO - Y_{\OTWO,{\rm l}}) - 4 \Big) + 1}{2} 
\ee
The limiting oxygen mass fraction, $Y_{\OTWO,{\rm l}}$, is 0.1, by default. 



\section{Plume Entrainment}

The mass entrainment of air into the plume, $\dme$, is estimated using either McCaffrey's~\cite{McCaffrey:1983} or Heskestad's~\cite{Heskestad:1984} correlation. McCaffrey divides the flame/plume into three regions: 
\be
   \frac{\dme}{\dQ} = \left\{ \begin{array}{r@{\quad \quad}l}
   0.011 \brackets{\ZQf}^{0.566} &  0.00 \leq \brackets{\ZQf}<0.08 \\[.1in]
   0.026 \brackets{\ZQf}^{0.909} &  0.08 \leq \brackets{\ZQf}<0.20 \label{eq:McCaffreyPlume} \\[.1in]
   0.124 \brackets{\ZQf}^{1.895} &  0.20 \leq \brackets{\ZQf} \end{array} \right.
\ee
Heskestad  analyzed both his own data \cite{Heskestad:1984} and that of Zukoski \cite{Zukoski:1981} to develop the correlation
\be
   \dm_{\rm e} = 0.071 \, \dQ_{\rm c}^{1/3} \; \brackets{z - z_0}^{5/3} \; \brackets{1 + 0.026 \, \dQ_{\rm c}^{2/3} \, \brackets{z-z_0}^{-5/3}}
\ee
where $z_0$ is a virtual origin for the fire plume defined as
\be
  z_0/D = -1.02 + 0.083 \dQ^{2/5} / D
\ee
which is based on the total heat release rate of the fire, $\dQ$.  Both correlations provide similar results in CFAST calculations.

In CFAST, there is a constraint on the mass entrainment rate because the plume can rise only so high for a given HRR.  Early in a fire, the plume may not have sufficient energy to reach the compartment ceiling. Therefore, a limit is placed on the entrainment rate. For the plume to be able to penetrate the hot upper layer, the density of the gas in the plume must be less than the density of the gas in the upper layer. This implies that the upper layer temperature must be less than the plume temperature:
\be
   \Tu < \Tp \approx \frac{ \dQc + \dme \, c_p \, \Tl }{ \dme \, c_p} 
\ee
Rearranging terms yields a limit on the mass entrainment:
\be 
   \dm_e < \frac{\dQc}{c_p (\Tu - \Tl)}
\ee


\section{Plume Centerline Temperature}

CFAST includes an empirical correlation of plume centerline gas temperature based on the work of Baum and McCaffrey~\cite{Baum:1989} with a modification by Evans~\cite{Evans:1984} to account for the presence of a hot gas layer. The correlation gives the excess temperature as a function of height above a fire, $z$, for the flaming, intermittent, and plume regions:
\be
   \frac{\Delta \Tp}{T_\infty} = \left\{ \begin{array}{l@{\quad \quad}l}
   2.91                   &  0.00 \leq z/D^* < 1.32  \\[.1in]
   3.81 \, (z/D^*)^{-1}   &  1.32 \leq z/D^* < 3.30 \\[.1in]
   8.41 \, (z/D^*)^{-5/3} &  3.30 \leq z/D^* \end{array} \right.
\ee
where
\be
  D^* = \left( \frac{\dQ}{\rho_\infty c_p T_\infty \sqrt{g}} \right)^{2/5}
\ee
Figure~\ref{fig:Plume_Temp} shows the correlation. When a hot layer forms, the correlation must be modified since the plume now includes added enthalpy due to the entrainment of hot layer gases. Evans~\cite{Evans:1984} defines a virtual source and heat release rate to extend the plume into the upper layer. Evans' method defines the strength and location of the substitute source with respect to the interface between the upper and lower layers by

\begin{figure}
\begin{center}
\includegraphics[width=4.0in]{FIGURES/Theory/Plume_Temperature}\\
\end{center}
\caption{Excess plume centerline temperature from Baum and McCaffrey correlation.}
 \label{fig:Plume_Temp}
\end{figure}

\be Q_{I,2}^* = \brackets{\frac{1 + C_T {Q_{I,1}^*}^{2/3}}{\xi C_T} - \frac{1}{C_T}}^{3/2} \ee

\be Z_{I,2} = \brackets{\frac{\xi Q_{I,1}^* C_T}{{Q_{I,2}^*}^{1/3}\brackets{\brackets{\xi - 1}\brackets{\beta^2 + 1}+ \x1 C_T {Q_{I,2}^*}^{2/3}}}}^{2/5} Z_{I,1}  \ee

\be Q_{I,1}^* = \frac{Q_{f,C}}{\rho_\infty c_p T_\infty \sqrt{g} Z_{I,1}^{5/2}}  \ee
where $Z_{I,1}$ is the distance from the fire to the interface between the upper and lower gas layers, $Z_{I,2}$ is the distance from the virtual source to the layer interface, $\xi$ is the ratio of the upper to lower layer temperature, $\beta$ is an experimentally determined constant \cite{Zukoski:1981} ($\beta^2 = 0.913$), and $C_T = 9.115$.  The effective source strength and distance between the virtual source and target position is given by

\be Q_{f,C,eff} = Q_{I,2}^* \rho_\infty c_{p\infty} T_\infty \sqrt{g} Z_{I,2}^{5/2}  \ee

\be z_{eff} = z - Z_{I,1} + Z_{I,2} \ee
(see Fig.~\ref{fig:Plume_Temp_Notation}). The new values of the fire source and  target location are then used in the standard plume correlation where the ambient conditions are now those of the upper layer.
\begin{figure}
\begin{center}
\includegraphics[width=4.0in]{FIGURES/Theory/Plume_Temp_Notation}\\
\end{center}
\caption{Geometry for plume centerline temperature calculation.}
 \label{fig:Plume_Temp_Notation}
\end{figure}

\section{Flame Height}
\label{sec:firemassbalance}


CFAST includes a calculation of average flame height based on the work of Heskestad~\cite{Heskestad:2002}. Valid for a wide range of hydrocarbon and gaseous fuels, the correlation is given by
\be
   H = -1.02 D + 0.235 \brackets{\frac{Q_f}{1000}}^{2/5}
\ee
where $H$ is the average flame height (m), $D$ the diameter of the fire (m), and $\dQ$ is the total heat release rate (kW). The mean flame height is defined as the distance from the fuel source to the top of the visible flame where the intermittency is 0.5.  A flame intermittency of 0.5 means that the visible flame is above the mean 50 \% of the time and below the mean 50~\% of the time.  This average flame height is  included in the printed output from CFAST.







\chapter{Ventilation}

Flow through vents is a dominant component of any fire model because it is sensitive to small changes in pressure and transfers the greatest amount of enthalpy on an instantaneous basis of all the source terms (except of course for the fire and plume).  Its sensitivity to environmental changes arises through its dependence on the pressure difference between compartments which can change rapidly.

Flow through vents can be forced (mechanical) or natural (convective). CFAST models three types of vent flow, natural flow through vertical vents (such as doors or windows),  natural flow through horizontal vents (such as ceiling holes or hatches) and forced flow throught fans.  Horizontal flow is the flow which is normally thought of when discussing fires.  Vertical flow is particularly important in two disparate situations: a ship, and the role of fire fighters doing roof venting.

Vent flow is determined using the pressure difference across a vent.  Natural vent flow at a given elevation may be computed using Bernoulli's law by first computing the pressure difference at that elevation.  The pressure on each side of the vent is computed using the pressure at the floor, the height of the floor and the density.  Forced flow can occur through either vertical or horizontal vents. The differences are primarily the selection rules for the source of the gases or whether the resultant plume enters the lower or upper layer of each compartment.

Atmospheric pressure is about 100~000 Pa. Fires produce pressure changes from 1~Pa to 1~000~Pa and mechanical ventilation systems typically involve pressure differentials of about 1~Pa to 100~Pa.  The pressure variables are solved to a higher accuracy than other solution variables because of the subtraction (with resulting loss of precision) needed to calculate vent flows from pressure differences.

Mass flow (in the remainder of this section, the term ``flow'' will be used to mean mass flow) is the dominant source term for the predictive equations because it fluctuates most rapidly and transfers the greatest amount of enthalpy on an instantaneous basis of all the source terms (except of course the fire).  Also, it is most sensitive to changes in the environment.  Horizontal flow encompasses flow through doors, windows and so on. Horizontal flow is discussed in section 3.4.3.1. Vertical flow occurs in ceiling vents.  It is important in two separate situations: on a ship with open hatches and in house fires with roof venting.  Vertical flow is discussed in section 3.4.3.2.

There is a special case of horizontal flow for long corridors. A corridor flow algorithm is incorporated to calculate the time delay from when a plume enters a compartment to when the effluent is available for flow into adjacent compartments.

Flow through vents can be modified, that is turned on or off. This applies to the three types of vents discussed below, horizontal flow through vertical vents (HVENT), vertical flow through horizontal vents (VVENT) and forced flow (MVENT). For each key word, there is a an initial opening fraction which is reflected in the first region in Fig.~\ref{fig:Opening_Fraction}. This initial opening fraction can be modified by by the EVENT key word to change the fraction. This change occurs over a transition time which defaults to one second. The final fraction is the third region depicted in Fig.~\ref{fig:Opening_Fraction}. There can be only a single transition per vent.

\begin{figure}[t]
\begin{center}
\includegraphics[width=4.0in]{FIGURES/Theory/Opening_Fraction}\\
\end{center}
\caption{Vent opening size fraction as a function of time.}
 \label{fig:Opening_Fraction}
\end{figure}

\subsection{Horizontal Flow Through Vertically-Oriented Vents (Doors and Windows)}

Flow through normal vents such as windows and doors is governed by the pressure difference across a vent.  A momentum equation for the zone boundaries is not solved directly.  Instead momentum transfer at the zone boundaries is included by using an integrated form of Euler's equation, namely Bernoulli's solution for the velocity equation.  This solution is augmented for restricted openings by using flow coefficients \cite{Quintiere:1984, Steckler_Coefficients} to allow for constriction from finite size doors.  The flow (or orifice) coefficient is an empirical term which addresses the problem of constriction of velocity streamlines at an orifice.

Bernoulli's equation is the integral of the Euler equation and applies to general initial and final velocities and pressures.  The implication of using this equation for a zone model is that the initial velocity in the doorway is the quantity sought, and the final velocity in the target compartment vanishes.  That is, the flow velocity vanishes where the final pressure is measured.  Thus, the pressure at a stagnation point is used.  This is consistent with the concept of uniform zones which are completely mixed and have no internal flow.

The mass flow through a region is found by first noting that the
velocity of the flow at an elevation $h$ is given by

\newcommand{\dmass}{d\dot{m}}
\be
v(h) = C \sqrt{\frac{2 \Delta P(h)}{\rho}}
\label{eq:vent_velocity}
\ee

\noindent where $C$ is the constriction (or flow) coefficient (taken to be 0.7 in CFAST \cite{Steckler_Coefficients}), $\rho$ is the gas density on the source side, and $\Delta P(h)$ is the pressure across the interface
at elevation $h$.  At present we use a constant value for $C$ for all gas temperatures.

The differential mass flow, $\dmass(h)$, at elevation $h$ through a region of width $w$ and differential height $dh$ is found using equation \ref{eq:vent_velocity} after noting that $\dmass(h)=v(h)\rho wdh$
to obtain

\be
\dmass(h)=C \sqrt{2 \rho\Delta P(h)}w dh
\label{eq:vent_massflow}
\ee

The total mass flow rate through a slab is found by integrating $\dmass(h)$ vertically over that slab.
The simplest means to define the limits of integration is with neutral planes, that is the height at which flow reversal occurs, and physical boundaries such as sills and soffits.  The mass flow equation can be integrated piecewise analytically and then summed by breaking the integral into intervals defined by flow reversal, a soffit, a sill, or a zone interface, .

The approach to calculating the flow field is of some interest.  The flow calculations are performed as follows.  The vent opening is partitioned into at most six slabs where each slab is bounded by a layer height, neutral plane, or vent boundary such as a soffit or sill.  The most general case is illustrated in Fig.~\ref{fig:Flow_Notation}.

\begin{figure}[t]
\begin{center}
\includegraphics[width=6.0in]{FIGURES/Theory/Flow_Notation}\\
\end{center}
\caption{Geometry and notation for horizontal flow vents in a two-zone fire model.}
 \label{fig:Flow_Notation}
\end{figure}

Let $b$ and $t$ denote the bottom and top slab elevations and $P_b$ and $P_t$ denote the cross-vent pressures at those elevations.  Because of the way that slabs are defined, the two cross pressures $P_b$ and $P_t$ will have the same sign. The mass flow through the slab can then be computed by integrating equation \ref{eq:vent_massflow} vertically from $b$ to $t$ to obtain

\begin{eqnarray}
\dm&=&\int_b^t \dmass(h)\\
&=&C\sqrt{2\rho}w\int_b^t\sqrt{\frac{|P_b(t-h)+P_t(h-b)|}{t-b}}dh\\
&=&\frac{2}{3}C\sqrt{2\rho}w(t-b)\frac{|P_t|^{3/2}-|P_b|^{3/2}}{|P_t|-|P_b|}
\label{eq:massflowone}
\end{eqnarray}

\noindent after noting that
\begin{eqnarray*}
\int \sqrt{A+Bh}dh=\frac{2}{3B}(A+Bh)^{3/2}+\mbox{constant}
\end{eqnarray*}

\noindent where $B=(|P_t|-|P_b|)/(t-b)$, $A+Bt=|P_t|$ and $A+Bb=|P_b|$.
Equation \ref{eq:massflowone} can be simplified to

\be \dm_{io} = \frac{2}{3} C \sqrt{2 \rho} A_{slab} \brackets{\frac{x^2+xy+y2}{x+y}} \ee

\noindent where $x = \sqrt{|P_t|}$, $y = \sqrt{|P_b|}$, $A_{slab}=w(t-b)$ the cross-sectional area of the slab.  The value of the density, $\rho$, is taken from the source compartment.

A mixing phenomenon occurs at vents which is similar to entrainment in plumes.  As hot gases from one compartment leave that compartment and flow into an adjacent compartment a door jet can exist which is analogous to a normal plume.  Mixing of this type occurs for $\dm_{13} > 0$ as shown in Fig.~\ref{fig:Flow_Patterns}.  To calculate the entrainment ($\dm_{43}$ in this example), once again we use a plume description consistent with the work of McCaffrey, but with an virtual origin.  The estimate for the virtual origin is given by Cetegen et al. \cite{Cetegen:1984}.  This  is chosen so that the flow at the door opening would correspond to a plume with the heating for a equivalent doorway fire source (with respect to the lower layer) given by

\begin{figure}[t]
\begin{center}
\includegraphics[width=5.0in]{FIGURES/Theory/Flow_Patterns}\\
\end{center}
\caption{Flow patterns and layer number conventions for horizontal flow through a vertical vent.}
 \label{fig:Flow_Patterns}
\end{figure}

\be Q_{f,eq} = c_p \brackets{T_1-T_4} \dm_{13} \ee
where $Q_{f,eq}$ is the heat release rate of the doorway fire. The concept of the virtual origin is that the enthalpy flux from the virtual point source should equal the actual enthalpy flux in the door jet at the point of exit from the vent using the same prescription.  Thus the entrainment is calculated the same way as was done for a normal plume.  The reduced height of the plume, $z_p$, (in units of m/kW$^{2/5}$) is

\be z_p = \frac{z_{13}}{Q_{f,eq}^{2/5}}+ v_p \ee
where $v_p$ the virtual point source, is defined by inverting the entrainment process to yield

\begin{eqnarray}
v_P = \brackets{\frac{8.10 \dm}{Q_{f,eq}}}^{0.528} &   0 < \brackets{\frac{\dm}{Q_{f,eq}}} < 0.0061 \nonumber \\
v_P = \brackets{\frac{38.5 \dm}{Q_{f,eq}}}^{1.1001} & 0.0061 < \brackets{\frac{\dm}{Q_{f,eq}}} \leq 0.026 \\
 v_P = \brackets{\frac{90.9 \dm}{Q_{f,eq}}}^{1.76} & \brackets{\frac{\dm}{Q_{f,eq}}} > 0.026  \nonumber   \nopagebreak
\end{eqnarray}

Although outside of the normal range of validity of the plume model, a level of agreement with experiment is apparent (section 6 includes discussion of validation experiments for the plume model).  Since a door jet forms a flat plume whereas a normal fire plume will be approximately circular, strong agreement is not expected.

The other type of mixing is much like an inverse plume and causes contamination of the lower layer.  It occurs when there is flow of the type $\dm_{42} > 0$.  The shear flow causes vortex shedding into the lower layer and thus some of the particulates end up in the lower layer.  The actual amount of mass or energy transferred is usually not large, but its effect can be large.  For example, even minute amounts of carbon can change the radiative properties of the gas layer, from negligible to something finite.  It changes the rate of radiation absorption significantly and invalidates the simplification of an ambient temperature lower layer.  This term is predicated on the Kelvin-Helmholz flow instability and requires shear flow between two separate fluids.  The mixing is enhanced for greater density differences between the two layers. However, the amount of mixing has never been well characterized. Quintiere et al. \cite{Quintiere:1984} discuss this phenomena for the case of crib fires in a single room, but their correlation does not yield good agreement with experimental data in the general case \cite{Quintiere:1981}.  In the CFAST model, it is assumed that the incoming cold plume behaves like the inverse of the usual door jet between adjacent hot layers; thus we have a descending plume.  The same equations are used to calculate this inverse plume as are used for the upright door mixing, above. It is possible that the entrainment is overestimated in this case, since buoyancy, which is the driving force, is not nearly as strong as for the usually upright plume.

\subsection{Vertical Flow Through Horizontally-Oriented Vents (Floor and Ceiling Vents)}

Flow through a ceiling or floor vent can be somewhat more complicated than through door or window vents.  The simplest form is uni-directional flow, driven solely by a pressure difference.  This is analogous to flow in the horizontal direction driven by a piston effect of expanding gases.  Once again, it can be calculated based on the Bernoulli equation, and presents little difficulty.  However, in general we must deal with more complex situations that must be modeled in order to have a proper understanding of smoke movement.  The first is an occurrence of puffing.  When a fire exists in a compartment in which there is only one hole in the ceiling, the fire will burn until the oxygen has been depleted, pushing gas out the hole.  Eventually the fire will die down.  At this point ambient air will rush back in, enable combustion to increase, and the process will be repeated.  Combustion is thus tightly coupled to the flow.  The other case is exchange flow which occurs when the fluid configuration across the vent is unstable (such as a hotter gas layer underneath a cooler gas layer).  Both of these pressure regimes require a calculation of the onset of the flow reversal mechanism.

Normally a non-zero cross vent pressure difference tends to drive unidirectional flow from the higher to the lower pressure side.  An unstable fluid density configuration occurs when the pressure alone would dictate stable stratification, but the fluid densities are reversed.  That is, the hotter gas is underneath the cooler gas.  Flow induced by such an unstable fluid density configuration tends to lead to bi-directional flow, with the fluid in the lower compartment rising into the upper compartment.  This situation might arise in a real fire if the room of origin suddenly had a hole punched in the ceiling. No pretense is made of being able to do this instability calculation analytically. Cooper's algorithm \cite{Cooper:1989} is used for computing mass flow through ceiling and floor vents. It is based on correlations to model the unsteady component of the flow.  What is surprising is that we can find a correlation at all for such a complex phenomenon. There are two components to the flow.  The first is a net flow dictated by a pressure difference. The second is an exchange flow based on the relative densities of the gases.  The overall flow is given by \cite{Cooper:1989, Cooper:1990, Cooper:1995}

\be \dm = C f\brackets{\gamma, \epsilon} \sqrt{\frac{\Delta P}{\overline{\rho}}} A_v \ee
where $\gamma = c_p/c_v$ is the ratio of specific heats, $C = 0.68 + 0.17 \epsilon$, $\epsilon = \frac{\Delta P}{P}$, and $f$ us a weak function of both $\gamma$ and $\epsilon$ \cite{Cooper:1989}. In the situation where we have an instability, we use Cooper's correlations for the function $f$.  The resulting exchange flow is given by

\be \dm_{ex} = 0.1 \brackets{\frac{g \Delta \rho A_v^{5/2}}{\rho_{av}}} \brackets{1.0 - \frac{2 A_v^2 \Delta P}{S^2 g \Delta \rho D^5}} \ee
where $D = 2 \sqrt{A_v / \pi}$ and $S$ is 0.754 for round or 0.942 for square openings, respectively \cite{Cooper:1989}. Vertical flow through horizontal vents is governed by the VFLOW routines. VENTCF is the module which calculates the mass flow from one compartment to another. The values returned are $\dm_{incoming}$ and $\dm_{outgoing}$ through each vent. These terms are symmetric: the outgoing flow from compartment 1 to 2 is the same as incoming flow from compartment 2 to 1, though source and destination layers may be different.

The energy flux into a compartment is then determined by the relative size and temperature of the layers of the compartment from which the mass is flowing (incoming, u and l):

\be \dq_{incoming} = c_p \dm_u T_u + c_p \dm_l T_l \ee

\be \dm_u = \dm_{incoming} \frac{V_u}{V} \ee

\be \dm_l = \dm_{incoming} \frac{V_l}{V} \ee


The mass and energy are then deposited into the upper or lower layer of the receiving compartment based on the effective temperature of the incoming flow relative to the upper and lower layers of the receiving compartment. If the temperature of the incoming flow is higher than the temperature of the lower layer, then the flow is deposited into the upper layer. This is similar to the usual plume from a fire or a doorway jet. These rules are implemented in VFLOW.

\subsection{Forced Flow}

Forced flow in this version of CFAST is a supply (or exhaust) system based on constant flow through a opening/fan/opening triplet . These systems are commonly used in buildings for heating, ventilation, air conditioning, pressurization, and exhaust. Figure \ref{fig:Fans_and_Ducts}(a) shows smoke management by an exhaust fan at the top of an atrium, and Fig.~\ref{fig:Fans_and_Ducts}(b) illustrates a kitchen exhaust.  Cross ventilation, shown in Fig.~\ref{fig:Fans_and_Ducts}(c), is occasionally used without heating or cooling.  Generally systems that maintain comfort conditions have either one or two fans.

\begin{figure}
\begin{center}
\includegraphics[width=5.0in]{FIGURES/Theory/HVAC_Fans_and_Ducts}\\
\end{center}
\caption{Some simple fan-duct systems.}
 \label{fig:Fans_and_Ducts}
\end{figure}

Further information about these systems is presented in Klote and Milke \cite{Klote:2002} and the American Society of Heating, Refrigerating and Air Conditioning Engineers (ASHRAE) \cite{ASHRAE:2001}.

This version of the model does not include duct work or variable fans. These equations are high-order, non-linear and in some cases ill-posed, which caused a great deal of difficulty in reaching a numerical solution.

The flow through mechanical vents can be filtered. Filtering affects particulates such as smoke and the trace species. Filtering can be turned on at any time. Effectiveness is from 0~\% (no effect) to 100~\% which completely blocks flow of these two species.





\chapter{Heat Transfer}

This section discusses radiation, convection and conduction, the three mechanisms by which heat is transferred between the gas layers and the enclosing compartment walls.  This section also discusses heat transfer algorithms for calculating target temperatures.

Gas layers exchange energy with their surroundings via convective and radiative heat transfer.  Different material properties can be used for the ceiling, floor, and walls of each compartment (although all the walls of a compartment must be the same).  Additionally, CFAST allows each surface to be composed of up to three distinct layers.  This allows the user to deal naturally with the actual building construction.  Material thermophysical properties are assumed to be constant, although we know that they actually vary with temperature. The user should also recognize that the mechanical properties of some materials may change with temperature, but these effects are not modeled.						

Radiative transfer occurs among the fire(s), gas layers and compartment surfaces (ceiling, walls and floor).  This transfer is a function of the temperature differences and the emissivity of the gas layers as well as the compartment surfaces.  Typical surface emissivity values only vary over a small range.  For the gas layers, however, the emissivity is a function of the concentration of species which are strong radiators, predominately smoke particulates, carbon dioxide, and water.  Thus errors in the species concentrations can give rise to errors in the distribution of enthalpy among the layers, which results in errors in temperatures, resulting in errors in the flows.  This illustrates just how tightly coupled the predictions made by CFAST can be.

\section{Radiation}
\label{sec:Radiation}

Radiation heat transfer forms a significant portion of the energy balance in a zone fire model, especially in the fire room.  Radiative heat transfer is computed from wall and gas temperatures, emisivities and fire heat release rates.  To calculate the radiation absorbed in a zone, a heat balance must be done accounting for all surfaces that radiate to and absorb radiation from a zone.

A radiation heat transfer calculation can easily dominate the computational requirements of any fire model.  Approximations are then required to perform these calculations in a time consistent with other zone fire model sources terms.  For example, it is assumed that all zones and surfaces radiate and absorb like a gray body, that the fires radiate as point sources and that the plume does not radiate at all.  Radiative heat transfer is approximated using a limited number of radiating wall surfaces, four in the fire room and two everywhere else.  The use of these and other approximations allows CFAST to perform the radiation computation in a reasonably efficient manner \cite{Forney_radiation}.

{ \bf Modeling Assumptions:}  The following assumptions are made in order to simplify the radiation heat exchange model used in CFAST and to make its calculation tractable.

\begin{itemize}
\item Iso-thermal - Each gas layer and each wall segment is assumed to be at a uniform temperature.

\item Equilibrium - The wall segments and gas layers are assumed to be in a quasi-steady state.  In other words, the wall and gas layer temperatures are assumed to change slowly over the duration of the time step of the associated differential equation.

\item Point Source Fires - The fire is assumed to radiate uniformly in all directions giving off a fraction, $\chi_R$, of the total energy release rate.  This radiation is assumed to originate from a single point.  Radiation feedback to the fire and radiation from the plume is not modeled in the radiation exchange algorithm.

\item Diffuse and gray surfaces - The radiation emitted is assumed to be diffuse and gray.  In other words, the radiant fluxes emitted are independent of direction and wavelength.  These assumptions allow us to infer that the emittance, $\epsilon$, absorptance, $\alpha$ and reflectance, $\rho$, are related via $\epsilon = \alpha = 1 - \rho$.

\item Geometry - Rooms or compartments are assumed to be rectangular boxes.  Each wall is either perpendicular or parallel to every other wall.  Radiation transfer through vent openings is lost from the room.
\end{itemize}

{\bf 4-Wall and 2-Wall Radiation Exchange:} When computing wall temperatures, CFAST partitions a compartment into four parts; the ceiling, the floor, the wall segments above the layer interface and the wall segments below the layer interface.  The radiation algorithm then computes a heat flux striking each wall segment using the surface temperature and emissivity.

The four wall algorithm used in CFAST for computing radiative heat exchange is based upon the equations developed in Siegel and Howell \cite{SiegelandHowell:1981} which in turn is based on the work of Hottel \cite{Hottel:1954}. Siegel and Howell model an enclosure with N wall segments and a homogeneous gas. A radiation algorithm for a two layer zone fire model requires treatment of an enclosure with two uniform gases.  Hottel and Cohen \cite{Hottel:1958} developed a method where the enclosure is divided into a number of wall and gas volume elements. An energy balance is written for each element. Each balance includes interactions with all other elements.  Treatment of the fire and the interaction of the fire and gas layers with the walls is based upon the work of Yamada and Cooper \cite{Yamada:1990}.  They model fires as point heat sources radiating uniformly in all directions and use the Lambert-Beer law to model the interaction between heat emitting elements (fires, walls, gas layers) and the gas layers. By implementing a four wall rather than an N wall model, significant algorithmic speed increases are achieved.  This is done by exploiting the simple structure and symmetry of the four wall problem.

The nomenclature used in this section follows that of Siegel and Howell \cite{SiegelandHowell:1981}.  The radiation exchange at the k'th surface is shown schematically in Fig.~\ref{fig:Rad_Exchange}.  For each wall segment k from 1 to N, a net heat flux, $\Delta \hat{q}_k\dprime$, must be found such that the energy balance,

\be \sigma A_k \epsilon_k T_k^4 + \brackets{1 - \epsilon_k}q_k^{in} = q_k^{in} + A_k \Delta q_k\dprime \ee

at each wall segment $k$ is satisfied, where $\sigma$ is the Stefan-Boltzman constant, $A_k$ is the area of the k'th wall segment , $\epsilon_k$ is the emissivity of the k'th wall segment, $T_k$ is the temperature of the k'th wall segment and $q_k^{in}$ is the energy arriving at the k'th wall segment from all other wall segments and heat sources.

Radiation exchange at each wall segment considers the emitted, reflected, incoming and net radiation terms.  The unknown net radiative fluxes, $\Delta q_k\dprime$ , are found by solving the modified net radiation equation

\be \Delta \hat{q}_k\dprime - \displaystyle\sum_{j=1}^N \brackets{1 - \epsilon_j} \Delta \hat{q}_j\dprime F_{k-j} \tau_{j-k} = \sigma T_k^4 - \displaystyle\sum_{j=1}^N \brackets{\sigma T_k^4 F_{k-j} \tau_{k-j}} - \frac{c_k}{A_k} \ee
where $\Delta \hat{q}_k\dprime = \Delta q_k / \epsilon$, $F_{k-j}$ is the configuration factor, $\tau$ is the transmittance and other terms are previously defined.

\begin{figure}
\begin{center}
\includegraphics[width=5.0in]{FIGURES/Theory/Radiation_Exchange}\\
\end{center}
\caption{Radiation Exchange in a two-zone fire model.}
 \label{fig:Rad_Exchange}
\end{figure}

The walls can be modeled using two surfaces or four.  The four wall model is necessary for fire rooms because the temperatures of the ceiling and upper walls differ significantly.  The two wall model is used for compartments that contain no fires.

To simplify the comparison between the two and four wall segment models, assume that the emissivities of all wall segments are one and that the gas absorptivities are zero.  Let the room dimensions be 4 m by 4 m by 4 m, the temperature of the floor and the lower and upper walls be 300 K, and the ceiling temperature vary from 300 K to 600 K. Figure \ref{fig:Rad_2vs4} shows a plot of the heat flux to the ceiling and upper wall as a function of the ceiling temperature \cite{Forney_radiation, Jones:1993}. The two wall model predicts that the extended ceiling (a surface formed by combining the ceiling and upper wall into one wall segment) cools, while the four wall model predicts that the ceiling cools and the upper wall warms.  The four-wall model better accounts for temperature differences that may exist between the ceiling and upper wall (or floor and lower wall) by allowing heat transfer to occur between the ceiling and upper wall. This problem is not as significant in compartments where a fire is not present.

Reference \cite{Forney_radiation} documents how to minimize the work required to compute the 16 configuration factors, $F_{k-j}$, required in a 4 wall model.

\begin{figure}
\begin{center}
\includegraphics[width=5.0in]{FIGURES/Theory/Radiation_2vs4}\\
\end{center}
\caption{An example of the calculated two-wall (RAD2) and four-wall (RAD4) contributions to radiation exchange on a ceiling and wall surface.}
 \label{fig:Rad_2vs4}
\end{figure}

{\bf Configuration Factors:}  A configuration factor between two finite areas denoted $F_{1-2}$ is the fraction of radiant energy given off by surface 1 that is intercepted by  surface 2 and is given by

\be F_{1-2} = \frac{1}{A_1} \int_{A_1} \int_{A_2} \frac{\cos \theta_1 \theta_2}{\pi L^2} dA_2 dA_1 \label{eq:config_factor} \ee
where $L$ is the distance along the line of integration,  $\theta_1$ and $\theta_2$ are the angles for surface 1 and 2 between the respective normal vectors and the line of integration, and $A_1$ and $A_2$ are the areas of the two surfaces.  These terms are illustrated in Fig.~\ref{fig:Rad_Config_Factor}.  When the surfaces $A_1$ and $A_2$ are far apart relative to their surface area, eq (\ref{eq:config_factor}) can be approximated by assuming that $\theta_1$, $theta_2$ and $L$ are constant over the region of integration to obtain

\begin{figure}
\begin{center}
\includegraphics[width=3.0in]{FIGURES/Theory/Radiation_Config_Factor}\\
\end{center}
\caption{Setup for a configuration factor calculation between two arbitrarily oriented finite areas.}
 \label{fig:Rad_Config_Factor}
\end{figure}

\be F_{1-2} = \frac{\cos \theta_1 \theta_2}{\pi L^2} A_2 \ee

{\bf Transmittance and Absorptance:} The transmittance of a gas volume is the fraction of radiant energy that will pass through it unimpeded and is given by

\be \tau (L) = e^{-\alpha L} \ee
where $\alpha$ is the absorptance of the gas volume and $L$ is a characteristic path length.

The absorptance, $\alpha$, of a gas volume is the fraction of radiant energy absorbed by that volume.  For a gray gas, $\alpha + \tau = 1$.

{\bf Calculating absorption for broad band gas layer radiation:}  In general, the transmittance and absorptance are a function of wavelength.   This is an important factor to consider for the major gaseous products ($\textnormal{CO}_2$  and $\textnormal{H}_2 \textnormal{O}$); however soot has a continuous absorption spectrum which allows the transmittance and absorptance to be approximated as ``gray" \cite{SiegelandHowell:1981} across the entire spectrum.

The gas absorptance, $\alpha_G$, is due to the combination of the $\textnormal{CO}_2$  and $\textnormal{H}_2 \textnormal{O}$ and is given by

\be \alpha_G = \alpha_{H_2O} + \alpha_{CO} - C\ee
where $C$ is a correction for band overlap.  For typical fire conditions, the overlap amounts to about half of the $\textnormal{CO}_2$ absorptance \cite{Tien:2002} so the gas transmittance is approximated by

\be \tau_G = 1 - \alpha_{H_2O} - 0.5 \alpha_{CO_2} \ee

The total transmittance of a gas-soot mixture is the product of the gas and soot transmittances, $\tau_T = \tau_S \tau_G$ so that

\be \tau_T = e^{-al} \brackets{1 - \alpha_{H_2O} - 0.5 \alpha_{CO_2}} \ee

In the optically thin limit the absorption coefficient, a, may be replaced by the Planck mean absorption coefficient and in the optically thick limit, it  may be replaced by the Rosseland mean absorption coefficient. For the entire range of optical thicknesses, Tien et al. \cite{Tien:2002} report that a reasonable approximation is $\alpha = k f_v T$ where $k$ is a constant that depends on the optical properties of the soot particles, $f_v$ is the soot volume fraction and $T$ is the soot temperature in Kelvin. Values of $a$, have been found to be about constant for a wide range of fuels \cite{Tien:1978}.   The soot volume fraction, $f_v$, is calculated from the soot mass, soot density and layer volume.  The soot is assumed to be in thermal equilibrium with the gas layer.

Edwards' absorptance data for $\textnormal{H}_2 \textnormal{O}$ and $\textnormal{CO}_2$ are reported \cite{Edwards:1985} as log(emissivity) versus log(pressure-pathlength), with  log(gas concentration) as a parameter. For each gas, these data were incorporated into a look-up table, implemented as a two-dimensional array of log(emissivity) values, with indices based on temperature and gas concentration.  It is assumed that absorptance and emittance are equivalent for the gaseous species as well as for soot.

An effective path length ( mean beam length, L) treats an emitting volume as if it were a hemisphere of a radius such that the flux impinging on the center of the circular base is equal to the average boundary flux produced by the real volume. The value of this radius is approximated as \cite{Tien:1978, Hottel:1942} $L = c 4 V / A$ where $L$ is the mean beam length in meters, $c$ is a constant (approximately 0.9, for typical geometries), $V$ is the emitting gas volume m$^3$ and $A$ is the surface area (m$^2$) of the gas volume. The volume and surface area are calculated from the dimensions of the layer.

For each gas, the log(absorptance) is estimated from the look-up table for that gas  by  interpolating both the log(temperature) and log(concentration) domains. In the event that the required absorptance lies outside the temperature or concentration range of the look-up table, the nearest acceptable value is returned. Error flags are also returned, indicating whether each parameter was in or out of range and, in the latter case, whether it was high or low.  This entire process is carried out for both $\textnormal{CO}_2$  and $\textnormal{H}_2 \textnormal{O}$.

\section{Computing Target Heat Flux and Temperature}

The calculation of the radiative heat flux to a target is similar to the radiative heat transfer calculation discussed previously.  The main difference is that CFAST does not compute feedback from the target to the wall surfaces or gas layers.  The target is simply a probe or sensor not interacting with the modeled environment.

The net flux striking a target can be used as a boundary condition in order to compute the temperature of the target.  If the target is thin, then its temperature quickly rises to a level where the heat flux to and from the target are in equilibrium.

There are four components of heat flux to a target: fires, walls (including the ceiling and floor), gas layer radiation and gas layer convection.

{\bf Heat Flux from a Fire to a Target:} Figure \ref{fig:Rad_Fire} illustrates terms used to compute heat flux from a fire to a target. Let $n_t$ be a unit vector perpendicular to the target and $\theta_t$ be the angle between the vectors $\overline{L}$ and $n_t$.

\begin{figure}
\begin{center}
\includegraphics[width=3.0in]{FIGURES/Theory/Radiation_Fire}\\
\end{center}
\caption{Radiative heat transfer from a point source fire to a target.}
 \label{fig:Rad_Fire}
\end{figure}

Using the definition that $q_{f,r}$ is the radiative portion of the energy release rate of the fire, then the heat flux on a sphere of radius $L$ due to this fire is $q_{f,r} / \brackets{4 \pi L^2}$. Correcting for the orientation of the target and accounting for heat transfer through the gas layers, the heat flux to the target is

\be q_{f,r}\dprime = \frac{q_{f,r}}{4 \pi L^2} \cos \brackets{\theta_t} \tau_U (L_U) \tau_L (L_L) = -q_{f,r} \frac{n_t \overline{L}}{4 \pi L^3} \tau_U (L_U) \tau_L (L_L) \ee

{\bf Radiative Heat Flux from a Wall Segment to a Target:} Figure \ref{fig:Rad_Gases} illustrates terms used to compute heat flux from a wall segment to a target. The flux, $q\dprime_{w,t}$, from a wall segment to a target can then be computed using

\be q_{w,t}\dprime = \frac{A_w q_{w,out}\dprime F_{w-t}}{A_t} \tau_U (L_U) \tau_L (L_L) \label{eq:wall_target_flux} \ee
where $q_{w,t}\dprime$  is the flux leaving the wall segment, $A_w$, $A_t$ are the areas of the wall segment and target respectively, $F_{w-t}$  is the fraction of radiant energy given off by the wall segment that is
intercepted by the target (i.e., a configuration factor) and $\tau_U (L_U)$ and $\tau_L (L_L)$ are defined as before. Equation (\ref{eq:wall_target_flux})  can be simplified using the symmetry relation $A_w F_{w-t}  = A_t F_{t-w}$ to obtain

\begin{figure}
\begin{center}
\includegraphics[width=5.0in]{FIGURES/Theory/Radiation_Gases}\\
\end{center}
\caption{Radiative heat transfer from the upper and lower layer gas to a target in the lower layer.}
 \label{fig:Rad_Gases}
\end{figure}

\be q_{w,t}\dprime = q_{w,out}\dprime F_{t-w} \tau_U (L_U) \tau_L (L_L) \ee
where

\be q_{w,out}\dprime = \sigma T_w^4 - \brackets{1 - \epsilon_w} \frac{\Delta q_w\dprime}{\epsilon_w}, \ee
$T_w$ is the temperature of the wall segment, $\epsilon_w$ is the emissivity of the wall segment and $\Delta q_w\dprime$ is the net flux striking the wall segment.

{\bf Radiation from the Gas Layer to the Target:} Figure \ref{fig:Rad_Gases} illustrates the setup for calculating the heat flux from the gas layers to the target.  The upper and lower gas layers in a room contribute to the heat flux striking the target if the layer absorptances are non-zero.

Let $q_{w,t,gas}\dprime$ denote the flux striking the target due to the gas g in the direction of wall segment w. Then

\be q_{w,t,gas}\dprime =
\left\{
\begin{array}{ll}
\sigma F_{t-w} \brackets{T_L^4 \alpha_L \tau_U + T_U^4 \alpha_U} & w \textnormal{ is in the lower layer}\\
\sigma F_{t-w} \brackets{T_U^4 \alpha_U \tau_L + T_L^4 \alpha_L} & w \textnormal{ is in the upper layer}
\end{array}
\right. \label{eq:Rad_Gases} \ee

The total target flux due to the gas (upper or lower layer) is obtained by summing eq (\ref{eq:Rad_Gases} over each wall segment or

\be q_{g,t}\dprime = \sum_w = q_{w,t,gas}\dprime  \ee

{\bf Computing the Steady State Target Temperature:} The steady state target temperature, $T_t$ can be found by solving an energy balance on the target; namely

\be \epsilon_t \sigma T_t^4 = \epsilon_t q_{r,in} + h \brackets{T_g - T_t} \label{eq:Target_Temp} \ee
Note that the local gas temperature, $T_g$, in the convection calculation, $h \brackets{T_g - T_t}$, is taken to be either the upper layer temperature if the target is located in the upper layer, the lower layer temperature if the target is located in the lower layer, or the plume centerline temperature if the target is located directly above a fire source.

Let $f(T_t)$ be the difference between the left and right hand side of equation (\ref{eq:Target_Temp}).  Then this equation may be solved using the Newton iteration

\be T_{new} = T_{old} - \frac{f(T_{old})}{f^\prime(T_{old})} \label{eq:Target_Newton} \ee

Equation (\ref{eq:Target_Newton}) is iterated until the difference $T_{new} - T_{old}$ is sufficiently small.

{\bf Computing the Transient Rectangular Target Temperature:}
A transient target temperature may be computed using two different methods
depending on whether the target is assumed to be thin or thick.  A thin target
is presumed to have a constant interior temperature profile. A differential equation model may then be used to estimate the
temperature rise (or fall) based upon the thermal properties of the target and the
heat flux striking the front and back sides; namely

\be c \rho V \frac{dT}{dt} = A(q''_f+q''_b) \label{eq:Target_ODE} \ee

where $c$, $\rho$ and $V$ are the the specific heat, density and volume of the target, $A$ is the cross-sectional area of the target and the two $q''$ terms are the heat flux (due to all sources) striking the front and back sides of the target.

Equation (\ref{eq:Target_ODE}) may be solved implicitly or explicitly.  When solved implicitly, the target temperature is added to the set of solution variables and equation (\ref{eq:Target_ODE}) is added to the equation set solved by DASSL.
When solved explicitly, equation (\ref{eq:Target_ODE}) is solved as a stand-alone equation advancing the target temperature in time.


If the target is thick then it is presumed that the temperature profile within the target varies as a function of depth and therefore a partial differential equation model must be used to estimate the changing profile; namely the heat equation

\be \frac{\partial T}{\partial t} = \frac{k}{\rho C}\frac{\partial^2T}{\partial x^2}
\label{eq:Target_PDE} \ee
where $k$, $\rho$ and $C$ are the thermal conductivity, density and heat capasity of the target.  As with the standard heat conduction model discussed later, the target heat conduction model in CFAST couples the solid to the gas phase using the relation

\be q''=-k\frac{dT}{dx} \label{eq:Target_Fourier}\ee

where $q''$ is the heat flux striking the target (again due to all sources).  This equation is the statment that the flux striking the target must be consistent with the temperature gradient at the surface.

Equation (\ref{eq:Target_PDE}) may be solved implicitly or explicitly.  When solved implicitly, the target temperature is added to the set of solution variables and equation (\ref{eq:Target_Fourier}) (not equation (\ref{eq:Target_PDE}) is added to the equation set solved by DASSL.
When solved explicitly, equation (\ref{eq:Target_PDE}) is solved as a stand-alone equation advancing the temperature profile in time.

{\bf Computing Transient Cable Target Heat Transfer:}

\newcommand{\Dt}{\Delta t}
\newcommand{\Dr}{\Delta r}
\newcommand{\Tipo}{T_{i+1}^{n+1}}
\newcommand{\Ti}{T_{i}^{n+1}}
\newcommand{\Timo}{T_{i-1}^{n+1}}

This section describes a CFAST implementation of a model for
predicting electrical cable failure first proposed
by Andersson and Van Hees Ref.~\cite{Andersson:2005}
and later implemented by McGrattan
in FDS~\cite{CAROLFIRE} .  This model uses a simple one-dimensional heat
transfer calculation, under the assumption that the cable can
be treated as a homogenous cylinder~\cite{Andersson:2005}.

The heat flux used to generate the heat transfer in the cable is provided
by CFAST which models the thermal environment of the compartment where
the cable is located.  In most realistic fire  scenarios, the heat flux
to the cable is not axially-symmetric.  CFAST therefore uses the maximum
heat flux value when modeling cable failure.

1D heat transfer may be computed within a cylindrically symmetric target
by splitting it into $N$ concentric control volumes and performing
an energy balance on each.  The energy balance for the $i$'th control
volume for $i=1\cdots N-1$, is

\begin{equation}
c\rho V_i\Delta T_i=(\dot{q}_i-\dot{q}_{i-1})\Dt
\label{eq:cylheat1}
\end{equation}

\noindent where $c\rho V_i\Delta T_i$ represents the change in internal
energy
and $(\dot{q}_i-\dot{q}_{i-1})\Dt$ represents the net heat flow across the
control volume's inner and outer boundary surface over a $\Dt$ time period.
The energy balance for the outermost or $N$'th control volume is similar
\begin{equation}
c\rho V_N\Delta T_N=(\dot{q}_{ext}''A_N-\dot{q}_{N-1})\Dt
\label{eq:cylheat2}
\end{equation}
with $\dot{q}_{ext}''A_N$ used to specify a boundary condition,
the combined net radiative and convective heat flux incident
on the the cylindrical target's outer surface.

\begin{figure}[h]
\begin{center}
\includegraphics[width=5.0in]{FIGURES/Theory/cylheat}\\
\end{center}
\caption{Schematic of a control volume for heat transfer in a cylindrical object.}
 \label{fig:cylheat}
\end{figure}
As illustrated in Figure \ref{fig:cylheat}, the $i$'th control volume has
volume $V_i$, temperature $T_i$ and heat flow at the inner
and outer boundaries of $\dot{q}_{i-1}$ and $\dot{q}_i$ respectively.
The $i$'th control volume has length $L$ and inner and outer radius
$r_{i-1}$ and $r_i$ where $r_i=i\Dr$ and $\Dr=R/N$.
The density and specific heat of material in all control volumes is
$c$ and $\rho$.

The right hand sides of equations \ref{eq:cylheat1} and (\ref{eq:cylheat2}) may be expressed in terms of temperature using
Fourier's law and noting that $\dot{q}_i=\dot{q}''_iA_i$ to obtain
\begin{equation}
c\rho V_i\Delta T_i=(\dot{q}''_iA_i-\dot{q}''_{i-1}A_{i-1})\Dt
=
\left[
k\left(\frac{T_{i+1}-T_i}{\Dr}\right)A_i-k\left(\frac{T_{i}-T_{i-1}}{\Dr}\right)A_{i-1}
\right]
\label{eq:cylheat2a}
\end{equation}

\noindent The volume of the $i$'th control volume is given by
\begin{eqnarray*}
V_i&=&\pi(r_i^2-r_{i-1}^2)L
=\pi\Dr(r_i+r_{i-1})L
=2\pi\Dr(i-0.5) L
\label{eq:cylheat3}
\end{eqnarray*}

\noindent The area, $A_i$, of the outer boundary surface of the $i$'th
control volume is given by
\begin{eqnarray*}
A_i=2\pi r_iL
=2\pi \Dr iL
\label{eq:cylheat4}
\end{eqnarray*}

\noindent Using the ratio $A_i/V_i=i/(\Dr(i-0.5))$
and $\Delta T_i=\Ti-T_i^n$,
equation (\ref{eq:cylheat2a}) may be simplified to
\begin{eqnarray}
\Ti-T_i^n&=&\frac{\Dt}{\rho c}
\left[
\left(\frac{\Tipo-\Ti}{\Dr}\right)
\frac{A_i}{V_i}-
\left(\frac{\Ti-\Timo}{\Dr}\right)
\left(\frac{A_{i-1}}{V_i}\right)
\right]
\nonumber\\
&=&\frac{\Dt}{\Dr^2}\frac{k}{\rho c}
\left[
\left(\Tipo-\Ti\right)
\left(\frac{i}{i-0.5}\right)-
\left(\Ti-\Timo\right)
\left(\frac{i-1}{i-0.5}\right)
\right]\nonumber\\
&=&\frac{\alpha\Dt}{\Dr^2(i-0.5)}
\left[
\left(\Tipo-\Ti\right)
i-
\left(\Ti-\Timo\right)
(i-1)
\right]
\label{eq:cylheat6}
\end{eqnarray}

\noindent where $\alpha-k/(\rho c)$.  Defining $C_i$ and $D_i$ as
\begin{eqnarray*}
C_i&=&\frac{\alpha\Dt}{\Dr^2}\left(\frac{i-1}{i-0.5}\right)\\
D_i&=&\frac{\alpha\Dt}{\Dr^2}\left(\frac{i}{i-0.5}\right),\\
\end{eqnarray*}
\noindent noting that $C_i+D_i=2\frac{\alpha\Dt}{\Dr^2}$
for $i=1\cdots N-1$ and
substituting into (\ref{eq:cylheat6}) results in

\begin{equation}
-C_i\Timo+(1+2\frac{\alpha\Dt}{\Dr^2})\Ti-D_i\Tipo=T_i^n
\label{eq:cylheat8}
\end{equation}

\noindent The energy balance for the $N$'th (outermost) control volume may be obtained by substituting $\Dr q''_{ext}/k$ for $\Tipo-\Ti$ in equation (\ref{eq:cylheat6}) to obtain
\begin{eqnarray*}
T_N^{n+1}-T_N^n=\frac{\alpha\Dt}{\Dr^2(N-0.5)}
\left[\frac{\Dr\dot{q}''_{ext}}{k}N-(T_N^{n+1}-T_{N-1}^{n+1})(N-1)\right]
\end{eqnarray*}

\noindent then substituting $C_N$ and $D_N$ and simplifying
to obtain

\begin{equation}
-C_NT_{N-1}^{n+1}+(1+C_N)T_N^{n+1}=T_N^n+D_N\frac{\Dr}{k}q''_{ext}
\label{eq:cylheat10}
\end{equation}

Equations (\ref{eq:cylheat8}) and (\ref{eq:cylheat10}) represent a  tri-diagonal linear system of equations which when written in matrix form
are given by
\begin{equation}
\left(
\begin{array}{ccccccccc}
  1+D_1 & -D_1 &  &  &  &  &  &  &  \\
   &  &  &  &  &  &  &  &  \\
   &  &\ddots  &  &  &  &  &  &  \\
   &  &  &  &  &  &  &  &  \\
   &  &  & -C_i &1+2\frac{\alpha\Dt}{\Dr^2}  &D_i  &  &  &  \\
   &  &  &  &  &  &  &  &  \\
   &  &  &  &  & \ddots &  &  &  \\
   &  &  &  &  &  &  &  &  \\
   &  &  &  &  &  &-C_N  &1+C_N  &
\end{array}
\right)
\left(
  \begin{array}{c}
    T_1^{n+1}\\
    T_2^{n+1}\\
    \vdots\\
    T_{i-1}^{n+1} \\
    T_i^{n+1} \\
    T_{i+1}^{n+1}\\
    \vdots\\
    T_{N-1}^{n+1}\\
    T_N^{n+1}
  \end{array}
\right)
=
\left(
  \begin{array}{c}
    T_1^{n}\\
    T_2^{n}\\
    \vdots\\
    T_{i-1}^{n} \\
    T_i^{n} \\
    T_{i+1}^{n}\\
    \vdots\\
    T_{N-1}^{n}\\
    T_N^{n}+D_N\frac{\Dr}{k}\dot{q}''_{ext}
  \end{array}
\right)
\label{eq:matrix}
\end{equation}

\noindent Equation (\ref{eq:matrix}) is then used to
advance the cable's temperature profile by $\Delta t$.

\section{Convection}

In general, convective heat transfer, $q\dprime$, across a surface of area
$A_S$, is defined as

\be q\dprime = h A_S \brackets{T_g - T_s} \ee

The convective heat transfer coefficient, $h$, is a function of the gas properties, temperature, and velocity. The Nusselt number is defined as $Nu_L = h L / k$, which for natural convection is related to the Rayleigh number, $Ra_L = g \beta \brackets{T_s - T_g}L^3 / \nu \alpha$ where $L$ is a characteristic length of the geometry, $g$ is the gravitational constant (m/s\superscript{2}), $k$ is the thermal conductivity (W/m\superscript{2} K), $\beta$ is a volumetric expansion coefficient (K\superscript{-1}), $T_s$ and $T_g$ are the temperatures of the surface and gas, respectively (K), $\nu$ is the kinematic viscosity (m\superscript{2}/s), and $\alpha$ is the thermal diffusivity (m\superscript{2}/s).  All properties are evaluated at the film temperature, $T_f = \brackets{T_s + T_g}/2$.  The typical correlations applicable to the problem at hand are available in the literature  \cite{Atreya:2003}. The table below gives the correlations used in CFAST.

\begin{table}[h!]
\begin{center}
\begin{tabular}{| p{1.5in} | c | c |}
\hline
Geometry & Correlation & Restrictions \\
 \hline
 Walls & {$\begin{array}{lll} Nu_L &=& \brackets{0.825 + \frac{0.387 Ra_L^{1/6}}{\brackets{1 + \brackets{0.492/Pr}^{9/16}}^{8/27}}}^2  \\ &=&0.12 \end{array}$}  & none \\
Ceilings and floors (hot surface up or cold surface down & {$Nu_L = 0.13 Ra_L^{1/3}$} & $2 x 10^8 \le Ra_L \le 10^{11}$ \\

 Ceilings and floors (cold surface up or hot surface down & {$Nu_L = 0.16 Ra_L^{1/3}$} & $2 x 10^8 \le Ra_L \le 10^{10}$ \\
 \hline
\end{tabular}
\end{center}
\end{table}

The Prandtl number, Pr, is the ratio of the kinematic viscosity and the thermal diffusivity. The thermal diffusivity, $\alpha$, and thermal conductivity, $k$, of air are defined as a function of the film temperature from data in reference \cite{Atreya:2003} as

\be \alpha = 1.0 x 10^{-9} T_f^{7/4} \ee

\be k = \frac{0.0209 + 2.33 x 10^{-5} T_f}{1 - 0.000267 T_f} \ee

 \section{Conduction}

 Procedures for solving 1-d heat conduction problems are well known, (e.g., backward difference (fully implicit), forward difference (fully explicit) or Crank-Nicolson \cite{Golub:1989}).  A finite difference approach  using a non-uniform spatial mesh is used to advance the wall temperature solution.  The heat equation is discretized using a second order central difference for the spatial derivative and a backward differences for the time derivative.  The resulting tri-diagonal system of equations is then solved to advance the temperature solution to time $t+\Delta t$.  This process is repeated , using the work of Moss and Forney \cite{Moss:1992},  until  the heat flux striking the wall (calculated from the convection and radiation algorithms) is consistent with the flux conducted into the wall calculated via Fourier's law

 \be q\dprime = -k \frac{dT}{dx} \ee
where $k$ is the thermal conductivity.  This solution strategy requires a differential algebraic equation (DAE) solver that can simultaneously solve both differential and algebraic equations.  With this method, only one or two extra equations are required per wall segment (two if both the interior and exterior wall segment surface temperatures are computed).  This solution strategy is more efficient than the method of lines since fewer equations need to be solved.  Wall segment temperature profiles, however, still have to be stored so there is no decrease in storage requirements.  Conduction is then coupled to the room conditions by temperatures supplied at the interior boundary by the differential equation solver.  The exterior boundary conditions are modeled as surfaces that exchange heat with an ambient surroundings for both convection and radiation.

A non-uniform mesh scheme was chosen to allow grid points in the calculation to cluster near the interior and exterior wall segment surfaces.  This is where the temperature gradients are the steepest.  A breakpoint  $x_b$ was defined by $x_b = MIN(x_p, W/2)$ where $W$ is the wall thickness and $x_p = 2 \sqrt{\alpha t_{final}} erfc^{-1} \brackets{.05}$ and $erfc^{-1}$ denotes the inverse of the complementary error function.  The value $x_p$ is the location in a semi-infinite wall where the temperature rise is 5~\% after $t_{final}$ seconds and is sometimes called the penetration depth.  Eighty percent of the grid points were placed on the interior side of $x_b$ and the remaining 20~\% were placed on the exterior side.

To illustrate the method, consider a one room case with one active wall.  There are four gas equations (pressure, upper layer volume, upper layer temperature, and lower layer temperature) and one wall temperature equation.  Implementation of the gradient matching method requires that storage be allocated for the temperature profile at the previous time, t, and at the next time, $t + \Delta t$.  Given the profile at time t and values for the five unknowns at time $t + \Delta t$ (initial guess by the solver), the temperature profile is advanced from time t to time $t + \Delta t$.  The temperature profile gradient at $x = 0$ is computed followed by the residuals for the five equations.  The DAE solver adjusts the solution variables and the time step until the residuals for all the equations are below an error tolerance.  Once the solver has completed the step, the array storing the temperature profile for the previous time is updated, and the DAE solver is ready to take its next step.

\subsection{Inter-compartment Heat Transfer}

Heat transfer between vertically connected compartments is modeled by merging the connected surfaces for the ceiling and floor compartments or for the connected horizontal compartments.  A heat conduction problem is solved for the merged walls using a temperature boundary condition for both the near and far wall.  As before, temperatures are determined by the DAE solver so that the heat flux striking the wall surface (both interior and exterior) is consistent with the temperature gradient at that surface.

For horizontal heat transfer between compartments, the connections may be between partial wall surfaces, expressed as a fraction of the wall surface. CFAST first estimates conduction fractions analogous to radiative configuration factors.    For example, if only one half of the rear wall in one compartment is adjacent to the front wall in a second compartment, the conduction fraction between the two compartments is 1/2.   Once these fractions are determined, an average flux, $q_{avg}\dprime$, is calculated using

\be q_{avg}\dprime = \sum_{walls} F_{ij} q_{wall_j}\dprime \ee
where $F_{ij}$ is the fraction of flux from wall $i$ that contributes to wall $j$, $q_{wall_j}\dprime$ is the flux striking wall $j$.

\subsection{Ceiling Jet}

Relatively early in the development of a fire, fire-driven ceiling jets and gas-to-ceiling convective heat transfer can play a significant role in room-to-room smoke spread and in the response of near-ceiling mounted detection hardware.  Cooper \cite{Cooper:1991} details a model and computer algorithm to predict the instantaneous rate of convective heat transfer from fire plume gases to the overhead ceiling surface in a room of fire origin.  The room is assumed to be a rectangular parallelepiped and, at times of interest, ceiling temperatures are simulated as being uniform.  Also presented is an estimate of the convective heat transfer due to ceiling-jet driven wall flows.  The effect on the heat transfer of the location of the fire within the room is taken into account.  This algorithm has been incorporated into the CFAST model.  In this section, we provide an overview of the model.  Complete details are available in reference [\cite{Cooper:1991}.

A schematic of a fire, fire plume, and ceiling jet is shown in Fig.~\ref{fig:CeilJet}. The buoyant fire plume rises from the height $Z_{fire}$ toward the ceiling.  When the fire is below the layer interface, its mass and enthalpy flow are assumed to be deposited into the upper layer at height $Z_{layer}$.  Having penetrated the interface, a portion of the plume typically continues to rise toward the ceiling.  As it impinges on the ceiling surface, the plume gases turn and form a relatively high temperature, high velocity, turbulent ceiling jet which flows radially outward along the ceiling and transfers heat to the relatively cool ceiling surface.  The convective heat transfer rate is a strong function of the radial distance from the point of impingement, reducing rapidly with increasing radius.  Eventually, the relatively high temperature ceiling jet is blocked by the relatively cool wall surfaces \cite{Cooper:1990a}.  The ceiling jet then turns downward and outward in a complicated flow along the vertical wall surfaces \cite{Cooper:1988, Jaluria:1989} .  The descent of the wall flows and the heat transfer from them are eventually stopped by upward buoyant forces.  They are then buoyed back upward and mix with the upper layer.

\begin{figure}
\begin{center}
\includegraphics[width=4.0in]{FIGURES/Theory/CeilJet}\\
\end{center}
\caption{Convective heat transfer to ceiling and wall surfaces via the ceiling jet.}
 \label{fig:CeilJet}
\end{figure}

The average convective heat transfer from the ceiling jet gases to the ceiling surface, $Q_{ceil}$, can be expressed in integral form as

\be Q_{ceil} = \int_0^{X_{wall}} \int_0^{Y_{wall}} q_{ceil}\dprime (x,y) dx dy \ee

The instantaneous convective heat flux, $q_{ceil}\dprime (x,y)$ can be determined as derived by Cooper \cite{Cooper:1991} as

\be q_{ceil}\dprime (x,y) = h \brackets{T_{ad} - T_{ceil}} \ee
where $T_{ad}$ is a characteristic ceiling jet temperature that would be measured adjacent to an adiabatic lower ceiling surface, and $h$ is a heat transfer coefficient.  $h$ and $T_{ad}$ are given by

\be \frac{h}{\tilde{h}} =
\left\{
\begin{array}{cc}
8.82 Re_H^{-1/2} Pr^{-2/3} \brackets{1 - \brackets{5 - 0.284 Re_H^{2/5}}\rH} & 0 \le \rH < 0.2 \\
\\
0.283 Re_H^{0.3} Pr^{-2/3} \brackets{\rH}^{-1.2} \frac{\rH - 0.0771}{\rH + 0.279} & 0.2 \le \rH
\end{array}
\right.
\ee

\be \frac{T_{ad} - T_u}{T_u {Q_H^*}^{2/3}} =
\left\{
\begin{array}{cc}
10.22 - 14.9 \rH & 0 \le \rH < 0.2 \\
\\
8.39 f\brackets{\rH} & 0.2 \le \rH
\end{array}
\right.
\ee
where

\be f\brackets{\rH} = \frac{1 - 1.10 \brackets{\rH}^{0.8}+ 0.808  \brackets{\rH}^{1.6}}
{1 - 1.10 \brackets{\rH}^{0.8}+ 2.20  \brackets{\rH}^{1.6} + 0.690 \brackets{\rH}^{2.4}}
\ee

\be r = \sqrt{\brackets{X-X_{fire}}^2 + \brackets{Y-Y_{fire}}^2} \ee
\be \tilde{h} = \rho_u c_p g^{1/2} H^{1/2} {Q_H^*}^{1/3} \; , \;
Re_H = \frac{g^{1/2} H^{3/2} {Q_H^*}^{1/3}}{\nu_u}  \; , \;
Q_H^* = \frac{Q}{\rho_u c_p T_u g^{1/2} H^{5/2} } \ee

\be Q =
\left\{
\begin{array}{cc}
Q_{f,C} \brackets{\frac{\sigma \dot{M}^*}{1 + \sigma}}& Z_{fire} < Z_{layer} < Z_{ceil} \\
\\
Q_{f,C} & {\begin{array}{c}
Z_{fire} \ge Z_{layer} \\
Z_{layer} = Z_{ceil}
\end{array} }
\end{array}
\right.
\; , \;
 \dot{M}^* =
\left\{
\begin{array}{cc}
0 & -1 < \sigma \le 0 < 0.2 \\
\\
\frac{1.04599 \sigma + 0.360391 \sigma^2}{1 + 1.37748 \sigma + 0.360391 \sigma^2} & \sigma > 0
\end{array}
\right.
\ee

\be \sigma = \frac{1 - \frac{T_u}{T_l}+C_T {Q_{EQ}^*}^{2/3}}{\frac{T_u}{t_l}}
\; , \;
C_T = 9.115
\; , \;
Q_{EQ}^* = \brackets{\frac{0.21 Q_{fc}}{c_p T_l \dot{m}_p}} \ee

In the above, $H$ is the distance from the (presumed) point source fire and the ceiling, $X_{fire}$  and $Y_{fire}$ are the position of the fire in the room, $Pr$ is the Prandtl number (taken to be 0.7) and $\nu$ is the kinematic viscosity of the upper layer gas which is assumed to have the properties of air and can
be estimated from $\nu = 0.04128 x 10^7 T_u^{5/2} /\brackets{T_u + 110.4}$. $Q_H^*$ and $Q_{EQ}^*$ are dimensionless numbers and are measures of the strength of the plume at the ceiling and the layer interface, respectively.

When the ceiling jet is blocked by the wall surfaces, the rate of heat transfer to the surface
increases.  Reference \cite{Cooper:1991} provides details of the calculation of wall surface area and convective heat flux for the wall surfaces.



\chapter{Fire Protection Devices}



\section{Heat Detectors}

Heat detection is modeled using temperatures obtained from the ceiling jet \cite{Cooper:1991}. Rooms without fires do not have ceiling jets. Sensors in these types of rooms use gas layer temperatures instead of ceiling jet temperatures. The characteristic detector temperature is simply the temperature of the ceiling jet (at the location of the detector). The characteristic heat detector temperature is modeled using the differential equation \cite{Heskestad:1976}

\be \frac{dT_L}{dt} = \frac{\sqrt{v(t)}}{RTI} \brackets{T_g(t) - T_L(t)} \; , \; T_L(0) = T_g(0)\label{eq:RTI} \ee
where $T_L$ and $T_g$ are the link and gas temperatures, $v$ is the gas velocity, and $RTI$ (response time index) is a measure of the sensor's sensitivity to temperature change (thermal inertia).  The heat detector differential eq (\ref{eq:RTI}) may be rewritten to

\be \frac{dT(t)}{dt} = a(t) - b(t)T(t) \; , \; T(t_0) = T_0 \label{eq:RTI_rewritten} \ee
where $a(t) = \frac{\sqrt{V(t)}T(t)}{RTI}$ and $b(t) = \frac{\sqrt{V(t)}}{RTI}$. Equation (\ref{eq:RTI_rewritten}) may be solved using the trapezoidal rule to obtain

\be \frac{T_{i+1} - T_i}{\Delta T} = \frac{1}{2} \brackets{\brackets{a_i - b_iT_i} + \brackets{a_{i+1} - b_{i+1}T_{i+1}}} \label{eq:RTI_trapezoidal} \ee
where the subscript $i$ denotes time at $t_i$. Equation (\ref{eq:RTI_trapezoidal}) may be simplified to

\be T_{i+1} = A_{i+1} - b_{i+1}T_{i+1} \label{eq:RTI_trapezoidal_simplified} \ee
where $A_{i+1} = T_i + \Dttwo \brackets{a_i - b_iT_i + a_{i+1}}$ and $B_{i+1} = \Dttwo b_{i+1}$ which has a solution

\be T_{i+1} = \frac{A_{i+1}}{1+B_{i+1}} = \frac{1 - \Dttwo b_i}{1 + \Dttwo b_{i+1}} T_i  + \frac{\Delta T}{1 + \Dttwo b_{i+1}}\brackets{\frac{a_i + a_{i+1}}{2}} \label{eq:RTI_solution1} \ee

Equation (\ref{eq:RTI_solution1}) reduces to the trapezoidal rule for integration when $b(t) = 0$. When $a(t)$ and $b(t)$ are constant (the gas temperature, $T_g$, and gas velocity, $V$ are not changing), eq (\ref{eq:RTI}) has the solution

\be T(t) = \frac{a}{b} + \frac{e^{-b(t-t_0)}\brackets{b T_0 - a}}{b} = T_g + e^{\frac{-\sqrt{V(t)}(t-t_0)}{RTI}}\brackets{T_0 - T_g}
 \ee

\section{Sprinkler Activation and Fire Attenuation} \label{sec:suppression}

For suppression, the sprinkler is modeled using a simple model \cite{Madrzykowski:1992} generalized for varying sprinkler spray densities \cite{Evans:1993}. It is then modeled by attenuating all fires in the room where the sensor activated by a term of the form $e^{-(t-t_{act})/t_{rate}}$ where $t_{act}$ is the time when the sensor activated and $t_{rate}$ is a constant determining how quickly the fire attenuates. The term $t_{rate}$ can be related to spray density of a sprinkler using a correlation developed in \cite{Evans:1993}. The suppression correlation was developed by modifying the heat release rate of a fire. For $t > t_{act}$ the heat release is given by

\be Q_f(t) = e^{-\brackets{t-t_{act}} / \brackets{3 Q_{spray}^{-1.8}}} Q_f(t_{act}) \ee
where $Q_{spray}$ is the spray density of a sprinkler. Note that decay rate can be formulated in terms of either the attenuation rate or the spray density. $t_{rate}$ can be expressed in terms of $Q_{spray}$ as $t_{rate} = 3Q_{spray}^{-1.8}$. A calculation is done to make sure that the fuel burned is consistent with the available oxygen. Once detection has occurred, then the mass and energy release rates are attenuated by

\be \dot{m}_f(t) = e^{-\brackets{t-t_{act}} / t_{rate}} \dot{m}_f(t_{act}) \ee
\be Q_f(t) = e^{-\brackets{t-t_{act}} / t_{rate}} Q_f(t_{act}) \ee

There are assumptions and limitations in this approach. Its main deficiency is that it assumes that sufficient water is applied to the fire to cause a decrease in the rate of heat release. This suppression model cannot handle the case when the fire overwhelms the sprinkler.  The suppression model as implemented does not include the effect of a second sprinkler. Detection of all sprinklers are noted but their activation does not make the fire go out any faster. Further, multiple fires in a room imply multiple ceiling jets. It is not clear how this should be handled, i.e.,how two ceiling jets should interact. When there is more than one fire, the detection algorithm uses the fire that results in the worst conditions (usually the closest fire) in order to calculate the fire sensor temperatures.  Finally, the ceiling jet algorithm that we use results in temperature predictions that are warmer (for a given heat release rate) than those used in the correlation developed by Madrzykowski \cite{Madrzykowski:1995}, which will cause activation sooner than expected.

\section{Species Concentration and Deposition}

CFAST uses a combustion chemistry scheme based on a carbon-hydrogen-oxygen balance.  The scheme is applied in three places.  The first is burning in the portion of the plume which is in the lower layer of the compartment of fire origin.  The second is the portion in the upper layer, also in the compartment of origin.  The third is in the vent flow which entrains air from a lower layer into an upper layer in an adjacent compartment.  Included in the combustion calculation is the generation and transport of a number of species that may be produced by a fire.  These species include unburned fuel, nitrogen, oxygen, carbon monoxide, carbon dioxide, hydrogen, carbon (assumed to be soot produced by the fire), hydrogen cyanide, hydrogen chloride, and an arbitrary trace species.

\subsection{Species Transport}

The species transport in CFAST is primarily a matter of bookkeeping to track individual species mass as it is generated by a fire, transported through vents, or mixed between layers in a compartment.  When the layers are initialized at the start of the simulation, they are set to ambient conditions.  These are the initial temperature prescribed by the user, and 23 \% by mass fraction (21 \% by volume fraction) oxygen, 77 \% by mass fraction (79 \% by volume fraction) nitrogen, a mass concentration of water prescribed by the user as a relative humidity, and a zero concentration of all other species.  As fuel is burned, the various species are produced in direct relation to the mass of fuel burned (this relation is the species yields prescribed by the user for the fuel burning).  Since oxygen is consumed rather than produced by the burning, the `yield' of oxygen is negative, and is set internally to correspond to the amount of oxygen used to burn the fuel (within the constraint of available oxygen limits discussed in sec. \ref{sec:Oxygen_Limit}). Two special separate species calculations are included in the model, a time-integrated value for a generic toxic species, Ct, and an arbitrary trace species, TS.  Both are assumed not to be part of the overall mass balance, but are rather generated by a fire and transported through a structure in a manner identical to other species.

Each unit mass of a species produced by a fire is carried in the flow to the various rooms and accumulates in the layers.  The model keeps track of the mass of each species in each layer, and knows the volume of each layer as a function of time.  The mass divided by the volume is the mass concentration, which along with the relative molecular mass gives the concentration in volume percent or parts per million as appropriate. Filters can be used in mechanical ventilation systems to remove species. The phenomenon has been implemented in CFAST to remove trace species and soot. It is implemented by modifying the source terms which describe gas flow. Mass that is filtered remains on the filter and is removed from the air stream. Both the resulting species density and total species removed can be analyzed. See reference \cite{Jones:2008} for an example on the use of filtering.

The calculation of radiation exchange in CFAST also depends in part on the species concentrations calculated by the model (and thus the user inputs for species yields). There are two separate radiation calculations done by CFAST. The first is for broadband radiation transfer for energy balance. The way this calculation is done is discussed in section \ref{sec:Radiation}. The second is a visible light calculation to answer the question of whether exit signs will be visible. The absorption of broadband radiation depends on the concentration of water, carbon dioxide and soot. The visibility calculation depends solely on the soot concentration For soot, the input for soot yield  assumes all the excess carbon goes to soot). This soot generation is then transported as a species to yield a soot mass concentration to use in the optical density calculation based originally on the work of Seader and Einhorn \cite{Seader:1976}. The most recent work is by Mulholland and Croakin\cite{Mullholland:2000}. Based on their experimental measurements, the soot mass density is multiplied by 3,817 m\superscript{2}/kg (formerly 3,500 m\superscript{2}/kg) to obtain an optical density (in units of m\superscript{-1}) which is the value reported by the model.

\subsection{HCl Deposition}\label{HClDeposition}

Hydrogen chloride produced in a fire can produce a strong irritant reaction that can impair escape from the fire.  It has been shown \cite{Galloway:1989} that significant amounts of the substance may be removed by adsorption by surfaces which contact smoke.  In our model, HCl production is treated in a manner similar to other species.  However, an additional term is required to allow for deposition on, and subsequent absorption into, material surfaces.

The physical configuration that we are modeling is a gas layer adjacent to a surface (Fig.~\ref{fig:HCl_Deposition}).  The gas layer is at some temperature $T_g$ with a concomitant density of hydrogen chloride, $\rho_{HCl}$.  The mass transport coefficient is calculated based on the Reynolds analogy with mass and heat transfer; that is, hydrogen chloride is mass being moved convectively in the boundary layer, and some of it simply sticks to the wall surface rather than completing the journey during the convective roll-up associated with eddy diffusion in the boundary layer.  The boundary layer at the wall is then in equilibrium with the wall.  The latter is a statistical process and is determined by evaporation from the wall and stickiness of the wall for HCl molecules.  This latter is greatly influenced by the concentration of water in the gas, in the boundary layer and on the wall itself.

\begin{figure}[h]
\begin{center}
\includegraphics[width=5.0in]{FIGURES/Theory/HCl_Deposition}\\
\end{center}
\caption{Schematic of hydrogen chloride deposition region.}
 \label{fig:HCl_Deposition}
\end{figure}

The rate of addition of mass of hydrogen chloride to the gas layer is given by

\be \frac{d}{dt} m_{HCl} = source - k_c \brackets{\rho_{HCl} - \rho_{bl-HCl}} A_w \ee
where source is the production rate from the burning object plus flow from other compartments. For the wall concentration, the rate of addition is

\be \frac{d}{dt} d_{w-HCl} = k_c \brackets{\rho_{HCl} - \rho_{bl-HCl}} - k_s m_{w-HCl} \ee
where the concentration in the boundary layer, $\rho_{bl-HCl}$  is related to the wall surface concentration by the equilibrium constant $k_e$, by the relation $\rho_{bl-HCl} = d_{w-HCl} / k_e$. We never actually solve for the concentration in the boundary layer, but it is available, as is a boundary layer temperature if it were of interest.  The transfer coefficients are

\be k_c = \frac{\dot{q}}{\Delta T \rho_g c_p} \ee
\be k_e = \frac{b_1 e^{1500/T_w}}{1 + b_2 e^{1500/T_w} \rho_{HCl}} \brackets{1 + \frac{b_5 \brackets{\rho_{H_2O}}^{b_6}}{\brackets{\rho_{H_2O,sat} - \rho_{H_2O,g}}^{b_7}} } \ee
\be k_s = b_3 e^{-\brackets{\frac{b_4}{R T_w}}} \ee

The only values currently available  for these quantities are shown in table \ref{tab:HCL_Deposition} \cite{Galloway:1990}.  The ``$b$'' coefficients are parameters which are found by fitting experimental data to the above equations. These coefficients reproduce the adsorption and absorption of HCl reasonably well.  Note though that error bars for these coefficients have not been reported in the literature.

\begin{table}
\begin{center}
\caption{Transfer coefficients for HCl deposition}
\label{tab:HCL_Deposition}
\begin{tabular}{| c | c | c | c | c | c | c | c |}
\hline
\multirow{2}{*}{Surface} & $b_1$ & $b_2$ & $b_3$ & $b_4$ & $b_5$ & $b_6$ & $b_7$ \\
 & (m) & (m\superscript{3}/kg) & (s\superscript{-1}) & (J/g mol) & (m\superscript{3}/kg)\superscript{$b_7 - b_6$} & (note a) & (note b) \\
 \hline
 Painted Gypsum & 0.0063 & 191.8 & 0.0587 & 7476 & 193 & 1.021 & 0.431 \\ \hline
 PMMA & $9.6 x 10^{-5}$ & 0.0137 & 0.0205 & 7476 & 29 & 1.0 & 0.431 \\ \hline
 Ceiling Tile & $4.0 x 10^{-3}$ & 0.0548 & 0.123 & 7476 & 30\superscript{a} & 1.0 & 0.431 \\ \hline
 Cement Block & $1.8 x 10^{-2}$ & 5.48 & 0.497 & 7476 & 30\superscript{a} & 1.0 & 0.431 \\  \hline
 Calcium Silicate Board & $1.9 x 10^{-2}$ & 0.137 & 0.030 & 7476 & 30\superscript{a} & 1.0 & 0.431 \\  \hline
\end{tabular}
\end{center}
a - very approximate value, insufficient data for high confidence value

b - non-dimensional
\end{table}

The experimental basis for poly(methyl methacrylate) and gypsum cover a sufficiently wide range of conditions that they should be usable in a variety of practical situations.  The parameters for the other surfaces do not have much experimental backing, and so their use should be limited to comparison purposes.

\section{Single Zone Approximation}

A single zone approximation is appropriate for smoke flow far from a fire source where the two-zone layer stratification is less pronounced than in compartments near the fire. In this situation, a single zone approximation may be derived by using the normal two-zone source terms and the substitutions:

\be
\begin{array}{rcl}
\dot{m}_U^{new} &=& \dot{m}_L + \dot{m}_U \\
\dot{m}_L^{new} &=& 0 \\
Q_U^{new} &=& Q_L + Q_U \\
Q_L^{new} &=& 0
\end{array}
\ee

This is used in situations where the stratification does not occur. Examples are elevators shafts, complex stairwells, natural venting ductwork, and compartments far from the fire.

