\documentclass[12pt]{book}
\usepackage{mathptm,times,color}
\usepackage[pdftex]{graphicx}
\usepackage{hyperref}
\usepackage{multirow}
\usepackage{rotating}
\usepackage{longtable}
\usepackage{amsmath}
\usepackage{xfrac}

\usepackage{datetime}
\newdateformat{mydate}{\monthname[\THEMONTH] \THEYEAR}

\usepackage{listings}
\usepackage{textcomp}
\definecolor{lbcolor}{rgb}{0.96,0.96,0.96}
\lstset{
    %backgroundcolor=\color{lbcolor},
    tabsize=4,
    rulecolor=,
    language=Fortran,
        basicstyle=\footnotesize\ttfamily,
        upquote=true,
        aboveskip={\baselineskip},
        belowskip={\baselineskip},
        columns=fixed,
        extendedchars=true,
        breaklines=true,
        breakatwhitespace=true,
        frame=none,
        showtabs=false,
        showspaces=false,
        showstringspaces=false,
        identifierstyle=\ttfamily,
        keywordstyle=\color[rgb]{0,0,0},
        commentstyle=\color[rgb]{0,0,0},
        stringstyle=\color[rgb]{0,0,0},
}

\usepackage{wrapfig}

\renewcommand{\bibname}{References}

\setlength{\textwidth}{6.5in}
\setlength{\textheight}{9.0in}
\setlength{\topmargin}{0.in}
\setlength{\headheight}{0.in}
\setlength{\headsep}{0.in}
\setlength{\parindent}{0.25in}
\setlength{\oddsidemargin}{0.0in}
\setlength{\evensidemargin}{0.0in}

\begin{document}

\bibliographystyle{unsrt}

\newcommand{\asqh}{$A_T/A\sqrt{h}$}
\newcommand{\degc}{$^{\circ}$C }
\newcommand{\degf}{$^{\circ}$F }

\newcommand{\brackets}[1]{ { \left( {#1} \right) } }

\newcommand{\superscript}[1]{\ensuremath{^{\textnormal{#1}}}}
\newcommand{\subscript}[1]{\ensuremath{_{\textnormal{#1}}}}

\newcommand{\dx}{\delta x}
\newcommand{\dy}{\delta y}
\newcommand{\dz}{\delta z}
\newcommand{\dt}{\delta t}
\newcommand{\dQ}{\dot{Q}}
\newcommand{\dm}{\dot{m}}

\newcommand{\ha}{\frac{1}{2}}
\newcommand{\ft}{\frac{4}{3}}
\newcommand{\ot}{\frac{1}{3}}
\newcommand{\fofi}{\frac{4}{5}}
\newcommand{\of}{\frac{1}{4}}
\newcommand{\twth}{\frac{2}{3}}

\newcommand{\be}{\begin{equation}}
\newcommand{\ee}{\end{equation}}

\newcommand{\RE}{\hbox{Re}}
\newcommand{\PR}{\hbox{Pr}}
\newcommand{\NU}{\hbox{Nu}}

\newcommand{\COTWO}{{\tiny \hbox{CO}_2}}
\newcommand{\OTWO}{{\tiny \hbox{O}_2}}
\newcommand{\CO}{{\tiny \hbox{CO}}}
\newcommand{\HTWOO}{{\tiny \hbox{H}_2\hbox{O}}}
\newcommand{\NTWO}{{\tiny \hbox{N}_2}}
\newcommand{\F}{{\tiny \hbox{F}}}
\newcommand{\So}{{\tiny \hbox{S}}}
\newcommand{\M}{{\tiny \hbox{M}}}
\newcommand{\HCN}{{\tiny \hbox{HCN}}}
\newcommand{\HCl}{{\tiny \hbox{HCl}}}
\newcommand{\Hy}{{\tiny \hbox{H}}}
\newcommand{\C}{{\tiny \hbox{C}}}
\newcommand{\N}{{\tiny \hbox{N}}}
\newcommand{\Oh}{{\tiny \hbox{O}}}
\newcommand{\Cl}{{\tiny \hbox{Cl}}}
\newcommand{\rb}[1]{\raisebox{1.5ex}[0pt]{#1}}

\pagestyle{empty}

\begin{minipage}[t][9in][s]{6.5in}

\flushright{
    \fontsize{20}{24}\selectfont
    \bf{NIST Technical Note XXXX\\}
     }

\vspace{1in}

\flushright{
  \fontsize{28}{33.6}\selectfont
   \bf{CFAST -- Consolidated Model of Fire \\
   Growth and Smoke Transport \\
 \vspace{.25in}
   Volume 2: User's Guide \\}
   }
   
\vspace{.5in}

\flushright{
  \fontsize{14}{16.8}\selectfont
   Richard D. Peacock \\
   Paul A. Reneke \\
   Glenn P. Forney \\
}

\vspace{.5in}

\normalsize
\flushright{http://dx.doi.org/10.6028/NIST.TN.xxxxv2}

\vfill

\flushright{\includegraphics[height=1.05in]{FIGURES/nistident}}

\end{minipage}

\newpage

\hspace{5in}

\newpage

\begin{minipage}[t][9in][s]{6.5in}

\flushright{
    \fontsize{20}{24}\selectfont
    \bf{NIST Technical Note XXXX\\
}
     }

\vspace{1.in}

\flushright{
  \fontsize{28}{33.6}\selectfont
   \bf{CFAST -- Consolidated Model of Fire \\
   Growth and Smoke Transport \\
\vspace{.25in}
   Volume 2: User's Guide \\}
   }

\vspace{.5in}

\normalsize
\flushright{
Richard D. Peacock \\
Paul A. Reneke \\
Glenn P. Forney \\
{\em Fire Research Division} \\
{\em Engineering Laboratory} 
 }
 
 \vspace{.5in}
 
\flushright{http://dx.doi.org/10.6028/NIST.SP.xxxxv2}

\vspace{.25in}

\flushright{\mydate\today\\
$SVN Repository$~$Revision: 831$}


\vfill

\flushright{\includegraphics[width=1in]{FIGURES/doc} }

\small
\flushright{U.S. Department of Commerce \\
{\em Rebecca Blank, Acting Secretary} \\
\hspace{1in} \\
National Institute of Standards and Technology \\
{\em Patrick D. Gallagher, Under Secretary of Commerce for Standards and Technology and Director} }

\end{minipage}




\newpage

\frontmatter

\pagestyle{plain}
\setcounter{page}{3}

\chapter{Disclaimer}

The U. S. Department of Commerce makes no warranty, expressed or implied, to users of 
CFAST and associated computer programs, and accepts no responsibility for its use.  Users of 
CFAST assume sole responsibility under Federal law for determining the appropriateness of its 
use in any particular application; for any conclusions drawn from the results of its use; and for 
any actions taken or not taken as a result of analyses performed using these tools. 
CFAST is intended for use only by those competent in the field of fire safety and is intended 
only to supplement the informed judgment of a qualified user. The software package is a 
computer model which may or may not have predictive value when applied to a specific set of 
factual circumstances. Lack of accurate predictions by the model could lead to erroneous 
conclusions with regard to fire safety. All results should be evaluated by an informed user.

\chapter{Intent and Use}

The algorithms, procedures, and computer programs described in this report constitute a 
methodology for predicting some of the consequences resulting from a prescribed fire.  They 
have been compiled from the best knowledge and understanding currently available, but have 
important limitations that must be understood and considered by the user.  The program is 
intended for use by persons competent in the field of fire safety and with some familiarity with 
personal computers. It is intended as an aid in the fire safety decision-making process.

\chapter{Abstract}

CFAST is a two-zone fire model capable of predicting the environment in a multi-compartment structure subjected to a fire. It calculates the time evolving distribution of smoke and gaseous combustion products as well as the temperature throughout a building during a user-prescribed fire. This report describes the use of the model, including installing and running the software, the computer platforms upon which it is supported and examples to verify correct installation.

\chapter{Acknowledgments}

\label{acksection}

Continuing support for CFAST is via internal funding at NIST. In addition, support is provided by other agencies of the U.S. Federal Government, most notably the Nuclear Regulatory Commission Office of Research and the U.S. Department of Energy. The U.S. NRC Office of Research has funded key validation experiments, the preparation of the CFAST manuals, and the continuing development of sub-models that are of importance in the area of nuclear power plant safety. Special thanks to Mark Salley and David Stroup for their efforts and support. 

Support to refine the software development and quality assurance process for CFAST has been provided by the U.S. Department of Energy (DOE). The assistance of Subir Sen and Debra Sparkman in understanding DOE software quality assurance programs and the application of the process to CFAST is gratefully acknowledged. Thanks are also due to Allan Coutts, Washington Safety Management Solutions for his insight into the application of fire models to nuclear safety applications and detailed review of the CFAST document updates for DOE.

\tableofcontents

\listoffigures

\listoftables

\mainmatter


\chapter{Background}

CFAST is a two-zone fire model used to calculate the evolving distribution of smoke, fire gases and temperature throughout compartments of a building during a fire. These can range from very small containment vessels, on the order of 1 m3 to large spaces on the order of 1000 m3.  This guide describes how to obtain the model, verify its correct installation, create input data in an appropriate form, and analyze of the output of a simulation. 

The modeling equations used in CFAST take the mathematical form of an initial value problem for a system of ordinary differential equations (ODEs).  These equations are derived using the conservation of mass, the conservation of energy (equivalently the first law of thermodynamics), the ideal gas law and relations for density and internal energy.  These equations predict as functions of time quantities such as pressure, layer height and temperatures given the accumulation of mass and enthalpy in the two layers.  The CFAST model then consists of a set of ODEs to compute the environment in each compartment and a collection of algorithms to compute the mass and enthalpy source terms required by the ODEs.  The formulation of the equations, their solution, and discussion of validation and verification of the code are presented in a companion document \cite{CFAST_Tech_Guide_6}.

All of the data to run the model is contained in a primary data file, together with databases for objects, thermophysical properties of boundaries, and sample prescribed fire descriptions \cite{Gross:1985}. These files contain information about the building geometry (compartment sizes, materials of construction, and material properties), connections between compartments (horizontal flow openings such as doors, windows, vertical flow openings in floors and ceilings, and mechanical ventilation connections), fire properties (fire size and species production rates as a function of time), and specifications for detectors, sprinklers, and targets (position, size, heat transfer characteristics, and flow characteristics for sprinklers). Materials are defined by their thermal conductivity, specific heat, density, thickness, and burning behavior. Throughout the discussion on the model inputs, notes are included to provide additional insight on the model�s operation.

The outputs of CFAST are the sensible variables that are needed for assessing the environment in a building subjected to a fire. These include temperatures of the upper and lower gas layers within each compartment, the ceiling/wall/floor temperatures within each compartment, the visible smoke and gas species concentrations within each layer, target temperatures and sprinkler activation time. 

Many of the outputs from the CFAST model are relatively insensitive to uncertainty in the inputs for a broad range of scenarios. However, the more precisely the scenario is defined, the more accurate the results will be. Not surprisingly, the heat release rate is the most important variable, because it provides the driving force for fire-driven flows. Other variables related to compartment geometry such as compartment height or vent sizes, while important for the model results, are typically more easily defined for specific design scenarios than fire related inputs. 

The first public release of CFAST was version 1.0 in June of 1990. This version was restructured from FAST \cite{Models:FAST} to incorporate the "lessons learned" from the zone model CCFM \cite{Models:CCFM}, namely that modifications and additions to the model are easier and more robust if the components such as the physical routines are separated from the solver code used by the model. Version 2 was released as a component of Hazard 1.2 in 1994 \cite{Models:HAZARDI}. The first of the 3.x series was released in 1995 and included a vertical flame spread algorithm, ceiling jets and non-uniform heat loss to the ceiling, spot targets, and heating and burning of multiple objects in addition to multiple prescribed fires. Ignition was assigned based on a critical heat flux, a critical temperature, or a critical time input by the user. As CFAST evolved over the next five years, version 3 included smoke and heat detectors, suppression through heat release reduction, better characterization of flow through doors and windows, vertical heat conduction through ceiling/floor boundaries, and non-rectangular compartments. In 2000, version 4 was released and included horizontal heat conduction through walls, and horizontal smoke flow in corridors. Version 5 improved the combustion chemistry.  Version 6 included a new user interface written for Windows and revisions to the input file and model. The current version is 6.3


\chapter{Getting Started}

\section{Installation from the CFAST Web Site}

CFAST is documented by three publications, this user�s guide, a technical reference guide \cite{CFAST_Tech_Guide_6} and a model evaluation guide \cite{CFAST_Valid_Guide_6}. The technical reference guide describes the underlying physical principles, provides a comparison with other models, and includes an evaluation of the model following the guidelines of ASTM E1355 \cite{ASTM:E1355}. The model evaluation guide documents verification and validation efforts for the model. This user�s guide describes how to use the model and applies to version 6 and later.  All the documentation is available on the web site.

All of the files associated with CFAST can be obtained at:

\begin{lstlisting}
http://cfast.nist.gov
\end{lstlisting}

\begin{wrapfigure}{r}{4.35in}
  \includegraphics[width=4.25in]{FIGURES/Getting_Started/Install}
\end{wrapfigure}

Information about new versions, bug fixes, and documentation for the model and software are available on this web site.  The CFAST distribution consists of a self-extracting set-up program for Windows-based PCs. 

After downloading the set-up program to a PC, double-clicking on the file�s icon walks the user through a series of steps for installation of the program.  The most important part of the installation is the creation of a folder (typically c:$\backslash$Program Files$\backslash$CFAST by default) in which the CFAST executable files and supplemental data files are installed.  Sample input files are placed in a folder called CFASTData off the user's �My Documents� folder.

\section{Computer Hardware Requirements}

CFAST requires a relatively fast processor and a sufficient amount of random-access memory (RAM) for complex cases. Typical calculation times for a multi-compartment scenario can range from a few seconds to multiple hours, depending on the details of the scenario. Plus, hard drive storage is needed to store the output of calculations.  It is not uncommon for a single calculation to generate output files as large as several hundred megabytes.

\section{Verifying Correct Installation and Operation}

Sample input files are provided with the program for new users who are encouraged to first run the sample calculations before attempting to create an input file. To run the model, browse to the location of the CFAST input files (default location is in a folder called CFASTData$\backslash$Samples off the user�s �My Documents� folder, and double click on the file named standard.in.  This should open the file in the CFAST input editor, CEdit.  The simple test case can be run from the program menu by selecting �Run!� and then �Model Simulation, CFAST�
This runs a very simple test case and it should be completed quickly. Additional details on running CFAST are included in the next chapter. To verify that the installation has been done correctly, the output of the model should appear as follows.

\begin{figure}[h!]
\begin{center}
\includegraphics[width=6.5in]{FIGURES/Getting_Started/Standard_Output}
\end{center}
\end{figure}

This case checks several attributes of the installation including the presence of the database files (see auxiliary files in the section on building input files). Additional explanation of the results of this run is described in Chapter 5.

A detailed process to insure that the model is installed correctly is given in Appendix B of this CFAST User�s Guide.


\chapter{Running CFAST}


The CFAST distribution includes a Windows-based input editor called CEdit that allows the user to enter details of a simulation in a standard Windows format, save the data file to disk, and run the simulation. It is suggested that a new user start with an existing data file, run it as is, and then make the appropriate changes to the input file for the desired scenario. 

A description of all the input keywords can be found in Appendix~\ref{sec:CFAST_Keywords}.

\section{The Feature Tabs}

The input parameters are organized via tabs near the top of the CEdit screen, as shown in Fig.~\ref{primary_screen}.

\begin{figure}[h!]
\includegraphics[width=6.5in]{FIGURES/Running_CFAST/Environment_Tab}
\caption[The primary CFAST input page]{The primary CFAST input page.}
\label{primary_screen}
\end{figure}

\begin{description}
\item[Simulation Environment] includes simulation time, specification of model outputs, and ambient conditions. Also included on the page are a constantly updated list of errors, warnings, and messages about the input file specification or model simulation.
\item[Compartments] defines the size, construction characteristics, and position of the compartments in a simulation.
\item[Wall Vents, Ceiling/Floor Vents, and Mechanical Ventilation] allows the user to connect compartments with doors and windows, ceiling and floor vents, or forced air ventilation systems.
\item[Fires] include user specification of the initial fire source and any additional burning objects in one or more of the compartments of the simulation.
\item[Detection / Suppression] defines any heat alarms and sprinklers in the compartments of the simulation.
\item[Targets] provide the ability calculate the temperature and net heat flux to objects placed and oriented arbitrarily in the structure.
\item[Surface Connections] allows for more detailed description of the connections between compartments in the simulation to better simulate the transfer of heat from compartment to compartment in the simulation.
\item[Visualizations] allows specification of one or more 2-D and 3-D visualizations to be added to the simulation for viewing with Smokeview. Note that these can require significant additional computational time than a basic CFAST simulation without visualizations.
\end{description}


\subsection{The Run! Menu}

The program includes a number of menu items for ancillary functions.  In addition to the normal file menu items to open and save input data files or to exit the program, a 'Run!' menu is included to execute or view the current simulation. Menu items include the following:

\textbf{Create Geometry File:} used to create a geometry file for visualization with the program smokeview.  The input data file is saved, if necessary, and CFAST is run with an option to only run through initialization. This is particularly useful to review placement of compartments, vents, and fires in a CFAST scenario. The resulting geometry can be viewed with the 'Simulation Visualization, Smokeview' menu item, below.

\textbf{Model Simulation, CFAST:} runs the current input data file specification with the fire model CFAST.  The input data file is saved, if necessary, and CFAST is run to completion.  Additional details are described below in the section on starting a CFAST calculation. In order to visualize the results of the simulation with the program smokeview, the Smokeview Output Interval must be set to a non-zero value on the simulation Environment page.  This is described in more detail in chapter 4.

\textbf{Simulation Visualization, Smokeview:} runs the program smokeview with the previously defined smokeview geometry file. This allows the user to see the compartment geometry and connections or view the results of the simulation visually.  Additional details on the use of smokeview are included in the user's guide for Smokeview \cite{Smokeview_Users_Guide_6}.

\begin{figure}[h!]
\begin{center}
\includegraphics[width=6.5in]{FIGURES/Running_CFAST/Smokeview_Sample}
\end{center}
\end{figure}

\subsubsection{Output Options}

Additional menu items can be selected to control details of the model simulation output using CFAST by selecting the Output Options submenu.
\begin{description}
\item[Detailed Output File] If checked, this menu item directs the CFAST model to produce a detailed text output file.  Details of the output are included in chapter 5.
\item[Total Mass Output File] If checked, this menu item directs the CFAST model to replace the flow output with total mass flow through (mechanical) vents rather than the default flow rate values.
\item[Net Heat Flux]  If checked, this menu item directs the CFAST model to calculate heat flux to targets as a net heat flux to the target with the target at ambient temperature rather than at the calculated temperature.  This output is particularly useful to compare predictions to heat flux measurement using water cooled heat flux gauges.
\item[Show CFAST Window] If checked, this menu item allows the user to see the windows command prompt that is used to execute the CFAST model when the Model Simulation, CFAST menu item is used.  By default, this is not checked.  Normally, this can be left unchecked.  For troubleshooting, this can be selected to see additional details of the calculation as it progresses.
\item[CFAST Validation Output]If checked, this menu item directs the CFAST model to output target fluxes at net heat flux and to output an abbreviated heading for spreadsheet columns that are better for automated processing of the data.
\end{description}


\subsection{The Edit Menu}


The 'Edit' menu allows the user to view or change the material thermophysical properties and fire definitions used by the model and to select desired engineering units used in the input editor CEdit. Menu items include the following.

\subsubsection{Thermal Properties}

Heat transfer through compartment surfaces, to secondary fire objects, or other targets that may be specified depends on user-specified thermal properties for the materials.  These may be viewed, changed, or added to by the user as desired with the \textbf{Edit Thermal Properties} menu item. Materials properties include thermal conductivity, specific heat, density, thickness, and emissivity. Materials included in the database provided with the program are textbook values of common building and furnishing materials.

When CEdit is started, no thermal properties or fires are defined \label{Thermal_Properties_Menu}.  The \textbf{Insert Thermal Properties} menu item allows thermal properties used in a different simulation to be inserted into the data for the current simulation by choosing an existing CFAST input file and selecting one or more of the thermal properties to be included in the current simulation.

\subsubsection{Fires}

Fires in CFAST are defined with one or more selected fire objects that define the heat release rate, pyrolysis rate, and species yields as a function of time for each fire.  These may include the default set of fire objects included with the software or additional or modified objects created by the model user. They may be viewed or changed by the user as desired with the \textbf{Edit Fires} menu.

When CEdit is started, no thermal properties or fires are defined.  The \textbf{Insert Fires} menu item allows fire definitions used in a different simulation to be inserted into the data for the current simulation by choosing an existing CFAST input file and selecting one or more of the fires to be included in the current simulation.





\subsubsection{Select Engineering Units} 

The CFAST model uses input values and provides output in S.I. units. Within the input editor, CEdit, the user may select engineering units of choice for input and output.  These values are saved in the windows registry and may be changed at any time. By default, most outputs are in S.I. units, with temperature in Celsius.


\subsection{View}

The View menu allows the user to view and / or print the input data file, output file (if the simulation has been run and a text output file generated) and the log file of the simulation. If one of the items does not exist on the user's hard disk, the selection is grayed out. \\~ \\

\subsection{Help}

The Help menu accesses this user's guide, the CFAST web site, or an about dialog box that displays the user license and version of the program.



\section{File Naming and Location}

By default, the CFAST installation places all program files in the directory 'c:$\backslash$Program Files$\backslash$CFAST6' (or directory 'c:$\backslash$Program Files (x86)$\backslash$CFAST6'  on Windows 7 machines) and sample input data files in the 'Examples' folder included in the installation directory, While these locations can be changed during installation, the documentation in this user's guide assumes these locations.

In addition, there are several files that CFAST uses to communicate with its environment.  They include 1) an input data file, required for every simulation, 2) a series of spreadsheet files of important output variables, and binary data files containing calculated values for visualization the simulation.  Documentation of the input data file is included as chapter 4 of this user's guide.

In CFAST, simulations are arranged as projects with all the files associated with a single simulation sharing a common base file name.  For a simulation with a base file name of 'project', the built-in naming conventions would identify the files of the simulation as follows:

\begin{itemize}
\item input: project.in
\item text output file: project.out
\item spreadsheet output files: (Normal output) project\_n.csv, (Species output) project\_s.csv, (Flow output) project\_f.csv, (Wall surface temperatures, targets and sprinklers) project\_w.csv
\item smokeview geometry file: project.smv
\item smokeview plot file: project.zone
\item smokeview slice file(s): project\_\#\#\#\#.sf (where \#\#\#\# is a four digit number automatically assigned by the software)
\item smokeview iso-surface file(s): project\_\#\#\#\#.iso (where \#\#\#\# is a four digit number automatically assigned by the software).
\end{itemize}

There may be additional files associated with a specific simulation.  Any fires defined for the simulation (fire object files are defined with a .o extension) and customized thermal properties may be included in a revised thermal.csv file.  These must be in the same directory as the input file for the model to run.

\section{Starting a CFAST Calculation}

\subsection{Running CFAST from CEdit}

Typically, model simulations are run directly from the input editor, CEdit.  To run the model, either open an existing input data file from the program menus with 'File', 'Open,' or create a new input data file within CEdit.  The model is run by selecting 'Run!' and then 'Model Simulation, CFAST.'

This opens a window that shows the progress of the simulation, with information on the environment in each compartment of the simulation.

\begin{figure}[h!]
\begin{center}
\includegraphics[width=6.5in]{FIGURES/Running_CFAST/Standard_Output}
\end{center}
\end{figure}

Normally, model outputs are displayed and updated only at any of the time intervals specified on the environment page. For complex calculations, there may be a significant time period between display updates. The update button allows the user to see the current state of the calculation at any time. The update button is only available when the simulation is in progress.


\subsection{Running CFAST from a Command Prompt}

The model CFAST can also be run from a Windows command prompt.  CFAST can be run from any folder, and refer to a data file in any other folder. The fires and thermophysical properties have to be in either the data folder, or the executable folder. The data folder is checked first and then the executable folder.

\begin{lstlisting}
[drive1:\][folder1\]cfast [drive2:\][folder2\]project
\end{lstlisting}

The project name will have extensions appended as needed (see below). For example, to run a test case when the CFAST executable is located in c:$\backslash$nist$\backslash$cfast6 and the input data file is located in c:$\backslash$data, the following command could be used:

\begin{lstlisting}
c:\nist\cfast6\cfast c:\data\testfire0   <<< note there is no extension.
\end{lstlisting}

If the command is entered as $\backslash$bin$\backslash$cfast $\backslash$bin$\backslash$data$\backslash$testfire0.in, then CFAST will try to open testfire0.in.in

The database files for thermal properties and fire objects may be located either in the folder with the input data file or in the folder with the CFAST executable. The model checks first in the data file folder and then in the CFAST executable folder.  If the files do not exist in either location, the simulation is not run. By default, names for these files are thermal.csv for the thermal properties file and *.o for the fire object files.

Command line options

\begin{itemize}
\item k - no keyboard access
\item i - initialization only
\item h - output header
\item c - compact output
\item f - full output (c and f are exclusive). Note the interaction of the f and c option. The default for the console output is /c. The default for the file output is /f. This default action can be overwritten by explicitly including the /f or /c option. Output goes to the screen if the print interval (second entry on the TIMES line) is positive and to the output file if the interval is negative.
\item t - replace the flow output with total flow through (mechanical) vents.
\item n - net heat flux option
\item v - validation output (outputs a modified set of spreadsheet files with different column headers designed to facilitate automated analysis of the output)
\end{itemize}

\chapter{Setting Up the Input File for CFAST}

\section{Overview}

CFAST is a computer program that uses an input file and generates one or more output files. The first step in performing a calculation is to generate a text input file that provides the program with all of the necessary information to describe the scenario under consideration.  The most important inputs determine the geometry of the compartments in the scenario and the connections between these compartments. Next, the fire, detection, and suppression characteristics of the scenario are defined. Finally, there are a number of parameters that customize the output from the model.  Each line of the file contains a keyword label that identifies the input, followed by a number of numerical or text inputs corresponding to the particular input keyword. A simple input file is shown below. This example is used in the discussion of the output in Chapter 5.

\begin{lstlisting}
VERSN,6,Cable tray fire -- base case
!!
!!Scenario Configuration Keywords
!!
TIMES,1800,-120,10,30
EAMB,293.15,101300,0
TAMB,293.15,101300,0,50
CJET,WALLS
WIND,0,10,0.16
!!
!!Material Properties
!!
MATL,CONCRETE,1.75,1000,2200,0.15,0.94,"Concrete, Normal Weight (6 in)"
MATL,GLASSFB3,0.036,795,105,0.013,0.9,"Glass Fiber, Organic Bonded (1/2 in)"
MATL,METHANE,0.07,1090,930,0.0127,0.04,"Methane, a transparent gas (CH4)"
MATL,HARDWOOD,0.16,1255,720,0.019,0.9,"Wood, Hardwoods (oak, maple) (3/4 in)"
!!
!!Compartment keywords
!!
COMPA,Compartment 1,9.1,5,4.6,0,0,0,GLASSFB3,CONCRETE,CONCRETE
!!
!!Vent keywords
!!
HVENT,1,2,1,1,2.4,0,1,0,0,1,1
!!
!!Fire keywords
!!
GLOBA,10,393.15
!!Wood_Wall
FIRE,1,4.55,2.5,0,1,1,0,0,0,1,Wood_Wall
CHEMI,1,4,0,0,0,0.33,1.81E+07,HARDWOOD
TIME,0,8000
HRR,0,1000000
SOOT,0.02,0.02
CO,0.02,0.02
TRACE,0,0
AREA,0.05,9
HEIGH,0,3
!!
!!Target and detector keywords
!!
TARGET,1,2.2,1.88,2.34,0,0,1,CONCRETE,IMPLICIT,PDE,0.5
\end{lstlisting}

All of the inputs to the model are discussed in this chapter.  Following the discussion that details each input, their engineering units and default values, notes are included that provided additional guidance or frequently addressed problems that may be encountered by the user. These notes take the form of a bulleted list such as:

\begin{itemize}
\item The inputs may be integers (e.g., a simulation time of 1800 s), real numbers (a mass loss rate of 0.0082 kg/s), or text (a floor material of CONCRETE), as appropriate. The input file is a comma-separated ASCII text file and may be edited with a spreadsheet program or any text editor. It is possible to use a word processor but it is important to save the file in ASCII text format and not in a word processing format. Note that some word processors will save punctuation and other characters incorrectly for the simple ASCII text file used by CFAST. It is recommended that the input files be created with the input editor, CEdit, provided as part of the CFAST distribution.  In addition to checking the input data for errors, it includes typical ranges for input values to assist in appropriate use of the model.

\item Numeric format for inputs to CFAST and the input editor CEdit assume a period is used to separate the integer and fractional parts of the number. No separator is used to group digits in the integer part of numbers.

\item Each line of input consists of a label followed by one or more alphanumeric parameters associated with that input label, separated by commas.  The label must always begin in the first space of the line and be in capital letters.  Following the label, the values may start in any column, and all values must be separated by a comma.  Values may contain decimal points if needed or desired.  They are not required.  

\item Inputs are in standard SI units.  The maximum line length is 1024 characters, so all data for each keyword must fit in this number of characters.
\end{itemize}

The installation program creates a shortcut to the input editor on the Windows start menu labeled �CFAST� that points to the input editor.  Once started, the user is presented with a series of tabbed-pages for the various inputs in a CFAST input data file.

\begin{figure}[h!]
\begin{center}
\includegraphics[width=6.5in]{FIGURES/Input_File/Tabs}
\end{center}
\end{figure}

These tabbed-pages organize the inputs for CFAST simulations into several categories as follows.
\begin{itemize}
\item \textbf{Simulation Environment} includes simulation time, specification of model outputs, and ambient conditions. Also included on the page are a constantly updated list of errors, warnings, and messages about the input file specification or model simulation.
\item \textbf{Compartment Geometry} defines the size, construction characteristics, and position of the compartments in a simulation.
\item \textbf{Wall Vents, Ceiling/Floor Vents, and Mechanical Flow Vents} allows the user to connect compartments with doors and windows, ceiling and floor vents, or forced air ventilation systems.
\item \textbf{Fires} include user specification of the initial fire source and any additional burning objects in one or more of the compartments of the simulation.
\item \textbf{Detection / Suppression}defines any heat alarms and sprinklers in the compartments of the simulation.
\item \textbf{Targets} provide the ability calculate the temperature and net heat flux to objects placed and oriented arbitrarily in the structure.
\item \textbf{Surface Connections} allows for more detailed description of the connections between compartments in the simulation to better simulate the transfer of heat from compartment to compartment in the simulation.
\end{itemize}

Each of these tabbed-pages is described in more detail below. In addition, a series of menus allow the user to open and save files; run the simulation, or access help and program information.

\section{Simulation Environment}

The Simulation Environment page defines the initial conditions and simulation time for the CFAST input file. 

\begin{figure}[h!]
\begin{center}
\includegraphics[width=6.5in]{FIGURES/Input_File/Environment_Tab}
\end{center}
\end{figure}

\subsection{Naming the Calculation, the Title Input}

The first thing to do when setting up an input file is to give the simulation a title.  The first line in the CFAST input data file must be the CFAST version identification along with an optional short title for the simulation.  This is a required input.  The title command is the line that CFAST keys on to determine whether it has a correct data file.

\begin{lstlisting}
VERSN,6,CFAST Simulation
\end{lstlisting}

\clearpage

\begin{figure}[h!]
\begin{center}
\includegraphics[width=3.458in]{FIGURES/Input_File/Title}
\end{center}
\end{figure}

\textbf{Title:} The title is optional and may consist of letters, numbers, and/or symbols and may be up to 50 characters. It permits the user to label each run.

\subsection{Setting Time Limits and Output Options}

A TIMES line specifies the length of time over which the simulation takes place and how often output will be generated. This is a required input. There are one to four entries in this line.

\begin{lstlisting}
TIMES,900,50,10,10
\end{lstlisting}

\begin{figure}[h!]
\begin{center}
\includegraphics[width=3.167in]{FIGURES/Input_File/Simulation_Times}
\end{center}
\end{figure}

\begin{itemize}
\item \textbf{Simulation Time} (default units: s, default value, 900 s): The length of time over which the simulation takes place.  This is a required input which should be entered even if all other fields are not included. The maximum value for this input is 86400 s (1 day).

\item \textbf{Text Output Interval} (default units: s, default value, 50 s): The print interval is the time interval between each printing of the output values.  If omitted or less than or equal to zero, no output values will occur.

\item \textbf{Spreadsheet Output Interval} (default units: s, default value, 10 s): CFAST can output a subset of the results of the model simulation in a comma-delimited alphanumeric format which can be read by most spreadsheet software. This is designed to be imported into a spreadsheet for further analysis or graphing of the results of the simulation.  This input defines the time interval between outputs of the model results in a spreadsheet-compatible format. A value greater than zero must be used if the spreadsheet file is to be used.

\item \textbf{Smokeview Output Interval} (default units: s, default value: 10 s): CFAST can output a subset of the results in a format compatible with the visualization program smokeview. This input defines the time interval between outputs of the model results in a smokeview-compatible format.  A value greater than zero must be used if the spreadsheet output is desired.
\end{itemize}

In addition to the input data file created specifically for a CFAST simulation, there are a number files that CFAST uses to define default values and other input information, and to output the results of the simulation for later analysis.  They include 1) a thermal properties file, 2) files of predefined fire objects, and 4) a spreadsheet-compatible output file.

The thermal properties file contains material properties for compartment surfaces, target objects that may be placed in compartments in the simulation to monitor surface temperature and heat flux to the objects, and fire objects, in addition to the main fire in the simulation that may ignite based on their surface temperature or incident flux onto the surface of the object. The predefined fire objects files contain definitions for a number of reference fires from the literature or developed by the user that may be included in a simulation. The thermal properties and fire objects files may be modified by the user.  Details of the files are included in the appendices.  There are default files included in the CFAST distribution.

\subsection{Ambient Conditions}

Ambient conditions define the environment at which the scenario begins. This allows the user to specify the temperature, pressure, and station elevation of the ambient atmosphere, as well as the absolute wind pressure to which the structure is subjected.  Pressure interior to a structure is calculated simply as a lapse rate (related to the height above sea level) based on the NOAA/NASA tables \cite{GPO:Atmosphere}.  This modification is applied to the vents which connect to the exterior ambient.  The calculated pressure change is modified by the wind coefficient for each vent.  This coefficient, which can vary from -1.0 to +1.0, nominally from -0.8 to +0.8, determines whether the vent is facing away from or into the wind.  The pressure change is multiplied by the vent wind coefficient and added to the external ambient for each vent which is connected to the outside. There is an ambient for the interior and for the exterior of the structure.  Three keywords define the ambient conditions: TAMB for the interior of the structure, EAMB for the exterior of the structure is EAMB, and WIND for the wind information.

\begin{lstlisting}
EAMB,293.15,101300,0
TAMB,293.15,101300,0,50
WIND,0,10,0.16
\end{lstlisting}

\begin{figure}[h!]
\begin{center}
\includegraphics[width=4.313in]{FIGURES/Input_File/Ambient_Conditions}
\end{center}
\end{figure}

\textbf{Ambient Temperature} (default units: �C, default value: 20 �C): Initial ambient temperature inside (for TAMB) or outside (For EAMB) the structure at the station elevation.

\textbf{Ambient Pressure} (default units: Pa, default value: 101300 Pa): Initial values for ambient atmospheric pressure inside (for TAMB) and outside (for EAMB) the structure at the station elevation. Default value is standard atmospheric pressure at sea level (0 m elevation) of 101.3 kPa. Input units are in Pa. These values define standard conditions as defined in Standard Atmosphere as noted in the Handbook of Chemistry and Physics  . There is a set of numerical approximations in the CFAST code which duplicate the pressure/temperature/altitude relationships in the handbook.

\textbf{Elevation} (defaults units: m, default value: 0 m): The height where the ambient pressure and temperature were specified.  This is the reference datum for calculating the density of the atmosphere as well as the temperature and pressure inside and outside of the structure as a function of height.  

\textbf{Relative Humidity} (default units \% RH, default value: 50 \%): The initial relative humidity in the system, only specified for the interior with the TAMB command.  This is converted to kilograms of water per cubic meter.

The wind speed, scale height, and power law are used to calculate the wind coefficient for each vent connected to the outside.  The wind velocity is specified at some reference height.  The power law then provides a lapse rate for the wind speed.  An assumption is that the wind speed is zero at the surface.  The formula used to calculate the wind speed at the height of any vent is show below.  The wind is applied to each external opening as a change in pressure outside of the vent.

\textbf{Wind Speed} (default units: m/s, default value 0 m/s): Wind speed at the reference elevation.

\textbf{Scale Height} (default units: m, default value: 0 m)): Reference height at which the reference wind speed is measured.

\textbf{Power Law Coefficient} (default units: dimensionless, default value 0.16): The power law used to calculate the wind speed as a function of height. Default value is 0.16. Using the notation that $V_W$, is the wind speed at the reference height $H_W$, and $P_W$ is the power law, the exterior pressure is modified by  $\delta P = {C_W}{V^2}$ and $V = {V_W}{\left( {\frac{{{H_i}}}{{{H_W}}}} \right)^{{P_W}}}$ where $H_i$ is the position of the vent \cite{CFAST_Tech_Guide_6}.

\begin{itemize}
\item In order to see the effect of wind, the corresponding parameter for the ventilation keyword must be specified. The default for the wind vector is 0, which turns off wind effects. Please see the HVENT command, below.

\item The choice for station elevation, temperature and pressure must be consistent.  Outside of that limitation, the choice is arbitrary.  It is often convenient to choose the base of a structure to be at zero height and then reference the height of the structure with respect to that height.  The temperature and pressure must then be measured at that position.  Another possible choice would be the pressure and temperature at sea level, with the structure elevations then given with respect to mean sea level.  This is also acceptable, but somewhat more tedious in specifying the construction of a structure.  Either of these choices works though, so long as they are consistent. Usually, the station elevation is set to zero and the pressure to ambient. The effect of changing these values is small for small changes. There will be an effect for places at altitude such as Denver, Colorado, but even there the effect is not pronounced. Note that the equations implemented in the model are not designed to handle negative elevations and altitudes. It is suggested that the defaults be used.

\item These three parameters are optional. If they are not included in the input file, default values are used.
\end{itemize}

\section{Compartment Geometry}

The Compartment Geometry page defines the size, position, materials of construction, and flow characteristics for the compartments in the simulation. Initially, only the simulation environment page and the �Add� button on the compartment geometry page is enabled; all other pages are not available to the user for detailed inputs until a compartment has been added to the simulation.  

\begin{figure}[h!]
\begin{center}
\includegraphics[width=6.5in]{FIGURES/Input_File/Compartment_Geometry_Tab}
\end{center}
\end{figure}

Most of the tabbed pages in the program are of similar design, with a summary of the defined items in table form at the top of the page, a series of buttons to add, remove, or modify the item highlighted in the summary table, and a number of individual inputs below which details all of the inputs for the item selected in the summary table. The buttons included on the compartment geometry page are as follows: \\

\begin{wrapfigure}{l}{0pt}
  \includegraphics[width=0.781in]{FIGURES/Input_File/Add_Button}
\end{wrapfigure}

Use the Add button to create a new compartment with default values for all entries. \\~ \\

\begin{wrapfigure}{l}{0pt}
  \includegraphics[width=0.781in]{FIGURES/Input_File/Duplicate_Button}
\end{wrapfigure}

Use the Duplicate button to create a copy of the compartment currently selected in the summary table at the top of the page. The new compartment is added to the end of the list with the named changed to indicate it is a copy of the selected item. \\

\begin{wrapfigure}{l}{0pt}
  \includegraphics[width=0.781in]{FIGURES/Input_File/Move_Up_Button}
\end{wrapfigure}

Use the Move Up and Move Down buttons to change the order of the list of compartments in the summary table. This simply changes the automatically assigned compartment numbers for the compartments. Compartments can be ordered as desired.

\begin{wrapfigure}{l}{0pt}
  \includegraphics[width=0.781in]{FIGURES/Input_File/Remove_Button}
\end{wrapfigure}

Use the Remove button to delete the selected compartment from the list of compartments in the summary table.  Other compartments are renumbered once the compartment is deleted.

\subsection{Defining the Compartment}

In order to model a fire scenario, the size and elevation of each compartment in the structure must be specified. For a compartment, the width, depth, compartment height and height of the floor of the compartment provide this specification. The maximum number of compartments for version 6 is thirty. The usual assumption is that compartments are rectangular parallelepipeds. However, the CFAST model can accommodate odd shapes as equivalent floor area parallelepipeds or with a cross-sectional area that varies with height.

At least one compartment must be specified in the input file.  There are no defaults for compartment size. There are defaults for absolute positioning (0,0,0). The fully mixed (single zone) and corridor models are turned off by default.

\begin{wrapfigure}{r}{0pt}
  \includegraphics[width=2.0in]{FIGURES/Input_File/CFAST_Coordinates}
\end{wrapfigure}

Compartments in CFAST are most typically defined by a width, depth, and height.  If desired, compartments can be prescribed by the cross-sectional area of the compartment as a function of height from floor to ceiling for other shapes. The absolute position of the compartment with respect to a single structure reference point can be defined to ease visualization or to allow exact placement of vents and surfaces relative to other compartments in a detailed calculation. This specification is important for utilizing the corridor flow algorithm with the HALL command and for positioning the compartments for visualization in SMOKEVIEW. 

The relevant CFAST keywords are COMPA to define the compartment size and materials, HALL or ONEZ to define flow characteristics in the compartment, and ROOMA / ROOMH to define a variable cross-sectional area for the compartment. The COMPA command is required for each compartment as a basic definition for the compartment, even if there are subsequent modifications by the HALL, ONEZ, ROOMA, or ROOMH keywords which follow.  Details of the CFAST keywords are included in Appendix A.

\begin{lstlisting}
COMPA,Compartment 1,9.1,5,4.6,0,0,0,GLASSFB3,CONCRETE,CONCRETE
\end{lstlisting}

\begin{figure}[h!]
\begin{center}
\includegraphics[width=4.0in]{FIGURES/Input_File/Compartment_Name}
\end{center}
\end{figure}

\textbf{Compartment Name:} Compartments are identified by a unique alphanumeric name.  This may be a simple as a single character or number, or a description of the compartment.

\begin{figure}[h!]
\begin{center}
\includegraphics[width=3.552in]{FIGURES/Input_File/Compartment_Size}
\end{center}
\end{figure}

\textbf{Width:} specifies the width of the compartment as measured on the X axis from the origin (0,0,0) of the compartment.  

\textbf{Depth:} specifies the depth of the compartment as measured on the Y axis from the origin (0,0,0) of the compartment.  

\textbf{Height:} specifies the height of the compartment as measured on the Z axis from the origin (0,0,0) of the compartment.  

\begin{wrapfigure}{r}{0pt}
  \includegraphics[width=2.0in]{FIGURES/Input_File/CFAST_Absolute_Positioning}
\end{wrapfigure}

\textbf{Absolute Width Position:} specifies the absolute x coordinate of the lower, left, front corner of the room.

\textbf{Absolute Depth Position:} specifies the absolute y coordinate of the lower, left, front corner of the room.

\textbf{Absolute Floor Height:} specifies the height of the floor of each compartment with respect to station elevation specified by the internal ambient conditions reference height parameter.  The reference point must be the same for all elevations in the input data.  For example, the two rooms in the sample to the right would be located at (0,0,0) and (0,2,2.3).

\subsection{Thermal Properties of Bounding Surfaces}

To calculate heat loss through the ceiling, walls, and floor of a compartment, the properties of the bounding surfaces must be known. This includes the thermophysical properties of the surfaces and the arrangement of adjacent compartments if calculation of inter-compartment heat transfer is to be calculated.

The thermophysical properties of the surfaces which define compartments are described by specifying the thermal conductivity, specific heat, emissivity, density, and thickness of the enclosing surfaces for each material and then assigning the material to the ceiling, walls, and floor of a compartment.  Currently, thermal properties for materials are read from the CFAST input file with a thermal database file of common materials included in the CFAST distribution.  The thermophysical properties are specified at one condition of temperature, humidity, etc.  In CFAST version 6, there can only a single layer per boundary (previous versions allowed up to three). See the explanation in the section on auxiliary files for additional details.

\begin{lstlisting}
MATL,CONCRETE,1.75,1000,2200,0.15,0.94,"Concrete, Normal Weight (6 in)"
MATL,GLASSFB3,0.036,795,105,0.013,0.9,"Glass Fiber, Organic Bonded (1/2 in)"
\end{lstlisting}

\begin{figure}[h!]
\includegraphics[width=6.5in]{FIGURES/Input_File/Compartment_Materials}
\end{figure}

The bounding surfaces are the ceilings, walls and floors that define a compartment. These are referred to as thermophysical boundaries, since each participates in conduction and radiation as well as defining the compartments, unless these phenomena are explicitly turned off. 

\textbf{Ceiling Material} (default value: Gypsum Board): material name from the thermal properties data file used for the ceiling surface of the compartment. 

\textbf{Wall Material} (default value: Gypsum Board): material name from the thermal properties data file used for the wall surfaces of the compartment.

\textbf{Floor Material} (default value: Off): material name from the thermal properties data file used for the floor surface of the compartment.

\begin{itemize}
\item If the thermophysical properties of the enclosing surfaces are not included, CFAST will treat them as adiabatic (no heat transfer).

\item If a name is used which is not in the database, the model should stop with an error message. The keyword in the data file simply gives a name (such as CONCRETE) which refers to the properties for that material in the thermal data file (see section 4.2.3 and Appendix A for details on the thermophysical database).

\item Since most of the heat conduction is through the ceiling, and since the conduction calculation takes a significant fraction of the computation time, it is recommended that initial calculations be made using the ceiling only.  Adding the walls generally has a small effect on the results, and the floor contribution is usually negligible.  Clearly, there are cases where the above generalization does not hold, but it may prove to be a useful screening technique. A caveat in including floor properties is that the set of equations describing heat transfer becomes difficult to solve once the thermal wave from the compartments reaches the unexposed side of a floor. The back surfaces of compartments are assumed to be exposed to ambient conditions unless specifically specified (see the section on Surface Connections) to specify heat transfer connections between compartments).
\end{itemize}

\section{Compartment Connections}

Flow through vents can be natural flow through doors, windows, or openings in ceilings and floors; or forced flow in a mechanical ventilation system.  Natural flow comes in two varieties.  The first is referred to as horizontal flow.  It is the flow which is normally thought of in discussing fires.  It encompasses flow through doors, windows and so on.  The other is vertical flow and can occur if there is a hole in the ceiling or floor of a compartment.  This latter phenomena is useful in some scenarios such as in a ship where openings in floors and ceilings through scuttles are common and in buildings with manual or automatic heat and smoke venting.

Flow through normal vents is governed by the pressure difference across a vent.  There are two situa�tions which give rise to flow through vents.  The first is flow of air or smoke driven from a compartment by buoyancy.  The second type of flow is due to expansion which is particularly important when conditions in the fire environment are changing rapid�ly.  Rather than depending entirely on density differences between the two gases, the flow is forced by volumetric expansion.

In addition to natural flow, forced flow from mechanical ventilation can affect a fire as well. More important than affecting the fire, however, is the dispersal of the smoke and toxic gases from the fire to adjacent spaces, if ventilation continues to operate after a fire starts.

Atmospheric pressure is about 100 000 Pa. Fires produce pressure changes from 1 Pa to 1000 Pa and mechanical ventilation systems typically involve pressure differentials of about 1 Pa to 100 Pa.  In order to address pressure-induced flow, pressure differences of about 0.1 Pa out of 100 000 Pa for the overall problem or $10^{-4}$ Pa for adjacent compartments must be tracked.

The keywords which describe the various flow regimes are:

\begin{itemize}
\item Windows and doors (horizontal flow through vertical vents): HVENT, specifies vent which connect compartments horizontally
\item Holes in a ceiling/floor (vertical flow through horizontal vents: VVENT, specifies a vent which connects compartments vertically
\item HVAC specification: MVENT specifies a vent which connects compartments with a forced flow
\end{itemize}

For all three types of vents the size of the vent opening (expressed as a fraction of the original opening) may be changed:

\begin{itemize}
\item EVENT change the opening fraction of the specified vent at a chosen time.
\item RAMP change the opening fraction as a function of time by specifying a series of time and fraction pairs
\end{itemize}

Each of these commands is discussed in the sections that follow.

\subsection{Defining Natural Flow Connections Through Doors and Windows}

Horizontal flow connections may include doors between compartments or to the outdoors as well as windows in the compartments.  These specifications do not necessarily correspond to physically connecting the walls between specified compartments.  Rather, lack of an opening simply prevents flow between the compartments.  Horizontal flow connections may also be used to account for leakage between compartments or to the outdoors. 

Horizontal connections can only be created between compartments that physically overlap in elevation at some point. These may include doors between compartments or windows in the compartments (between compartments or to the out�doors).  Openings to the outside are included as openings to a compartment with a number one greater than the number of compartments described in the geometry section.  The relevant CFAST keyword is HVENT.  Details of the CFAST keywords are included in the appendix.

\begin{figure}[h!]
\includegraphics[width=6.5in]{FIGURES/Input_File/Natural_Flow_Tab}
\end{figure}

Most of the tabbed pages in the program are of similar design, with a summary of the defined items in table form at the top of the page, a series of buttons to add, remove, or modify the item highlighted in the summary table, and a number of individual inputs below which details all of the inputs for the item selected in the summary table. The buttons included on the horizontal flow vents page are as follows.

\begin{wrapfigure}{l}{0pt}
  \includegraphics[width=0.781in]{FIGURES/Input_File/Add_Button}
\end{wrapfigure}

Use the Add button to create a new horizontal flow vent with default values for all entries. \\~ \\

\begin{wrapfigure}{l}{0pt}
  \includegraphics[width=0.781in]{FIGURES/Input_File/Duplicate_Button}
\end{wrapfigure}

Use the Duplicate button to create a copy of the horizontal flow vent currently selected in the summary table at the top of the page. The new vent is added to the end of the list with the named changed to indicate it is a copy of the selected item. \\

\begin{wrapfigure}{l}{0pt}
  \includegraphics[width=0.781in]{FIGURES/Input_File/Move_Up_Button}
\end{wrapfigure}

Use the Move Up and Move Down buttons to change the order of the list of horizontal flow vents in the summary table. This simply changes the automatically assigned vent numbers for the vents. Vents can be ordered as desired. \\~ \\

\begin{wrapfigure}{l}{0pt}
  \includegraphics[width=0.781in]{FIGURES/Input_File/Remove_Button}
\end{wrapfigure}

Use the Remove button to delete the selected horizontal flow vent from the list of horizontal flow vents in the summary table.  Other vents are renumbered once the compartment is deleted. \\~ \\

\begin{lstlisting}
HVENT,1,2,1,1,2.4,0,1,0,0,1,1
\end{lstlisting}

\begin{figure}[h!]
\includegraphics[width=6.5in]{FIGURES/Input_File/Compartment_From_To}
\end{figure}

\textbf{First Compartment:} First of the two compartments to be connected by a horizontal flow vent.  Compartments are numbered automatically by the input editor and by the model in the order they are read from the input data file and/or the order they appear in the summary table on the compartment geometry page. Compartment numbers begin with 1, so the first compartment is number 1, the second 2, and so forth.

\textbf{Second Compartment:} Second of the two compartments to be connected by a horizontal flow vent.  Compartments are numbered automatically by the input editor and by the model in the order they are read from the input data file and/or the order they appear in the summary table on the compartment geometry page. Compartment numbers begin with 1, so the first compartment is number 1, the second 2, and so forth.

\textbf{Vent Number:} It is possible to define a total of 25 horizontal flow connections between any pair of compartments. A number from 1 to 25 uniquely identifies the connection. This number is automatically assigned by the input editor based on the order they appear in the summary table on the horizontal flow vents page.


\begin{figure}[h!]
\begin{center}
\includegraphics[width=1.1in]{FIGURES/Input_File/Vent_Size}
\end{center}
\end{figure}

\textbf{Sill} (default units: m, default value: none): Sill height is the height of the bottom of the opening relative to the floor of the compartment selected as the first compartment. 

\textbf{Soffit} (default units: m, default value: none): Position of the top of the opening relative to the floor of the compartment selected as the first compartment.   

\textbf{Width} (default units: m, default value: none): The width of the opening.

Horizontal flow vents may be opened or closed during the fire. The relevant CFAST keyword is EVENT. The initial format of EVENT is similar to HVENT specifying the connecting compartments and vent number.  Each EVENT line in the input file details the open/close time dependent characteristics for one horizontal flow vent by specifying a fractional value and a time.  The default is 1.0 which is a fully open vent.  A value of 0.5 would specify a vent which is halfway open.

Initial Opening Fraction: Flow through horizontal vents is calculated based on the area of the vent.  Normally, the vent is fully open.  If desired, the user may specify a fraction between 0 and 1 that allows the vent to be partially or fully closed at the beginning of the simulation.  In the model calculation, the vent width is multiplied by this fraction.  The opening fraction may be changed at any time in the simulation through the use of the EVENT command.

Change Opening Fraction At Time: Time during the simulation at which to change the opening fraction.

Final Opening Fraction: for horizontal flow vents, the fraction specifies the fractional width opening of the vent. Fractional values must be between 0 and 1.

Wind: The wind coefficient is the cosine of the angle between the wind vector and the vent opening.  This applies only to vents which connect to the outside.  The range of values is -1.0 to 1.0 with a default value of zero.  In the input editor, this is specified as the angle between the face of the vent and the wind direction.

There are also two parameters which are used to locate the compartments relative to each other. These are used to incorporate additional three dimensional information of the relative location of the vents with respect to each other. In the compartment view of CFAST, the orientation is that the rotation/translation point of the compartment is the back/bottom/left. In this view, both parameters would be with respect to the left hand side of the respective compartments. This allows the corridor filling model to incorporate a delay time for filling based on the separation between the vents. These parameters are needed only if the HALL command is used. 

First Compartment Offset: Horizontal distance between the centerline of this vent and the reference point in the first compartment.

Second Compartment Offset: Horizontal distance between the centerline of this vent and the reference point in the second compartment.
\begin{itemize}
\item The soffit and sill specifications are with respect to the first compartment specified and is not necessarily symmetric since the elevation of the second compartment may be different than the first.  Reversing the order of the compartment designations does make a difference.
\end{itemize}

\chapter{Output from CFAST}
\label{Output_Chapter}

The output of CFAST includes the temperatures of the upper and lower gas layers within each compartment, the ceiling/wall/floor temperatures within each compartment, the visible smoke and gas species concentrations within each layer, target temperatures and sprinkler activation time.  The amount of information can be very large, especially for complex geometries and long simulations.

\section{Compact Output}

The default output to the console is called the compact form, and shows the basic information about a scenario, including layer temperatures and the size of fires. Default text output provides a simple overview for the user to make sure the case runs as expected.
\begin{lstlisting}[basicstyle=\scriptsize]
****************************
* Time =   3600.0 seconds. *
****************************

Compartment    Upper     Lower      Inter.      Upper           Upper      Lower       Pressure
               Temp.     Temp       Height      Vol             Absor      Absorb
               (C)       (C)        (m)         (m^3)           (1/m)      (1/m)       (Pa)
----------------------------------------------------------------------------------------------------
Comp 1         48.52      20.89      1.542       36.    ( 49%)  0.172      9.306E-02   -0.570
Comp 2         138.4      36.39      1.586       35.    ( 47%)  0.171      9.338E-02    -1.86
Comp 3         103.7      27.50      1.000       50.    ( 67%)  0.161      0.104        0.983
\end{lstlisting}
The first column contains the compartment ID.  On each row with its compartment number from left to right is the upper layer temperature, lower layer temperature, the height of the interface between the two layers, the total pyrolysis rate, and finally the total fire size.  The only value given for the outside is the total heat release rate of fires venting to the outside.

\section{Detailed Outputs}

The following sections describe each of the outputs from the model.  Each section refers to a specific part of the print out and appears in the order the output appears. A description of each option follows. When running CFAST from within CEdit an output file is automatically generated based on the input file name. For example, running the example file  {\ct Users\_Guide\_Example.in} generates an output file named  {\ct Users\_Guide\_Example.out}. All the detailed outputs described in the following sections are included in the output file.

\subsection{Output for Initialization}

This prints the initial conditions to the output before the actual run starts.  This merely mimics the inputs specified by the user in the input data file  The initial conditions break down into seven sections.  Each is described below with the section name. The following explanation uses the output from the case {\ct Users\_Guide\_Example.in} which is included in the distribution.

\subsubsection{Overview}

The overview gives a general description of the case.  The output is fairly self explanatory. ``Doors, ...'' is the total number of horizontal natural flow vent connections or wall vents in all compartments of the simulation.  ``Ceil. Vents, ...'' gives the total number of vertical natural flow vent connections or ceiling/floor vents in all compartments of the simulation.  The last header on the line ``MV Connections'' has the total number mechanical flow connections to all compartments in the simulation. Times in these outputs are the times discused in section \ref{info:TIME}. All times are in s.
\begin{lstlisting}[basicstyle=\tiny]
CFAST

Release Version  : CFAST 7.3.1
Revision         : CFAST7.3.1-0-g6b9b32bb
Revision Date    : Mon Apr 8 12:38:20 2019 -0400
Compilation Date : Thu 04/18/2019  12:05 PM

Data file: C:\Users\rpeacoc\firemodels\cfast\Utilities\for_bundle\Bin\Data\Users_Guide_Example.in
Title: Users Guide Example Case


OVERVIEW


Compartments    Doors, ...    Ceil. Vents, ...    MV Connects
-------------------------------------------------------------
   3               3             1                    2

Simulation     Output         Smokeview      Spreadsheet
Time           Interval       Interval       Interval
   (s)            (s)            (s)            (s)
--------------------------------------------------------
   3600.00          60.00          60.00          60.00
\end{lstlisting}

\subsubsection{Ambient Conditions}

This section, like the overview section, needs little elaboration.  It gives the starting atmospheric conditions for the simulation both for outside and inside the structure. Temperatures are in \degc and pressure in Pa.
\begin{lstlisting}[basicstyle=\tiny]
AMBIENT CONDITIONS

Interior       Interior       Exterior       Exterior
Temperature    Pressure       Temperature    Pressure
  (C)            (Pa)           (C)            (Pa)
-----------------------------------------------------
    20.          101325.          20.          101325.
\end{lstlisting}

\subsubsection{Compartments}
The compartments section gives a summary of the geometry for the simulation.  A simple table summarizes the geometry with compartments running down the page in order specified.  The various dimensions for each compartment are on the row with its compartment number and name.  Two columns need explanation.  The second to last column ``Ceiling Height'' gives the height of the ceiling relative to the base height which is 0.  Similarly the ``Floor Height'' refers to the height of the floor above the base height.

\begin{lstlisting}[basicstyle=\tiny]
COMPARTMENTS

Compartment  Name                Width        Depth        Height       Floor        Ceiling    Shaft    Hall
                                                                        Height       Height
                                 (m)          (m)          (m)          (m)          (m)
-------------------------------------------------------------------------------------------------------------
    1               Comp 1        5.00         5.00         3.00         0.00         3.00
    2               Comp 2        5.00         5.00         3.00         0.00         3.00
    3               Comp 3        5.00         5.00         3.00         3.00         6.00
\end{lstlisting}
All the information in this table comes from the compartments tab discused in section  \ref{info:COMP}


\subsubsection{Wall Vents}
This is the first table in the vent connections section.  Each row in the table characterizes one vent.  The first two columns contain the two compartments connected by the vent. The third column gives the vent number.  Column four is the width of the vent.  The next two columns report the sill and soffit height for the vent relative to the floor of the first compartment.  The seventh and eighth columns have a second listing of the sill and soffit height, this time relative to the base height.

\begin{lstlisting}[basicstyle=\tiny]
VENT CONNECTIONS

Wall Vents (Doors, Windows, ...)

From         To           Vent    Width  Sill    Soffit  Open/Close  Trigger             Initial  Initial     Final   Final
Compartment  Compartment  Number         Height  Height  Type        Value       Target  Time     Fraction    Time    Fraction
                                  (m)    (m)     (m)     (m)         (C/W/m^2)           (s)                  (s)
------------------------------------------------------------------------------------------------------------------------------
Comp 1       Outside      1       1.00   0.00    2.00    Time                            0.00      1.00       0.00    1.00
Comp 1       Comp 2       2       1.00   0.00    2.00    Time                            0.00      1.00       0.00    1.00
Comp 3       Outside      3       1.00   1.00    2.00    Time                            0.00      1.00       0.00    1.00
\end{lstlisting}
From compartment, to compartment, vent number, width, sill height, and soffit height all come directly from the wall vent tab discussed in section \ref{info:VENT}. Vent opening and closing information can be specified by time, by an associated time / fraction history or by an associated target to trigger a change in the vent opening by temperature or heat flux are all set in the same tab.

\subsubsection{Ceiling and Floor Vents}

The first column is the upper compartment.  The upper compartment is the compartment where the vent opens into the floor.  The second column is the lower compartment where the vent is in the ceiling. Each vent connection between compartment pairs is also identified by a vent number index beginning with 1. The fourth column describes the shape of the vent, which can be either round or square.  The fifth column gives the area of the vent. Vent opening and closing information can be specified by time, by an associated time / fraction history or by an associated target to trigger a change in the vent opening by temperature or heat flux.
\newpage
\begin{lstlisting}[basicstyle=\tiny]
Ceiling and Floor Vents

Top          Bottom       Vent    Shape  Area      Open/Close  Trigger                 Initial   Initial     Final     Final
Compartment  Compartment  Number                   Type        Value       Target      Time      Fraction    Time      Fraction
                                         (m^2)                 (C/W/m^2)               (s)                   (s)
--------------------------------------------------------------------------------------------------------------------------------
Comp 3       Comp 2        1      Round  1.00      RAMP # 1
\end{lstlisting}
Top compartment, bottom compartment, shape, and area come from the Ceiling/Floor Vent tab discuussed in section \ref{info:VENT2}. Relative height is the height of the vent above the floor of the bottom compartment and absolute height is the height of the vent above the base elevation.  Vent opening and closing information can be specified by time, by an associated time / fraction history or by an associated target to trigger a change in the vent opening by temperature or heat flux are all set in the same tab.

\subsubsection{Mechanical Flow Connections}

This section lists all connections to compartments and fans that connect between compartments. The table lists, in order, the compartments connected by the fan, a numeric index assigned to the fan beginning with 1.  A fan actually draws air from the first or ``from'' compartment and pushes it to the second or ``to'' compartment. The fourth column is the cross-sectional area of the duct connection to the chosen compartment. Vent opening and closing information can be specified by time, by an associated time / fraction history or by an associated target to trigger a change in the vent opening by temperature or heat flux.

\begin{lstlisting}[basicstyle=\tiny]
Mechanical Vents (Fans)

From         To            Fan      Area    Flowrate   Open/Close  Trigger              Initial   Initial     Final     Final
Compartment  Compartment   Number                      Type        Value       Target   Time      Fraction    Time      Fraction
                                    (m^2)   (m^3/s)                (C/W/m^2)            (s)                   (s)
--------------------------------------------------------------------------------------------------------------------------------
Outside      Comp 1        1        0.02    0.25       RAMP # 2
Comp 2       Outside       2        0.02    0.25       RAMP # 3
\end{lstlisting}
From compartment, to compartment, fan number, area, and flowrate are all set in the Mechanical Ventilation tab discussed in sections \ref{info:VENT3} and \ref{info:VENT4}.  Vent opening and closing information can be specified by time, by an associated time / fraction history or by an associated target to trigger a change in the vent opening by temperature or heat flux are all set in the same tab.


\subsubsection{Ventilation Opening Time Specifications}

Opening and closing of vents in CFAST can be specified by a series of times and opening fractions for a specified vent. By default, all vents are open at the beginning of a simulation and remain open throughout the simulation. On the tab for each type of ventilation a time / fraction history can be associated with the vent. These time / fraction histories are called Vent Ramps and have their own tables.
\begin{lstlisting}[basicstyle=\tiny]
VENT RAMPS

Type  From           To              Vent
      Compartment    Compartment     Number
                                                          (s)       (s)       (s)
-----------------------------------------------------------------------------------
V     Comp 3         Comp 2          1       Time           0       100       500
                                             Fraction    0.00      0.50      1.00
M                    Comp 1          1       Time           0       100       500
                                             Fraction    0.00      0.50      1.00
M     Comp 2         Outside         2       Time           0       100       500
                                             Fraction    0.00      0.50      1.00
\end{lstlisting}

\subsubsection{Thermal Properties}

The thermal properties section is broken into two parts.  The first part is a table that lists the material for each surface of each compartment.  The compartments appear as rows down the page in order of specification.  From left to right next to the compartment name comes the material for the ceiling, wall and floor.  The second table lists the properties of each material used in the simulation. For each listing of a material, the name is followed by the conductivity, specific heat, density, thickness and emissivity. In addition to materials for compartment surfaces, any thermal properties specified for targets are also listed (this may include thermal properties for gaseous materials specified as fire sources in a simulation.)

\begin{lstlisting}[basicstyle=\tiny]
THERMAL PROPERTIES

Compartment    Ceiling      Wall         Floor
-----------------------------------------------
Comp 1       CONCRETE     CONCRETE     CONCRETE
Comp 2       CONCRETE     CONCRETE     CONCRETE
Comp 3       CONCRETE     CONCRETE     CONCRETE


Name    Conductivity      Specific Heat     Density        Thickness     Emissivity
-----------------------------------------------------------------------------------
CONCRETE     1.75           1.000E+03       2.200E+03       0.150           0.940
DEFAULT     0.120            900.            800.           1.200E-02       0.900
\end{lstlisting}

Material choices of the ceiling, walls, and floors is discussed in section \ref{info:COMP2}. Setting thermalphysical properties is done in the Thermal Properites tad discussed in section \ref{info:MATL}. Units for thermal properties are standard S.I. units.  For thermal conductivity,  kW/(m$\cdot$K); for specific heat,  kJ/(kg$\cdot$K); for density, kg/m$^3$; for thickness, m; emissivity is dimensionless.


\subsubsection{Fires}

The fire section lists all the information about all fires that might exist.  All the information for each fire is listed separately.   Each fire listing has the same form.  First is the name of the fire followed by a list of general information.  Listed left to right is the compartment the fire is in, the type of fire, the initial x (width), y (depth), z (height) position of the fire, the relative humidity, the lower oxygen limit, and finally the radiative fraction for the fire.

A table of time history curves for the fire follows.  The table contains all the time history curves for the fire.  Each row on the table is a specific time given in the left most column.  The rest of the columns give the values at that particular time.  The column headers indicate each input quantity and correspond to specific keywords in the fire definition. The headings are defined as follows: `Mdot' is pyrolysis rate; `Hcomb' is the heat of combustion; `Qdot' is the heat release rate; `Zoffset' is height of the fire above the base z-position; `Soot' is the fraction of the fuel mass converted to soot during combustion; `CO' is the fraction of the fuel mass converted to carbon monoxide during combustion; `HCN' is the fraction of the fuel mass converted to hydrogen cyanide during combustion; `HCl' is the fraction of the fuel mass converted to hydrogen chloride during combustion; and `TS' is the fraction of fuel mass converted to trace species during combustion.

\begin{lstlisting}[basicstyle=\tiny]
FIRES


Name: Cushion   Referenced as object #  1 Normal fire

Compartment    Fire Type       Time to Flaming      Position (x,y,z)     Relative    Lower O2    Radiative
                                                                         Humidity    Limit       Fraction
--------------------------------------------------------------------------------------------------------------
Comp 1         Constrained           0.0           2.50   2.50   0.00     50.0        10.00        0.33


Chemical formula of the fuel
Carbon     Hydrogen  Oxygen    Nitrogen  Chlorine
--------------------------------------------------
  9.000     6.000     2.000     2.000     0.000


  Time      Mdot      Hcomb     Qdot      Zoffset   Soot      CO        HCN       HCl       TS
  (s)       (kg/s)    (J/kg)    (W)       (m)       (kg/kg)   (kg/kg)   (kg/kg)   (kg/kg)   (kg/kg)
------------------------------------------------------------------------------------------------------
     0.      0.0      5.00E+07   0.0       0.0      0.23      8.46E-02  0.31       0.0       0.0
    60.     2.00E-03  5.00E+07  1.00E+05   0.0      0.23      8.46E-02  0.31       0.0       0.0
   120.     3.00E-03  5.00E+07  1.50E+05   0.0      0.23      8.46E-02  0.31       0.0       0.0
   180.     4.00E-03  5.00E+07  2.00E+05   0.0      0.23      8.46E-02  0.31       0.0       0.0
   240.     3.00E-03  5.00E+07  1.50E+05   0.0      0.23      8.46E-02  0.31       0.0       0.0
   300.     2.50E-03  5.00E+07  1.25E+05   0.0      0.23      8.46E-02  0.31       0.0       0.0
   360.     2.00E-03  5.00E+07  1.00E+05   0.0      0.23      8.46E-02  0.31       0.0       0.0
   420.     1.80E-03  5.00E+07  9.00E+04   0.0      0.23      8.46E-02  0.31       0.0       0.0
   480.     1.60E-03  5.00E+07  8.00E+04   0.0      0.23      8.46E-02  0.31       0.0       0.0
   540.     1.50E-03  5.00E+07  7.50E+04   0.0      0.23      8.46E-02  0.31       0.0       0.0
  1800.     1.50E-03  5.00E+07  7.50E+04   0.0      0.23      8.46E-02  0.31       0.0       0.0
\end{lstlisting}
All of the inputs for all fires come from the fire specifications set in the Fires tab discussed in section \ref{info:FIRE}.  Units for most values are included in the output.  Fire position is in m, relative humidity is in \%, lower oxygen limit is in volume \%, and pyrolysis temperature is in K.


\subsubsection{Targets}

The entry for targets shows the orientation of additional targets specified in the data file. Targets explicitly specified in the data file are listed first in the order they are included in the data file. The compartment number, position of the target within the compartment, direction of the front face of the target object expressed as a normal unit vector to the surface, and object material.

\begin{lstlisting}[basicstyle=\tiny]
TARGETS

Target                      Compartment    Position (x, y, z)         Direction (x, y, z)      Material
------------------------------------------------------------------------------------------------------
    1   Targ 1                Comp 1       2.20     1.88     2.34     0.00     0.00     1.00   CONCRETE
\end{lstlisting}

All of the inputs for targets come from the Targets tab discussed in chapter \ref{info:DEVC}. Direction is specified as a unit vector as described in the section on target input. Units for position and direction are all in m.

\subsubsection{Detectors and Sprinklers}

The entry for each detector or sprinkler shows the compartment and position of the device and its activation characteristics. For smoke detectors, activation is based on the smoke obscuration at the position of the detector; for heat detectors and sprinkler, the temperature of the detector.


\begin{lstlisting}[basicstyle=\tiny]
DETECTORS/ALARMS/SPRINKLERS
Target Compartment Type   Position (x, y, z) Activation  Flaming     Smoldering
                                             Obscuration Obscuration Obscuration Temperature RTI        Spray Density
                          (m)  (m)  (m)      (%/m)       (%/m)       (%/m)       (C)         (m s)^1/2  (mm/s)
---------------------------------------------------------------------------------------------------------------------
  1    Comp 1      SPRINK 3.00 3.00 2.97                                         73.89       100.00     7.00E-02
  2    Comp 1      SMOKE  2.00 2.00 2.97     23.93
  3    Comp 1      HEAT   2.00 2.00 2.97                                         30.00         5.00
\end{lstlisting}

All of the inputs for detectors and sprinklers come from the Detector / Surpression tab discussed in section \ref{info:DEVC2}. Units for position are all in m.

\subsection{Output for Main Variables}

The normal print out is the first information printed at each time interval.  This information includes the layer temperatures, interface height, volume of the upper layer, layer absorption coefficients, and compartment pressure (relative to ambient).

\begin{lstlisting}[basicstyle=\tiny]
Time =   3600.0 seconds.

Compartment    Upper     Lower      Inter.      Upper           Upper      Lower       Pressure
               Temp.     Temp       Height      Vol             Absor      Absorb
               (C)       (C)        (m)         (m^3)           (m^-1)     (m^-1)      (Pa)
----------------------------------------------------------------------------------------------------
Comp 1         78.70      23.01      1.404       40.    ( 53%)  0.235      0.100       -0.430
Comp 2         176.5      44.97      1.495       38.    ( 50%)  0.240      9.153E-02    -1.49
Comp 3         104.3      27.58      1.248       44.    ( 58%)  0.241      9.556E-02   -0.336
\end{lstlisting}
The second table of the normal print out has information about the fires.  In essence it is two tables joined.  The first part lists information by fire. It lists fires in the order they are specified in the input file down the page.  The fires are listed in the second column followed by the plume flow rate, the pyrolysis rate, the fire size, and flame height.  The next three columns are then skipped.  The next column with information is the amount of heat given off by each fire convectively, followed by the amount of heat given off radiantly. The last two columns give the total mass pyrolyzed and the amount of trace species produced.  The second part starts after all the fires have been individually listed.  It gives the totals for all fires in each compartment.  The first column has the compartment name.  The compartments are listed down the page in order specified.  The third to fifth columns are the same as the first part except the values are totals for the compartment and not just for one fire.  The sixth column has the total heat release rate that occurs in the upper layer.  The next column has the same total in the lower layer.  The eighth column has the total size of vent fires in the compartment.

\begin{lstlisting}[basicstyle=\tiny]
FIRES

Compartment Fire      Ign Plume     Pyrol     Fire      Flame  Fire in Fire in   Vent Convec.   Radiat.   Pyrolysate Trace
                          Flow      Rate      Size      Height Upper   Lower     Fire
                          (kg/s)    (kg/s)    (W)       (m)    (W)     (W)       (W)  (W)       (W)       kg)        (kg)
 --------------------------------------------------------------------------------------------------------------------------
            Cushion   Y   1.802E-04 1.503E-07 7.51      0.00                          5.04      2.48       1.06      0.00
            Wood_Wall Y   1.57      2.486E-02 4.500E+05 0.382                         3.015E+05 1.485E+05 44.8       0.00

Comp 1                    1.802E-04 1.503E-07 7.51             0.00    7.51      0.00
Comp 2                    1.57      2.486E-02 4.500E+05        0.00    4.500E+05 0.00
\end{lstlisting}
% This seems out of place. Like it should be in the technical reference, which it probably is, but not here.
Flame height is calculated from the work of Heskestad~\cite{Heskestad:2002}. The average flame height is defined as the distance from the fuel source to the top of the visible flame where the intermittency is 0.5.  A flame intermittency of 0.5 means that the visible flame is above the mean 50~\% of the time and below the mean 50~\% of the time.

\subsection{Output for Wall Surfaces, Targets, and Detectors/Sprinklers}

The printed output provides two tables displaying information about wall surface or target temperatures and fluxes, and heat detectors or sprinklers. The left most column specifies the compartment name; followed by four columns providing the temperatures of the bounding surfaces of the compartment in contact with the ceiling, upper wall surface (in contact with the upper layer gases), lower wall surface (in contact with the lower layer gases), and floor, in that order. Next comes information about targets in the compartment, with each target listed on a separate line.  Information in the columns includes the surface temperature of the target, net heat flux to the target, and the percentage of that net flux that is due to radiation from the fire, radiation from compartment surfaces, radiation from the gas layers, and convection from the gas surrounding the target.  CFAST includes one target in the center of the floor for all compartments. Information on additional targets specified by the user in the input data file are also included, in the order specified in the input file.

For smoke detectors, heat detectors, and sprinklers, the temperature of the device, its current state (activated or not), and the nearby gas temperature and velocity are included.

\begin{lstlisting}[basicstyle=\tiny]
SURFACES AND TARGETS

Compartment Ceiling Up wall Low wall Floor  Target  Gas       Surface   Interior Incident  Net      Gas    Heat
            Temp.   Temp.   Temp.    Temp.          Temp.     Temp.     Temp.    Flux      Flux     FED    FED
            (C)     (C)     (C)      (C)            (C)       (C)       (C)      (W/m^2)   (W/m^2)
-----------------------------------------------------------------------------------------------------------------
Comp 1       25.7   24.5    21.4     22.0
                                            Targ 1  48.5      24.5      22.2     451.      190.     94.9   0.134
Comp 2      129.    51.8    76.4     46.9
Comp 3       33.8   33.7    22.7     23.1

\end{lstlisting}

\begin{lstlisting}[basicstyle=\tiny]
DETECTORS/ALARMS/SPRINKLERS
                                      Sensor         Smoke

Number  Compartment        Type       Temp (C)       Temp (C)      Vel (m/s)     Obs (1/m)          Activated
--------------------------------------------------------------------------------------------------------------
  1     Comp 1             SPRINK     4.706E+01      4.852E+01     1.194E-01                        YES
  2     Comp 1             SMOKE                     4.852E+01     1.194E-01     7.875E-01          YES
  3     Comp 1             HEAT       4.850E+01      4.852E+01                                      YES
\end{lstlisting}
In all cases, the flux to/from a target is net radiation or net convection. That is, it is the incoming minus the outgoing. So while a target or object is heating, the flux will be positive, and once it starts to cool, the flux will be negative. Values for radiation from fires (fire rad.), radiation from surfaces (surface rad.), radiation from the gas layers (gas rad.), and convection from surfaces (convect) are expressed as the net flux to target (flux to target). Positive values indicate heat gains by the target and negative values indicate heat losses.


\subsection{Output for Gas Species}

The output has two tables displaying information about the amounts of species in each layer. The species information follows the normal print out.  The first table gives species volume \% for the upper layers of all the compartments and the second reports the same for the lower layers of all the compartments.  Again the compartments are listed down the page and the information across the page.  The species are each given in one of several different terms.  Below each header are the units for the given value.  Most of the headers are simply the chemical formula for the species being tracked (Elemental nitrogen, oxygen, carbon dioxide, carbon monoxide, hydrogen cyanide, hydrochloric acid, and water).  However, several are not obvious.  ``TUHC'' is the total unburned hydrocarbons or the pyrolyzed fuel that hasn't burned yet.  ``OD'' is the optical density, which is a measure of the amount of smoke. ``OD S'' and ``OD F'' and the optical density resulting from only smoldering and flaming smoke production, respectively. ``TS'' is trace species.

\begin{lstlisting}[basicstyle=\tiny]
UPPER LAYER SPECIES

Compartment N2    O2    CO2   CO         HCN        HCL   TUHC  H2O   OD     OD_F   OD_S   TS
            (%)   (%)   (%)   (%)        (%)        (%)   (%)   (%)   (1/m)  (1/m)  (1/m)  kg
-------------------------------------------------------------------------------------------------
Comp 1      77.3  19.1  1.29  8.077E-03  2.025E-04  0.00  0.00  2.27  0.788  0.788  0.00   0.00
Comp 2      77.0  18.6  1.71  1.061E-02  2.008E-05  0.00  0.00  2.63  0.808  0.808  0.00   0.00
Comp 3      77.0  18.7  1.64  1.018E-02  2.190E-05  0.00  0.00  2.57  0.847  0.847  0.00   0.00
\end{lstlisting}

The report by species are \% by volume. Optical depth per meter is a measure of the visibility in the smoke. Trace species (TS) is the total mass of the trace species that is present in the layer in kilograms. It is an absolute measure and not percent or density.

\subsection{Output for Vent Flows}
% Need to find out if the second option still exists. It doesn't appear to be in CEdit Whole section needs to be reviewed after getting the current output of CFAST
Information about vent flow is obtained in this section.  It includes information detailing mass flow through horizontal, vertical, and mechanical vents. There are two forms for the vent flow. The first is flow through the vents as kg/s. The second gives the total mass which has flowed through the vent(s) and the relative mass of trace species divided by the total mass of trace species produced up to the current time.

The section for vent flow is titled ``FLOW THROUGH VENTS (kg/s).''  Because flow is always given in positive values, each vent is listed twice, once for flow going from compartment A to compartment B (labelled as ``Flow relative to `from''') and a second time for flow from B to A (labelled as ``Flow relative to `To'''.  As the example below shows, the first column specifies the vent, including the type of vent (an ``H'' in this column stands for horizontal flow, such as through a doorway or window; a ``V'' here would mean vertical flow, such as through an opening in the ceiling, and an ``M'' stands for a mechanical ventilation connection) and the compartment from which the flow comes. The second column lists the name of the compartment. Up to six additional columns detail the flow at this vent. Flow into and out of the compartment through the vent in the upper and lower layers are included.

A second table shows total (mass) flow through vents. At present this is confined to mechanical ventilation (It applies only to vents which can be filtered, in this case mechanical ventilation). The last column is obtained by summing the outflow/inflow for each vent and then dividing that sum by the total trace species produced by all fires. For details on this value, see the section on output listing for fires.

\begin{lstlisting}[basicstyle=\tiny]
FLOW THROUGH VENTS (kg/s)

                                Flow relative to 'From'                             Flow Relative to 'To'
                               Upper Layer             Lower Layer              Upper Layer            Lower Layer
Vent From/Bottom To/Top    Inflow      Outflow    Inflow       Outflow      Inflow      Outflow    Inflow     Outflow
------------------------------------------------------------------------------------------------------------------------
H 1  Comp 1      Outside  0.5279E-03              0.1480E+01                                                  0.1480E+01
H 2  Comp 1      Comp 2   0.1093E+00  0.1051E+00              0.1508E+01   0.3196E-03  0.1093E+00  0.1613E+01
H 3  Comp 3      Outside              0.1516E+01              0.3120E-03   0.1517E+01
V 1  Comp 2      Comp 3               0.1461E+01              0.5241E-01   0.1514E+01
M 1  Outside     Comp 1                                       0.2391E-01   0.2391E-01
M 2  Comp 2      Outside              0.1703E-01                                                   0.1703E-01


TOTAL MASS FLOW THROUGH MECHANICAL VENTS (kg)

To             Through              Upper Layer               Lower Layer              Trace Species
Compartment    Vent             Inflow       Outflow      Inflow       Outflow       Vented    Filtered
--------------------------------------------------------------------------------------------------------
Outside        Comp 1                                                 0.8190E+02
Comp 2         Outside                      0.6667E+02                0.2223E-01

\end{lstlisting}


\section{Spreadsheet Output}

CFAST generates a number of output files in a plain text spreadsheet format.  These files capture a snap shot of the modeling data at an instant of time. This instance is on the Simulation tab discussed in section \ref{info:TIME}. \emph{However}, there are events which can occur in between these reporting periods. Examples are the ignition of objects and the activation of detectors or sprinklers. These are \emph{not} reported in these output files.

Several spreadsheet files are generated with each CFAST simulation:

\begin{enumerate}
\item \textbf{project\_compartments.csv} contains layer temperature and height, species concentrations, and fire-related outputs.
\item \textbf{project\_devices.csv} contains outputs for targets and detectors (detailed target info with -V option)
\item \textbf{project\_masses.csv} contains mass of species in compartment layers and total trace species masses (detailed unburned fuel info with -V flag)
\item \textbf{project\_vents.csv} contains vent flows (additional flow details with -V option)
\item \textbf{project\_walls.csv} contains compartment surface temperatures
\end{enumerate}

\subsection{Primary Output Variables (project\_compartments.csv)}

There are three sets of information in this file. The first includes compartment information such as layer temperature. This is output by compartment and there are entries for each compartment plus a column that indicates the current simulation time:

\begin{description}
\item[Time] (s)
\item[Upper Layer Temperature] (\degc)
\item[Lower Layer Temperature] (\degc)
\item[Layer Height]  (m)
\item[ Upper Layer Volume] (m$^3$): total volume of the upper layer. This is just the floor area times the difference between the ceiling height and the layer height.
\item[ Pressure] (Pa): pressure at compartment floor relative to the outside at the absolute height of the floor.
\item[N2 Upper/Lower Layer] (mol \%): nitrogen concentration in the upper (or lower) layer in the current compartment
\item[O2 Upper/Lower Layer] (mol \%): oxygen concentration in the upper (or lower) layer in the current compartment
\item[CO2 Upper/Lower Layer] (mol \%):  carbon dioxide concentration in the upper (or lower) layer in the current compartment
\item[CO Upper/Lower Layer] (mol \%):  carbon monoxide concentration in the upper (or lower) layer in the current compartment (multiply by 10\,000 to convert to ppm)
\item[HCN Upper/Lower Layer] (mol \%):  HCN concentration in the upper (or lower) layer in the current compartment (multiply by 10\,000 to convert to ppm)
\item[HCL Upper/Lower Layer] (mol \%):  HCl concentration in the upper (or lower) layer in the current compartment (multiply by 10\,000 to convert to ppm)
\item[H2O Upper/Lower Layer] (mol \%):  water vapor concentration in the upper (or lower) layer in the current compartment
\item[Optical Density Upper/Lower Layer] (m$^{-1}$):  optical density in the upper (or lower) layer in the current compartment
\item[Optical Density Upper/Lower Layer] (m$^{-1}$):  optical density from flaming-generated smoke in the upper (or lower) layer in the current compartment
\item[Optical Density Upper/Lower Layer] (m$^{-1}$):  optical density from smoldering-generated smoke in the upper (or lower) layer in the current compartment
\item[Trace Species Upper/Lower Layer] (kg):  total mass of trace species in the upper (or lower) layer in the current compartment
\item[HRR Door Jet Fires] (W): total heat release rate of all door jet fires \emph{adding} heat to this compartment.
\end{description}

The second set of information is for fires. This information is displayed for each fire:

\begin{description}
\item[Ignition]: indicates whether the fire has ignited. entry is zero if the fire has not yet ignited and one if it has ignited.
\item[Plume Entrainment Rate] (kg/s): current mass entrained from the lower layer into the plume for this fire.
\item[Pyrolysis Rate] (kg/s): current rate of mass loss for this fire.
\item[HRR Expected] (W): current heat release rate input for this fire. This is the HRR input by the user before it is adjusted for available oxygen or sprinkler activation.
\item[HRR Actual] (W): current total heat release rate for this fire. This is just the sum of the heat release rate for the lower layer and upper layer for this fire. It may be lower than the user input if the fire is constrained by available oxygen or if a sprinkler has activated in the compartment.
\item[HRR Convective actual] (W): current rate of heat release by convection for this fire.  The remainder is released by radiation to the surroundings.
\item[HRR Lower Actual] (W): current heat release rate for burning in the lower layer for this fire.
\item[HRR Upper Actual] (W):  current heat release rate for burning in the upper layer for this fire.
\item[Flame Height] (m): current calculated flame height for this fire.
\item[Total Pyrolysate Released] (kg): total mass released by the fire up to the current time.
\item[Total Trace Species Released] (kg): total mass of trace species released by the fire up to the current time.
\end{description}

The last set of information is for door jet fires. This information is displayed for each compartment and for the outside:

\begin{description}
  \item[HRR Door Jet Fires] (W): current total heat release rate for door jet fires in this compartment.
\end{description}


\subsection{Target Temperature and Heat Flux (project\_devices.csv)}

This file provides information on target temperatures and flux, and reports on the current state of detectors and sprinklers (as a sub-set of detectors). The output is in two sections, one for target temperature and heat flux, and one for detector/sprinkler temperature and activation. The first set of columns pertain to the targets.
The second set of columns pertain to the user-defined targets included in the simulation.
\begin{description}
\item[Target Surrounding Gas Temperature] (\degc): gas temperature nearby the current target
\item[Target Surface Temperature] (\degc): temperature of the surface of the current target
\item[Target Internal Temperature] (\degc): interior temperature of the current target
\item[Target Incident Flux] (kW/m$2$): total heat flux striking the front surface of the current target
\item[Target Net Flux] (kW/m$2$): total net heat flux to the front surface of the current target
\item[Target Gas FED]: fractional effective dose due to toxic gases at the current target location
\item[Target Gas FED Increment]: incremental fractional effective dose due to toxic gases at the current target location
\item[Target Heat FED]: fractional effective dose due to convective heat at the current target location
\item[Target Heat FED Increment]: incremental fractional effective dose due to convective heat at the current target location
\item[Target Obscuration] (m$^{-1}$):  optical density at the current target location
\item[Target Convective Flux] (kW/m$^2$): convective heat flux to the  front surface of the current target (validation output only)
\item[Target Radiative Flux] (kW/m$^2$): total net radiative heat flux to the front surface of the current target (validation output only)
\item[Target Fire Radiative Flux] (kW/m$^2$): radiative heat flux from fires to the front surface of the current target (validation output only)
\item[Target Surface Radiative Flux] (kW/m$^2$): radiative heat flux from compartment surfaces to the front surface of the current target (validation output only)
\item[Target Gas Radiative Flux] (kW/m$^2$):   radiative heat flux from the upper and lower gas layers to the front surface of the current target (validation output only)
\item[Target Radiative Loss Flux] (kW/m$^2$):   radiative heat flux from the current target to surroundings at the calculated temperature of the target (validation output only)
\item[Target Total Gauge Flux] (kW/m$2$): total net heat flux to the front surface of the current target assuming the target radiative losses are at ambient temperature (validation output only)
\item[Target Radiative Gauge Flux] (kW/m$^2$): total net radiative heat flux to the front surface of the current target assuming the target radiative losses are at ambient temperature (validation output only)
\item[Target Radiative Loss Gauge Flux] (kW/m$^2$):   radiative heat flux from the current target to surroundings assuming the target radiative losses are at ambient temperature (validation output only)
\end{description}

The second set of columns pertain to the detector/sprinkler output.

\begin{description}
\item[Sensor Temperature] (\degc): temperature of the current detector / sprinkler
\item[Sensor Activation] (none): indicator of activation of the current detector / sprinkler; takes a value of zero if the sensor has not activated and one if it has
\item[Sensor Surrounding Gas Temperature] (\degc): gas temperature nearby the current detector / sprinkler. This is the ceiling jet temperature at the device location if the device is in the ceiling jet or the appropriate gas layer temperature if the device is lower in the compartment
\item[Sensor Surrounding Gas Velocity] (m/s): gas velocity nearby the current detector / sprinkler. This is the velocity of the ceiling jet at the device location if the device is in the ceiling jet or a default value of 0.1 m/s if the device is lower in the compartment
\item[Sensor Obscuration] (m$^{-1}$):  optical density at the current sensor location
\end{description}

\subsection{Layer Masses (project\_masses.csv)}

\begin{description}
\item[Time] (s)
\item[N2 Upper/Lower Layer] (kg): total nitrogen mass in the upper (or lower) layer in the current compartment
\item[O2 Upper/Lower Layer] (kg): total oxygen mass in the upper (or lower) layer in the current compartment
\item[CO2 Upper/Lower Layer] (kg):  total carbon dioxide mass in the upper (or lower) layer in the current compartment
\item[CO Upper/Lower Layer] (kg):  total carbon monoxide mass in the upper (or lower) layer in the current compartment (multiply by 10\,000 to convert to ppm)
\item[HCN Upper/Lower Layer] (kg):  total HCN mass in the upper (or lower) layer in the current compartment (multiply by 10\,000 to convert to ppm)
\item[HCL Upper/Lower Layer] (kg):  total HCl mass in the upper (or lower) layer in the current compartment (multiply by 10\,000 to convert to ppm)
\item[H2O Upper/Lower Layer] (kg):  total water vapor mass in the upper (or lower) layer in the current compartment
\item[Optical Density Upper/Lower Layer] (kg):  total soot mass in the upper (or lower) layer in the current compartment
\item[Optical Density Upper/Lower Layer] (kg):  total soot mass from flaming-generated smoke in the upper (or lower) layer in the current compartment
\item[Optical Density Upper/Lower Layer] (kg):  total soot mass from smoldering-generated smoke in the upper (or lower) layer in the current compartment
\item[Trace Species Upper/Lower Layer] (kg):  total mass of trace species in the upper (or lower) layer in the current compartment
\end{description}

\subsection{Vent Flow (project\_vents.csv)}

The entries in this file pertain to the flow vents such as windows/doors, ceiling/floor vents, mechanical vents, and compartment leakage.
\begin{description}
\item[Time] (s)
\item[Net Inflow] (kg/s): net mass flow into the current compartment through the current horizontal flow (door/windows) vent connected to the current compartment.
\item[Opening Fraction]: fraction the vent is open. Zero indicated a fully-closed vent; one indicates a fully-opened vent.
\item[Trace Species Flow] (kg): total mass of trace species through this vent up to the current time (mechanical vents only).
\item[Trace Species Flow] (kg): total mass of trace species removed by a specified filter in this vent up to the current time (for mechanical vents only).
\item[Net Inflow] (kg/s): net mass flow into the current compartment via wall and floor leakage (for each compartment that includes leakage).
\end{description}

\subsection{Compartment Surface Temperature(project\_walls.csv)}

This file provides information on compartment surface temperatures
\begin{description}
\item[Time] (s)
\item[Ceiling Temperature] (\degc): temperature of the ceiling surface in the current compartment
\item[Upper Wall Temperature] (\degc): temperature of the wall surface adjacent to the upper layer in the current compartment
\item[Lower Wall Temperature] (\degc): temperature of the  wall surface adjacent to the lower layer in the current compartment
\item[Floor Temperature] (\degc): temperature of the floor surface in the current compartment
\end{description}

\newpage

\section{Error Messages}

In some (hopefully rare) cases, a simulation will fail to complete. In those cases, an error message provides guidance to the user on possible reasons for the failure. The message will contain an error number which provides a reference to additional information from the table below. Most often, these errors result from improper information in the input data files.
During initialization of the program for a simulation, CFAST may stop with an error message if the simulation cannot be initialized due to a missing or incorrect file specification. The error codes are as follows:
\begin{description}
\item[100] program called with no arguments (no input file)
\item[101] internal error in fire input; code for a free burning fire should not be reachable
\item[102] project file does not exist
\item[103] total file name length including path  is more than 256 characters
\item[104] one of the output files is not accessible (for example, if a CFAST case with this name is already running)
\item[105] error writing to an output file (openoutputfiles)
\item[106] a system fault has occurred. Applies to all open/close pairs once the model is running
\item[107] incompatible options
\item[108] not currently used
\item[109] cannot find/open a file
\item[110] error in handling the status input/output
\end{description}
Error codes from 1 to 99 are from the routine which parses the input and will be reported in the .log file.  The first set indicates a command with the wrong number of arguments. These errors indicate an error in a particular input command as follows:
\begin{description}
\item[1] TIMES command
\item[2] TAMB command
\item[3] EAMB command
\item[4] LIMO2 command
\item[5] THERMAL or FIRE commands
\item[7] MAINF command
\item[8] COMPA command
\item[10] HVENT command
\item[11] EVENT command
\item[12] MVENT command
\item[23] VVENT command
\item[24] WIND command
\item[25] INTER command
\item[26] MVOPN command
\item[28] MVDCT command
\item[29] MVFAN command
\item[32] OBJECT command
\item[34] CJET and DETEC command
\item[35] STPMAX command
\item[37] VHEAT command
\item[39] ONEZ command
\item[41] TARGE command
\item[46] HALL command
\item[47] ROOMA command
\item[51] ROOMH command
\item[55] DTCHE command
\item[56] SETP command
\item[58] HHEAT command
\item[65] HEATF command
\end{description}
The second set of errors related to parsing the input indicate specific errors with a command as follows:
\begin{description}
\item[9, compa] Compartment out of range
\item[26, inter] Not a defined compartment
\item[27, mvopn] Specified node number too large for this system
\item[30, mvfan] Fan curve has incorrect specification
\item[31, mvfan] Exceeded allowed number of fans
\item[33, object] Object must be assigned to an existing compartment
\item[35, detect] Invalid DETECTOR specification
\item[36, detect] A referenced compartment is not yet defined
\item[38, vheat] VHEAT has specified a non-existent compartment
\item[42, target] Too many targets are being defined
\item[43, target] The compartment specified by TARGET does not exist
\item[44, target] Invalid TARGET solution method specified
\item[45,  target] Invalid equation type specified in TARGET
\item[49, rooma] Compartment specified by ROOMA does not exist
\item[52, roomh] Compartment specified by ROOMH is not defined
\item[53, roomh] ROOMH error on data line
\item[54, roomh] Data on the ROOMA (or H) line must be positive
\item[57, setp] Trying to reset the SETP parameters
\item[61, hheat] HHEAT specification error in compartment pairs
\item[62, hheat] Error in fraction for HHEAT
\item[63, object] Fire type out of range
\item[64, object] The fire must be assigned to an existing compartment
\item[66, heatf] The heat source must be assigned to an existing compartment
\item[67, mvent] Compartment has not been defined
\item[68, mvent] Exceed one of the array bounds, ierror=68 (external), 69 (internal)  and 70 (fan)
\item[71, event] Undefined vent type
\item[72, inter] Specification for interface height is outside of allowable range
\item[73, inter] Compartments must be defined in pairs
\item[74, setp] The requested “SETP” command does not exists
\item[75, setp] Incorrect file reference
\item[76, setp] Cannot read the parameter file
\item[77, setp] Unsupported parameter
\end{description}
Errors 400 and above are failures while the model is running. 610 through 685 are failures in the numerical routines; these are rarely seen, but typically result from an internal error in the model.



%\chapter{Summary and Conclusions}

How to best quantify the comparisons between model predictions and experiments is not obvious. The necessary and perceived level of agreement for any variable is dependent upon both the typical use of the variable in a given simulation, the nature of the experiment, and the context of the comparison in relation to other comparisons being made. For instance, the user may be interested in the time it takes to reach a certain temperature in the room, but have little or no interest in peak temperature for experiments that quickly reach a steady-state value. Insufficient experimental data and understanding of how to compare the numerous variables in a complex fire model prevent a complete validation of the model. 

A true validation of a model would involve proper statistical treatment of all the inputs and outputs of the model with appropriate experimental data to allow comparisons over the full range of the model. Thus, the comparisons of the differences between model predictions and experimental data discussed here are intentionally simple and vary from test to test and from variable to variable due to the changing nature of the tests and typical use of different variables.

Table \ref{tab:Summary_Relative_Diffs} summarizes the comparisons in this report.

\begin{table}
\begin{center}
\caption{Summary of Model Comparisons}
\label{tab:Summary_Relative_Diffs}
\vspace{0.1in}
\begin{tabular*}{1.0\textwidth}{@{\extracolsep{\fill}} | l | c | c | c | c |}
\hline
Quantity & Average & Median & Within & 90th \\
& Difference$^{a}$ &Difference$^b$ & Experimental & Percentile$^d$ \\
& & & Uncertainty$^c$ & \\
& (\%) & (\%) & (\%) & (\%) \\
\hline
HGL Temperature & 6 &  14 &  52 &  30  \\ \hline
HGL Depth & 3 & 15 & 40 & 28 \\ \hline
Plume Temperature & 15 & 12 & 54 & 25 \\ \hline
Ceiling Jet Temperature & 16 & 5 & 70 & 61 \\ \hline
Oxygen Concentration & -6 & 18 & 12 & 32 \\ \hline
Carbon Dioxide Concentration & -16 & 16 & 21 & 52 \\ \hline
Smoke Obscuration$^e$ & 272/22 & 227/18 & 0/82 & 499/40 \\ \hline
Pressure & 43 & 13 & 77 & 206$^f$ \\ \hline
Target Flux (Total) & -23 & 27 & 42 & 51 \\ \hline
Target Temperature & 0 & 18 & 38 & 34 \\ \hline
Surface Flux (Total) & 5 & 25 & 40 & 61 \\ \hline
Surface Temperature & 24 & 35 & 17 & 76 \\ \hline
\end{tabular*}  
\end{center}
a - average difference includes both the sign and magnitude of the relative differences in order to show any general trend to over- or under-prediction. \\
b - median difference is based only on the magnitude of the relative differences and ignores the sign of the relative differences so that values with opposing signs do not cancel and make the comparison appear closer than individual magnitudes would indicate. \\
c - the percentage of model predictions that are within experimental uncertainty. \\
d - 90 \% of the model predictions are within the stated percentage of experimental values. For reference, a difference of 100~\% is a factor of 2 larger or smaller than experimental values. \\
e - the first number is for the closed door NIST/NRC tests and the second number if for the open door NIST/NRC tests. \\
f - high magnitude of the 90th percentile value driven in large part by two tests where under-prediction was approximately 2 Pa.
\end{table}

For five of the quantities,  the physics of the model is appropriate to represent the experimental conditions, and the calculated relative differences comparing the model and the experimental values are consistent with the combined experimental and input uncertainty.  A few notes on the comparisons are appropriate:

\begin{itemize}
\item The CFAST predictions of the HGL temperature and height are, with a few exceptions, within or close to experimental uncertainty.  The CFAST predictions are typical of those found in other studies where the HGL temperature is typically somewhat over-predicted and HGL height somewhat lower (HGL depth somewhat thicker) than experimental measurements.  Still, predictions are mostly within 10~\% to 20~\% of experimental measurements.  Calculation of HGL temperature and height has higher uncertainty in rooms remote from the fire (compared to those in the fire compartment).
\item For most of the comparisons, CFAST predicts ceiling jet temperature well within experimental uncertainty.  For cases where the HGL temperature is below 70 \degc, significant and consistent over-prediction was observed.
\item CFAST predicts the flame height consistent with visual observations of flame height for the experiments.  This is not surprising, given that CFAST simply uses a well-characterized experimental correlation to calculate flame height.
\item Gas concentrations are typically under-predicted by CFAST, with an average difference of -6~\% for oxygen concentration and -16~\% for carbon dioxide concentration. 
\item Compartment pressure predicted by CFAST are within or close to experimental uncertainty for most tests.
\end{itemize}


Four of the quantities were seen to require additional care when using the model to evaluate the given quantity.  This typically indicates limitations in the use of the model.  A few notes on the comparisons are appropriate:

\begin{itemize}
\item CFAST typically predicts plume temperature near to experimental uncertainty, but tends to under-predict temperatures nearer to the fire source and over-predict temperatures farther away.
\item CFAST typically over-predicts smoke concentration.  Predicted concentrations for open-door tests are within experimental uncertainties, but those for closed-door tests are far higher.
\item With exceptions, CFAST predicts cable surface temperatures within experimental uncertainties.  Total heat flux to targets is typically predicted to within about 30~\%, and often under-predicted.  Radiative heat flux to targets is typically over-predicted compared to experimental measurements, with higher relative difference values for closed-door tests.  Care should be taken in predicting localized conditions (such as target temperature and heat flux) because of inherent limitations in all zone fire models.
\item Predictions of compartment surface temperature and heat flux are typically within 10~\% to 30~\%.  Generally, CFAST over-predicts the far-field fluxes and temperatures and under-predicts the near-field measurements.  This is consistent with the single representative layer temperature assumed by zone fire models.
\end{itemize}

CFAST predictions in this validation study were consistent with numerous earlier studies, which show that the use of the model is appropriate in a range of fire scenarios.  The CFAST model has been subjected to extensive evaluation studies by NIST and others.  Although differences between the model and the experiments were evident in these studies, most differences can be explained by limitations of the model as well as of the experiments.  Like all predictive models, the best predictions come with a clear understanding of the limitations of the model and the inputs provided to perform the calculations.

%\backmatter


\bibliography{../Bibliography/CFAST_refs}
\addcontentsline{toc}{chapter}{References}

\appendix
\addcontentsline{toc}{chapter}{Appendices}

%\chapter[Appendix A:  Calculation of Layer Height and the Average Upper and Lower Layer Temperatures]{Appendix A:  \\
\vspace{16 pt}
Calculation of Layer Height and the Average Upper and Lower Layer Temperatures}
\label{Appendix_layerheight}

Fire protection engineers often need to estimate the location of the
interface between the hot, smoke-laden upper layer and the cooler
lower layer in a burning compartment.  Relatively simple fire models,
often referred to as {\em two-zone models}, compute this quantity
directly, along with the average temperature of the upper and lower
layers.  In a computational fluid dynamics (CFD) model like FDS, there
are not two distinct zones, but rather a continuous profile of
temperature. Nevertheless, there are methods that have been developed
to estimate layer height and average temperatures from a continuous
vertical profile of temperature. One such
method~\cite{Janssens:1992} is as follows: Consider a continuous
function $T(z)$ defining temperature $T$ as a function of height above
the floor $z$, where $z=0$ is the floor and $z=H$ is the
ceiling. Define $T_u$ as the upper layer temperature, $T_l$ as the
lower layer temperature, and $z_{int}$ as the interface
height. Compute the quantities:
\begin{eqnarray*} (H-z_{int})\; T_u + z_{int} \; T_l = \int_0^H \; T(z) \; dz &=& I_1 \\
                  (H-z_{int})\; \frac{1}{T_u} + z_{int} \; \frac{1}{T_l} = \int_0^H \; \frac{1}{T(z)} \; dz &=& I_2 \end{eqnarray*}
Solve for $z_{int}$:
\be z_{int} = \frac{ T_l(I_1 \, I_2 - H^2)}{I_1+I_2 \, T_l^2 - 2\, T_l \, H} \ee
Let $T_l$ be the temperature in the lowest mesh cell and, using
Simpson's Rule, perform the numerical integration of $I_1$ and
$I_2$. $T_u$ is defined as the average upper layer temperature via
\be (H-z_{int})\; T_u = \int_{z_{int}}^H \; T(z) \; dz \ee
Further discussion of similar procedures can be found in Ref.~\cite{He:1998}.



\clearpage

\end{document}
