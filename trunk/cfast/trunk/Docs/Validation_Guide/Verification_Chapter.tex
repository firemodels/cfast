
\chapter{Verification}

The terms {\em verification} and {\em validation} are often used interchangeably to mean the process of checking the accuracy of a numerical model. For many, this entails comparing model predictions with experimental measurements. However, there is now a fairly broad-based consensus that comparing model and experiment is largely what is considered {\em validation}. So what is {\em verification}? ASTM~E~1355~\cite{ASTM:E1355}, ``Standard Guide for Evaluating the Predictive Capability of Deterministic Fire Models,'' defines verification as
\begin{quote}
The process of determining that the implementation of a calculation method accurately represents the developer's conceptual description of the calculation method and the solution to the calculation method.
\end{quote}
and it defines validation as
\begin{quote}
The process of determining the degree to which a calculation method is an accurate representation of the real world from the perspective of the intended uses of the calculation method.
\end{quote}
Simply put, verification is a check of the math; validation is a check of the physics. If the model predictions closely match the results of experiments, using whatever metric is appropriate, it is assumed by most that the model suitably describes, via its mathematical equations, what is happening. It is also assumed that the solution of these equations must be correct. So why do we need to perform model verification? Why not just skip to validation and be done with it? The reason is that rarely do model and measurement agree so well in all applications that anyone would just accept its results unquestionably. Because there is inevitably differences between model and experiment, we need to know if these differences are due to limitations or errors in the numerical solution, or the physical sub-models, or both.

Whereas model validation consists mainly of comparing predictions with experimental measurements, as documented later in this guide, model verification consists of a much broader range of activities, from checking the computer program itself to comparing calculations to analytical (exact) solutions to understanding the impact on model outputs from a range of different model inputs.


\section{Thermal Equlibrium}

For most of the examples presented in this section, the same basic geometry is used, a single 5~m by 5~m by 5~m compartment.

\subsection{Temperature Equilibrium via Heat Conduction}

As a simple test of the energy balance, raising the external temperature of the base case compartment from an initial condition of 20~\degc to 25~\degc allows the internal temperature to equilibrate to the exterior. From the ideal gas law, the pressure inside the compartment is expected to rise to
\begin{equation}
   P_{\rm final} = P_{\rm initial} \; \frac{T_{\rm final}}{T_{\rm initial}} = 101325 \; {\rm Pa} \times \frac{298.15 \; {\rm K}}{293.15 \; {\rm K}} = 103027.78 \; {\rm Pa} \label{eq:Temperature_Equilibrium}
\end{equation}
or a pressure rise of 1728.21, matching the output from CFAST.  Figure \ref{fig:Temperature_Equilibrium} shows the simulated conditions for this test.

\begin{figure}[!ht]
\begin{tabular*}{\textwidth}{l@{\extracolsep{\fill}}r}
\includegraphics[width=3.0in]{FIGURES/Verification/basic_tempequilib_temp} &
\includegraphics[width=3.0in]{FIGURES/Verification/basic_tempequilib_pres}
\end{tabular*}
\caption[Results of the test case {\ct basic\_tempequilib.in}]{Interior temperature and pressure in equilibrium with exterior in the case {\ct basic\_tempequilib.in}.}
\label{fig:Temperature_Equilibrium}
\end{figure}

\subsection{Temperature Equilibrium via a Window}

Now an open window is added to the compartment, with an with an exterior temperature of 25~\degc. Figure~\ref{fig:Temperature_Equilibrium_With_Window} shows the interior conditions coming into equilibrium with the exterior.

\begin{figure}[!ht]
\begin{tabular*}{\textwidth}{l@{\extracolsep{\fill}}r}
\includegraphics[width=3.0in]{FIGURES/Verification/basic_tempequilib_window_temp} &
\includegraphics[width=3.0in]{FIGURES/Verification/basic_tempequilib_window_pres}
\end{tabular*}
\caption[Results of the test case {\ct basic\_tempequilib\_window.in}]{Interior temperature and pressure in equilibrium with exterior in the case {\ct basic\_tempequilib\_window.in}.}
\label{fig:Temperature_Equilibrium_With_Window}
\end{figure}

\subsection{Temperature Equilibrium via a Window at a High Elevation}

With the exterior temperature still set to 25~\degc, the elevation is raised to 1500~m, approximately the average elevation of Idaho.  Since CFAST calculations are relative to the exterior ambient, conditions are expected to be identical to the previous examples and equilibrate to those of the exterior. Figure \ref{fig:Temperature_Equilibrium_Elevation} shows the simulated conditions for the test case.

\begin{figure}[!ht]
\begin{tabular*}{\textwidth}{l@{\extracolsep{\fill}}r}
\includegraphics[width=3.0in]{FIGURES/Verification/basic_tempequilib_window_elevation_temp} &
\includegraphics[width=3.0in]{FIGURES/Verification/basic_tempequilib_window_elevation_pres}
\end{tabular*}
\caption[Results of the test case {\ct basic\_tempequilib\_window\_elevation.in}]{Interior temperature and pressure in equilibrium with exterior in the case {\ct basic\_tempequilib\_window\_elevation.in}.}
\label{fig:Temperature_Equilibrium_Elevation}
\end{figure}


\clearpage

\section{Mass Balance}

\label{mass_conservation}
\subsection{A Fire in a Single, Sealed Compartment}
\label{sec:spec1}
A natural gas fire burns in a sealed compartment of dimension 5~m by 6~m by 3~m. The heat release rate ramps up linearly to 1~kW in 30~s, then remains steady for 5~min, and then ramps down linearly to 0 in 30~s. The total energy released is 330~kJ, and the total mass of fuel consumed is
\begin{equation}
  \frac{ 330 \; {\rm kJ} }{ 50000 \; {\rm kJ/kg} } = 0.0066 \; {\rm kg}
\end{equation}
The combustion chemistry is given by
\begin{equation}
   \mathrm{CH_4 + 2 \, O_2 \to CO_2 + 2 \, H_2O}
\end{equation}
The molecular weight of CH$_4$ is 16~g/mol and CO$_2$ is 44~g/mol; thus, the mass of CO$_2$ produced by the fire is
\begin{equation}
   m_{\rm CO_2} = 0.0066 \; {\rm kg} \times  \frac{ 44 \; {\rm g/mol} }{ 16 \; {\rm g/mol} } = 0.01815 \; {\rm kg}
\end{equation}
The molecular weight of H$_2$O is 18~g/mol; thus, the mass of H$_2$O produced by the fire is
\begin{equation}
   m_{\rm H_2O} = 0.0066 \; {\rm kg} \times  \frac{ 2(18) \; {\rm g/mol} }{ 16 \; {\rm g/mol} } = 0.01485 \; {\rm kg}
\end{equation}
CFAST predicts that the mole fractions of O$_2$, CO$_2$ and H$_2$O in the upper layer are 0.2069, 0.00012 and 0.00024, respectively. The remainder is N$_2$, whose mole fraction is 0.7927. These mole fractions can be converted to mass fractions by
\begin{equation}
Y_k = \frac{X_{k} M_{k}}{\sum_{i=1}^N X_{i}M_{i}}
\end{equation}
The mass of the upper layer can be calculated from the equation of state:
\begin{equation}
m_{\rm u} = \frac{P \, V}{R \, T} \quad ; \quad R = \frac{\gamma-1}{\gamma} \, c_p \approx 289.14 \; {\rm  \frac{J}{kg \cdot K}}
\end{equation}
The mass of CO$_2$ and H$_2$O produced is given by
\begin{equation}
m_{k} = m_{\rm u} \, Y_{k}
\end{equation}
Figure~\ref{specmass1} shows the resulting product masses.

\begin{figure}[!ht]
\centering
\includegraphics[width=3.0in]{FIGURES/Verification/species_mass_1}
\caption[Results of the test case {\ct species\_mass\_1.in}]{Expected and predicted masses of CO$_2$ and H$_2$O for the case {\ct species\_mass\_1.in}.}
\label{specmass1}
\end{figure}


\subsection{A Fire in a Compartment Connected to Another via a Door}
\label{sec:spec2}

The same natural gas fire described in Section~\ref{sec:spec1} burns in a compartment of dimension 2~m by 5~m by 8~m which is connected to another compartment of dimension 5~m by 3~m by 8~m. A doorway connects the compartments, which has a width of 1~m and a height of 6~m. Because the fire and the fuel source have not changed, the theoretical calculations for the mass of CO$_2$ and H$_2$O produced will remain the same. The remaining portion of the problem is approached in the same manner, but since there are two compartments, the mass of CO$_2$ and H$_2$O produced in each layer of each compartment must be individually calculated and then summed together to produce the net yields of CO$_2$ and H$_2$O. Figure~\ref{specmass2} shows the calculated mass of both CO$_2$ and H$_2$O.
\begin{figure}[!ht]
\centering
\includegraphics[width=3.0in]{FIGURES/Verification/species_mass_2}
\caption[Results of the test case {\ct species\_mass\_1.in}]{Expected and predicted masses of CO$_2$ and H$_2$O for the case {\ct species\_mass\_2.in}.}
\label{specmass2}
\end{figure}

\subsection{A Fire in a Compartment Connected to Another via a Ceiling Vent}

The same natural gas fire described in Section~\ref{sec:spec1} burns in a compartment of dimension 9~m by 5~m by 4~m which is connected to another compartment of dimension 9~m by 5~m by 2~m. The compartments are placed such that the second one is located directly above the first one. There is a square ceiling vent between the compartments that has an area of 4~m$^2$. This problem is approached in the same exact manner as in Section~\ref{sec:spec2} because the only difference between the two scenarios is the specific alignment of the compartments.
Figure~\ref{specmass2} shows the calculated mass of both CO$_2$ and H$_2$O.
\begin{figure}[!ht]
\centering
\includegraphics[width=3.0in]{FIGURES/Verification/species_mass_3}
\caption[Results of the test case {\ct species\_mass\_3.in}]{Expected and predicted masses of CO$_2$ and H$_2$O for the case {\ct species\_mass\_3.in}.}
\label{specmass3}
\end{figure}

\subsection{A Fire in a Four Compartment Assembly}
\label{sec:specmass4}

Four 4~m by 4~m by 4~m compartments are arranged such that two compartments are placed adjacent to one another and the following two compartments are placed directly on top of the first two. The same natural gas fire described in Section~\ref{sec:spec1} burns in the first compartment of this setup. After 2500~s, the wall between compartments one and two is removed, forcing the gases in the two rooms to mix. Next, at 5000~s, the wall between compartments three and four is removed. Lastly, at 7500~s, the ceiling of compartment four is removed, allowing the system to slowly return to ambient conditions. Figure~\ref{fig:specmass4ab} and Figure~\ref{fig:specmass4cd} show how the masses of CO$_2$ and H$_2$O in each compartment change with respect to time. Figure~\ref{fig:specmass4TP} shows comparisons of the expected temperature and pressure values of compartment one with the values produced by CFAST.



\begin{figure}[!ht]
\begin{tabular*}{\textwidth}{l@{\extracolsep{\fill}}r}
\includegraphics[width=3.0in]{FIGURES/Verification/species_mass_4a} &
\includegraphics[width=3.0in]{FIGURES/Verification/species_mass_4b} 
\end{tabular*}
\caption[Results of the test case {\ct species\_mass\_4.in}]{Expected and CFAST calculated values for the masses of CO$_2$ and H$_2$O in compartments one and two for case {\ct species\_mass\_4.in}.}
\label{fig:specmass4ab}
\end{figure}

\begin{figure}[!ht]
\begin{tabular*}{\textwidth}{l@{\extracolsep{\fill}}r}
\includegraphics[width=3.0in]{FIGURES/Verification/species_mass_4c} &
\includegraphics[width=3.0in]{FIGURES/Verification/species_mass_4d} 
\end{tabular*}
\caption[Results of the test case {\ct species\_mass\_4.in}]{Expected and CFAST calculated values for the masses of CO$_2$ and H$_2$O in compartments three and four for case {\ct species\_mass\_4.in}.}
\label{fig:specmass4cd}
\end{figure}

\begin{figure}[!ht]
\begin{tabular*}{\textwidth}{l@{\extracolsep{\fill}}r}
\includegraphics[width=3.0in]{FIGURES/Verification/species_mass_4_temperature} &
\includegraphics[width=3.0in]{FIGURES/Verification/species_mass_4_pressure}
\end{tabular*}
\caption[Results of the test case {\ct species\_mass\_4.in}]{Expected and CFAST calculated values for pressure and temperature of the first compartment for the case {\ct species\_mass\_4.in}.}
\label{fig:specmass4TP}
\end{figure}

\clearpage


\section{Energy Balance}

\subsection{A Fire in a Single, Sealed Compartment with a Single Zone}
\label{sealed_test}

A 100~kW natural gas (methane) fire burns in a sealed compartment with no ventilation, adiabatic walls, and no radiative emission. A single zone simulation is run in which it is assumed that the entire volume is taken up by the upper layer.  The governing equation for the pressure and temperature of the single zone compartment are:
\begin{eqnarray}
   \frac{{\rm d} P}{{\rm d}t} &=& \frac{\gamma-1}{V} \, \dot{h} \quad ; \quad \dot{h} = \dot{Q} + c_p \, \dot{m}_{\rm f} \, T_\infty \\[0.1in]
   \frac{{\rm d} T}{{\rm d}t} &=& \frac{1}{c_p \, m} \left( \dot{h} - c_p \, \dot{m}_{\rm f} \, T + V \, \frac{{\rm d} P}{{\rm d}t} \right)
\end{eqnarray}
where $\dm_{\rm f}$ is the fuel flow rate and $m$ is the total mass. Figure~\ref{fig:Analytical_Closed_Compartment} includes comparisons of the temperature and pressure as predicted by CFAST and a simple second-order accurate ODE solver of the equations above.
\begin{figure}[!ht]
\begin{tabular*}{\textwidth}{l@{\extracolsep{\fill}}r}
\includegraphics[width=3.0in]{FIGURES/Verification/sealed_test_temp} &
\includegraphics[width=3.0in]{FIGURES/Verification/sealed_test_pres}
\end{tabular*}
\caption[Results of the test case {\ct sealed\_test.in}]{Temperature and pressure rise due to a fire in a closed compartment. The case is called {\ct sealed\_test.in}.}
\label{fig:Analytical_Closed_Compartment}
\end{figure}

\subsection{A Fire in a Single, Sealed Compartment with Two Zones}

Consider now the same case as in Section~\ref{sealed_test}, but now with two zones rather than one. The compartment pressure ought to be the same as before, and the upper layer temperature ought to converge to the single layer temperature (Fig.~\ref{fig:Analytical_Closed_Compartment2}).
\begin{figure}[!ht]
\begin{tabular*}{\textwidth}{l@{\extracolsep{\fill}}r}
\includegraphics[width=3.0in]{FIGURES/Verification/sealed_test_2_layers_temp} &
\includegraphics[width=3.0in]{FIGURES/Verification/sealed_test_2_layers_pres}
\end{tabular*}
\caption[Results of the test case {\ct sealed\_test\_2\_layers.in}]{Hot gas layer temperature and pressure rise due to a fire in a closed compartment with two zones. The case is called {\ct sealed\_test\_2\_layers.in}.}
\label{fig:Analytical_Closed_Compartment2}
\end{figure}

\clearpage


\section{Ventilation}


\subsection{Temperature and Pressure Changes of Steady-State Air Flow}
\label{sec:vent2}

A similar setup to the one used in Section~\ref{sec:specmass4} is employed here. A fan blows air into the first compartment, on the ground floor, at a rate of 1~m$^3$/s and then continues to the second compartment after passing through a doorway. The air then travels to the third compartment by passing through a ceiling vent with an area of 3~m$^2$ and continues to compartment four through another doorway. Air is finally extracted from compartment four at a rate of 1~m$^3$/s. Figure~\ref{ventilation_2TP} shows how temperature and pressure change as the air flows through the fourth compartment.

\begin{figure}[!ht]
\begin{tabular*}{\textwidth}{l@{\extracolsep{\fill}}r}
\includegraphics[width=3.0in]{FIGURES/Verification/ventilation_2_temperature} &
\includegraphics[width=3.0in]{FIGURES/Verification/ventilation_2_pressure}
\end{tabular*}
\caption[Results of the test case {\ct ventilation\_2.in}]{Expected and CFAST calculated values for pressure and temperature of the fourth compartment for the case {\ct ventilation\_2.in}.}
\label{ventilation_2TP}
\end{figure}


\subsection{Temperature and Pressure Changes of Non Steady-State Air Flow}

The same setup that was constructed in Section~\ref{sec:vent2} is used again in this scenario. All vents and fans are the same as in the previous case, except the fan in the fourth compartment has been converted to a round ceiling vent with an area of 4~m$^2$. Initially, all of the vents are closed and air is pumped into the first compartment at a rate of 1~m$^3$/s, for a period of 15~s. After 15~s pass, the fan is shut off. Then, at 200~s, the door between compartments one and two opens completely. Next, at 500~s, the vent between compartments two and three opens completely. At 700~s, the door between compartments three and four opens completely. Finally, at 1000~s, the ceiling vent in compartment four opens completely. Figure~\ref{ventilation_3TP} shows how temperature and pressure change as the air flows through the first compartment.

\begin{figure}[!ht]
\begin{tabular*}{\textwidth}{l@{\extracolsep{\fill}}r}
\includegraphics[width=3.0in]{FIGURES/Verification/ventilation_3_temperature} &
\includegraphics[width=3.0in]{FIGURES/Verification/ventilation_3_pressure}
\end{tabular*}
\caption[Results of the test case {\ct ventilation\_3.in}]{Expected and CFAST calculated values for pressure and temperature of the first compartment for the case {\ct ventilation\_3.in}.}
\label{ventilation_3TP}
\end{figure}


